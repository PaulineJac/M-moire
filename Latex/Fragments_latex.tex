\documentclass[12pt,onecolumn,twoside,a4paper]{memoir}
\usepackage[no-math]{fontspec}
\usepackage{ebgaramond}
\usepackage{graphicx}
\usepackage{amssymb}
\usepackage{xcolor}
\usepackage{polyglossia}% typographie française
\setdefaultlanguage{french}
\usepackage{microtype}
\usepackage{xspace}% gérer les espaces
\usepackage{amssymb}% symboles utiles (math, etc.)
\usepackage{ulem} % souligner
\usepackage{alltt} % environnement "télétype"
\usepackage{setspace} % réglage des interlignes etc.
\usepackage{fancyhdr} % hauts et pieds de page
\usepackage{multicol} % texte sur plusieurs colonnes
\usepackage{fancybox} % boîtes améliorées
\usepackage{array} % tableaux
\usepackage{multirow} % les tableaux améliorés
\usepackage{url} % écrire les url
\usepackage{appendix} % annexes améliorées
 \usepackage{geometry}
\usepackage{csquotes}
\usepackage{hyperref}
\usepackage{ragged2e}
\usepackage[backend=biber, bibstyle=verbose, backref=false, hyperref=false, citestyle=authortitle-ibid]{biblatex}
\DeclareFieldFormat{title}{\textit{#1}}
\DeclareFieldFormat{booktitle}{\textit{#1}}
\DeclareFieldFormat{journaltitle}{\textit{#1}}
\addbibresource{Biblio.bib}
%%%%%%%%
%%%%%%%%
%%%%%%%%%%%%%%%%
\usepackage{calc} 
\usepackage{fourier-orns} 
%%%% voir http://jacques-andre.fr/fontex/Fourier-orn.pdf
%%%%%%%
\pagestyle{fancyplain} \renewcommand{\chaptermark}[1]{% 
	\markboth{\chaptername\ \thechapter.\ #1}% 
	{\chaptername\ \thechapter.\ #1}} \renewcommand{\sectionmark}[1]{% 
	\markright{\thesection\ #1}}
% 
\lhead[\fancyplain{}{\bfseries\thepage}]%
{\fancyplain{}{\bfseries\nouppercase{\leftmark}}} 
\rhead[\fancyplain{}{\bfseries\nouppercase{\rightmark}}]%
{\fancyplain{}{\bfseries\thepage}} \fancyfoot{}
%%%%%%%%%%%%
%%%%%%%%%%%%
\newcommand{\phipaireblanche}
{\newpage{\pagestyle{empty}\cleardoublepage}} \newcommand{\phietc}{\textit{etc.}\xspace}
\newcommand{\phigg}[1]{\og #1 \fg}
%%%%%%%
\makeatletter
\newcommand{\finirphiimpaire}{\clearpage\if@twoside \ifodd\c@page
	\hbox{}\newpage\if@twocolumn\hbox{}\newpage\fi\fi\fi} \makeatother
\newcommand{\phiimpaireblanche}{% 
	\newpage{\pagestyle{empty}\finirphiimpaire}}
\renewcommand{\baselinestretch}{1,2}%%% interligne
\makeatother
%%%%%%%
%%%%%%
%Style Chapitre établi par Vincent Zoonekynd : http://zoonek.free.fr/LaTeX/LaTeX_samples_chapter/0.html
\makeatletter
\setlength\midchapskip{7pt}
\makechapterstyle{VZ21}{
	\renewcommand\chapnamefont{\Large\scshape}
	\renewcommand\chapnumfont{\Large\scshape\centering}
	\renewcommand\chaptitlefont{\huge\bfseries\centering}
	\renewcommand\printchaptertitle[1]{%
		\setlength\tabcolsep{7pt}% used as indentation on both sides
		\settowidth\@tempdimc{\chaptitlefont ##1}%
		\setlength\@tempdimc{\textwidth-\@tempdimc-2\tabcolsep}%
		\chaptitlefont
		\ifdim\@tempdimc > 0pt\relax% one line
		\begin{tabular}{c}
			\toprule  ##1\\ \bottomrule
		\end{tabular}
		\else% two+ lines
		\begin{tabular}{%
				>{\chaptitlefont\arraybackslash}p{\textwidth-2\tabcolsep}}
			\toprule ##1\\ \bottomrule
		\end{tabular}
		\fi
	}
}
%%%%%%%%%%%%%%
%%%%% Pour la traduction %%%%%%%%
\usepackage[ widthliketwocolumns,
nocritical,
noeledsec,
noend,
nofamiliar,
noledgroup,
series={}
]{reledmac}
\usepackage{reledpar}
%%%%%%%
\setcounter{stanzaindentsrepetition}{1}
\setstanzaindents{0,0,0}
\AtEveryStopStanza{\vspace{1\baselineskip}}
\numberstanzatrue
\renewcommand{\thestanzaL}{\MakeUppercase{\roman{stanzaL}}}
%%%%%
\firstlinenum*{100000}
%%%%%%%
\setlength{\Lcolwidth}{.430\textwidth}
\setlength{\Rcolwidth}{.450\textwidth}
\columnsposition{C}
\setlength{\beforecolumnseparator}{0.035\textwidth}
\setlength{\aftercolumnseparator}{0.0001\textwidth}
\sidenotemargin{left}
%%%%% 
%%%%%%%%%%%%%%%%%
%%%%% 
\AtBeginDocument{\renewcommand{\abstractname}{}}
%%%%%%%%%%%%%%%%%
                \title{\textit{Corpus sélectionné des fragments de tragédies de l'époque républicaine}}
                \author{}
                \date{\today}
                \begin{document}
            
         
            \section{Livius Andronicus}
            Texte latin établi et organisé selon les choix de reconstruction et d'édition de \cite{TrRF_I_2012}.\par
            
               \subsection*{Égisthe}
               \begin{abstract}
                  Source hellène : Sophocle \textit{Αἴγισθος}, Eschyle
                        \textit{Όρεστεια}.\par
                   L'intrigue est proche de l'\textit{Άγαμέμνων} d'Eschyle
                     et de la version latine de Sénèque: elle se concentre donc sur l'assassinat
                     d'Agamemnon par Égisthe et Clytemnestre.\par
               \end{abstract}
               \begin{pairs}
                  \begin{Leftside}
			\beginnumbering
			\setcounter{stanzaL}{0}
                     
                         \stanza \ledsidenote{\color{gray}{\textbf{Non.512\vspace{-0,5cm}}}}Nam
                              ut
                              Pergama & 
                              accensa
                              et
                              præda
                              per
                              participes
                              æquiter & 
                     partita
                              est. \&
                         \stanza \ledsidenote{\color{gray}{\textbf{Non.335,21\vspace{-0,5cm}}}}Tum
                              autem
                              lasciuum
                              Nerei
                              simum
                              pecus & 
                     ludens
                              ad
                              cantum
                              classem
                              lustratur... \&
                         \stanza \ledsidenote{\color{gray}{\textbf{Non.166,29\vspace{-0,5cm}}}}
                     Nemo
                              hæce
                              uostrum
                              ruminetur
                              mulieri. \&
                         \stanza \ledsidenote{\color{gray}{\textbf{Non.176,13\vspace{-0,5cm}}}}
                     ...Sollemnitusque
                              {Sollemnitusque}
                              deo
                              litat
                              laudem
                              lubens. \&
                         \stanza \ledsidenote{\color{gray}{\textbf{Non.127,30\vspace{-0,5cm}}}}... in
                              sedes
                              conlocat
                              se
                              regias: & 
                     Clytemnestra
                              iuxtim
                              ,
                              tertias
                              natæ
                              occupant. \&
                         \stanza \ledsidenote{\color{gray}{\textbf{Non.110,29\vspace{-0,5cm}}}}
                     Ipsus
                              se
                              in
                              terram
                              saucius
                              fligit
                              cadens. \&
                         \stanza \ledsidenote{\color{gray}{\textbf{Non.23,25\vspace{-0,5cm}}}}Quin
                              quod
                              parere
                              mihi
                              uos
                              maiestas
                              mea & 
                     procat
                              ,
                              toleratis
                              temploque
                              {temploque}
                              hanc
                              deducitis
                              ? \&
                         \stanza \ledsidenote{\color{gray}{\textbf{Non.132,32 386,23\vspace{-0,5cm}}}}
                     Iamne
                              {Iamne}
                              oculos
                              specie
                              lætauisti
                              optabili
                              ? \&
                     
                  \endnumbering
		\end{Leftside}
                  \begin{Rightside}
			\beginnumbering
			\numberstanzafalse
                     
                         \stanza En réalité, quand Pergame fût & enflammée on eût partagé équitablement & 
                     le butin entre les participants. \&
                         \stanza Alors, voici la horde enjouée de Nérée, avec son bec aplati, & 
                     qui tourne autour de la flotte en s’amusant des chants [des
                              marins]... \&
                         \stanza 
                     Que personne parmi vous ne rumine [ses pensées] à la femme. \&
                         \stanza 
                     Et la solennité extatique offre un éloge au dieu. \&
                         \stanza . . .il s’assied sur le trône royal : & 
                     à côté Clytemnestre, ses filles occupent les troisièmes [places]. \&
                         \stanza 
                     Le blessé se heurte lui-même en tombant à terre. \&
                         \stanza Comment? Vous ne supportez pas que ma majesté, & 
                     vous demande de m’obéir ? Pourquoi n’emmenez-vous pas celle-ci au
                              sanctuaire ? \&
                         \stanza 
                     N’as-tu pas désormais contenté tes yeux par cette vision convoitée
                              ? \&
                     
                  \endnumbering
		\end{Rightside}
               \end{pairs}
	\Columns
            
         
         
            \section{Gnaeus Nævius}
            Texte latin établi et organisé selon les choix de reconstruction et d'édition de \cite{SpaltenNaevius}.\par
            
               \subsection*{Danæ}
               \begin{abstract}
                  Sources hellènes : Eschyle \textit{Δικτυλκοί}; Sophocle
                        \textit{Ακρισιος}, \textit{Δαναη},
                     et \textit{Λαρισαῖοι}; Euripide \textit{Δαναη}.\par
                  Argument: cette tragédie aurait pour sujet la maternité de Danæ ; l’intrigue
                     pourrait inclure l’intrigue amoureuse entre Jupiter et Danæ, mais aussi son
                     accouchement et ses conséquences.\par
               \end{abstract}
               \begin{pairs}
                  \begin{Leftside}
			\beginnumbering
			\setcounter{stanzaL}{0}
                     
                         \stanza \ledsidenote{\color{gray}{\textbf{Non.470,1\vspace{-0,5cm}}}}
                     
                              Contemplo
                              placide
                              formam
                              et
                              faciem
                              virginis. \&
                         \stanza \ledsidenote{\color{gray}{\textbf{Non.186,23\vspace{-0,5cm}}}}
                     Omnes
                              formidant
                              homines
                              eius
                              valentiam
                              . \&
                         \stanza \ledsidenote{\color{gray}{\textbf{Non.262,25\vspace{-0,5cm}}}}
                     
                              Excidit
                              orationis
                              omnis
                              confidentia
                              . \&
                         \stanza \ledsidenote{\color{gray}{\textbf{Non.138,17\vspace{-0,5cm}}}}
                     Manubiæ
                              subpetant
                              pro
                              me \&
                         \stanza \ledsidenote{\color{gray}{\textbf{Non.110,17\vspace{-0,5cm}}}}
                     Suo
                              sonitu
                              claro
                              fulgorivit
                              Juppiter \&
                         \stanza \ledsidenote{\color{gray}{\textbf{Non.124,14\vspace{-0,5cm}}}}
                     .
                              .
                              .
                              .
                              quæ
                              quondam
                              fulmine
                              icit
                              Iuppiter \&
                         \stanza \ledsidenote{\color{gray}{\textbf{Non.456,25\vspace{-0,5cm}}}}
                     Eam
                              nunc
                              esse
                              inventam
                              probris
                              compotem
                              Scis
                              . \&
                         \stanza \ledsidenote{\color{gray}{\textbf{Non.306,7\vspace{-0,5cm}}}}
                     Desubito
                              famam
                              tollunt
                              ,
                              si
                              quam
                              solam
                              videre
                              in
                              via
                              . \&
                         \stanza \ledsidenote{\color{gray}{\textbf{Non.366,4\vspace{-0,5cm}}}}
                     Quin
                              ,
                              ut
                              quisque
                              est
                              meritus
                              ,
                              præsens
                              pretium
                              pro
                              factis
                              ferat. \&
                         \stanza \ledsidenote{\color{gray}{\textbf{Non.291,5\vspace{-0,5cm}}}}
                     .
                              .
                              .
                              indigne
                              exigor
                              patria
                              innocens
                              . \&
                         \stanza \ledsidenote{\color{gray}{\textbf{Non.504,1\vspace{-0,5cm}}}}
                     .
                              .
                              amnis
                              niveo
                              eo
                              fonte
                              lavere
                              me
                              memini
                              manum
                              . \&
                     
                  \endnumbering
		\end{Leftside}
                  \begin{Rightside}
			\beginnumbering
			\numberstanzafalse
                     
                         \stanza 
                     Je considère paisiblement le corps et le visage de la vierge. \&
                         \stanza 
                     Tous les hommes redoutent sa force physique. \&
                         \stanza 
                     La confiance en toute parole est perdue  \&
                         \stanza 
                     Les éclairs qu’ils soient à ma disposition \&
                         \stanza 
                     Jupiter a lancé des éclairs avec le bruyant fracas qui est le
                              sien. \&
                         \stanza 
                     ... autrefois, Jupiter les a frappées avec la foudre \&
                         \stanza 
                     Maintenant, tu sais que celle qui a obtenu le déshonneur a été
                              découverte. \&
                         \stanza 
                     Ils portent la rumeur subitement, toutes les fois qu’elle est aperçue
                              seule dans la rue. \&
                         \stanza 
                     Eh bien, chacun a mérité qu'il emporte le salaire présent en vertu de
                              ce qu'il a fait. \&
                         \stanza 
                     ... innocente, je suis jugée indignement par ma patrie. \&
                         \stanza 
                     .. je me souviens de ma main trempé ici par l'eau jaillissante d'un
                              fleuve blanc comme la neige. \&
                     
                  \endnumbering
		\end{Rightside}
               \end{pairs}
	\Columns
            
            
               \subsection*{Lycurgus}
               \begin{abstract}
                  Sources hellènes : Homère l'\textit{Ἰλιάς}; Eschylle
                        \textit{Λυκουργεια}.\par
                  Argument : Après avoir été averti de la venue d’un étranger accompagné de
                     bacchantes, Lycurge décide de chasser ces agitateurs en envoyant ses troupes à
                     leur poursuite. Des bacchantes sont tuées, d’autres capturées puis emprisonnées
                     avec Bacchus. Quand le dieu, resté anonyme, comparait au palais, il montre son
                     pouvoir, et punit ses persécuteurs.\par
               \end{abstract}
               \begin{pairs}
                  \begin{Leftside}
			\beginnumbering
			\setcounter{stanzaL}{0}
                     
                         \stanza \ledsidenote{\color{gray}{\textbf{Non.476,9\vspace{-0,5cm}}}}
                     Tuos
                              qui
                              celsos
                              terminos
                              tutant
                              .
                              .
                              . \&
                         \stanza \ledsidenote{\color{gray}{\textbf{Non.191,16\vspace{-0,5cm}}}}
                     
                              Alte
                              iubatos
                              angues
                              in
                              sese
                              gerunt
                              . \&
                         \stanza \ledsidenote{\color{gray}{\textbf{Non.192,29\vspace{-0,5cm}}}}
                     Quaque
                              incedunt
                              ,
                              omnes
                              arvas
                              obterunt
                              . \&
                         \stanza \ledsidenote{\color{gray}{\textbf{Non.323,1\vspace{-0,5cm}}}}
                              Vos
                              ,
                              qui
                              regalis
                              corporis
                              custodias & agitatis
                              ,
                              ite
                              actutum
                              in
                              frundiferos
                              locos
                              , & 
                     ingenio
                              arbusta
                              ubi
                              nata
                              sunt
                              ,
                              non
                              obsitu
                              . \&
                         \stanza \ledsidenote{\color{gray}{\textbf{Non.6,17\vspace{-0,5cm}}}}Alii & Sublime
                              in
                              altos
                              saltus
                              inlicite
                              .
                              .
                              . & 
                     Ubis
                              bipedes
                              volucres
                              lino
                              linquant
                              lumina
                              . \&
                         \stanza \ledsidenote{\color{gray}{\textbf{Non.225,1\vspace{-0,5cm}}}}
                              Pergite
                              , & 
                     Thyrsigeræ
                              Bacchæ
                              ,
                              Bacchico
                              cum
                              schemate
                              . \&
                         \stanza \ledsidenote{\color{gray}{\textbf{Non.213,10\vspace{-0,5cm}}}}
                     suavisonum
                              melos \&
                         \stanza \ledsidenote{\color{gray}{\textbf{Prisc. VI 695P\vspace{-0,5cm}}}}
                     Ignotæ
                              iteris
                              sumus
                              ,
                              tute
                              scis
                              .
                              .
                              . \&
                         \stanza \ledsidenote{\color{gray}{\textbf{Non.14,19\vspace{-0,5cm}}}}Ut
                              in
                              venatu
                              vitulantes
                              ex
                              suis & 
                     lucis
                              nos
                              mittant
                              pœnis
                              decoratas
                              feris
                              . \&
                         \stanza \ledsidenote{\color{gray}{\textbf{Non.540,5\vspace{-0,5cm}}}}
                     Pallis
                              patagiis
                              crocotis
                              malacis
                              mortualibus
                              . \&
                         \stanza \ledsidenote{\color{gray}{\textbf{Non.487,8\vspace{-0,5cm}}}}
                     Iam
                              ibi
                              nos
                              duplicat
                              advenientis
                              liberi
                              timos
                              pavos
                              . \&
                         \stanza \ledsidenote{\color{gray}{\textbf{Non.547,28\vspace{-0,5cm}}}}Nam
                              ludere
                              ut
                              lætantis
                              inter
                              se
                              vidimus
                              præter
                              amnem & 
                     creterris
                              sumere
                              aquam
                              ex
                              fonte
                              . \&
                         \stanza \ledsidenote{\color{gray}{\textbf{Varron, De ling. lat. VII, 53 \vspace{-0,5cm}}}}
                     Diabathra
                              in
                              pedibus
                              habebat
                              ,
                              erat
                              amictus
                              epicroco
                              . \&
                         \stanza \ledsidenote{\color{gray}{\textbf{Non.481,28\vspace{-0,5cm}}}}
                     Dic
                              quo
                              pacto
                              eum
                              potiti
                              :
                              pugnan
                              {pugnan}
                              an
                              dolis
                              ? \&
                         \stanza \ledsidenote{\color{gray}{\textbf{Non.159,7\vspace{-0,5cm}}}}
                     sine
                              ferro
                              ut
                              pecua
                              manibus
                              ad
                              mortem
                              meant \&
                         \stanza \ledsidenote{\color{gray}{\textbf{Non.9,24\vspace{-0,5cm}}}}Ducite & 
                     eo
                              cum
                              argutis
                              linguis
                              mutas
                              quadrupedis
                              . \&
                         \stanza \ledsidenote{\color{gray}{\textbf{Non.259,5\vspace{-0,5cm}}}}
                     Cave
                              {sis}
                              tuam
                              contendas
                              iram
                              contra
                              cum
                              ira
                              Liberi
                              . \&
                         \stanza \ledsidenote{\color{gray}{\textbf{Non.73,17\vspace{-0,5cm}}}}
                     Ne
                              ille
                              mei
                              feri
                              ingeni
                              iram
                              atque
                              animi
                              acrem
                              acrimoniam \&
                         \stanza \ledsidenote{\color{gray}{\textbf{Non. 124,30\vspace{-0,5cm}}}}Oderunt
                              di
                              homines
                              iniuros
                              . & 
                     —
                              Egone
                              {Egone}
                              an
                              ille
                              iniurie
                              Facimus
                              ? \&
                         \stanza \ledsidenote{\color{gray}{\textbf{Non. 109,25\vspace{-0,5cm}}}}
                     Ut
                              videam
                              Vulcani
                              opera
                              hæc
                              flammis
                              fieri
                              flora \&
                         \stanza \ledsidenote{\color{gray}{\textbf{Non. 503,19\vspace{-0,5cm}}}}
                     Late
                              longeque
                              longeque
                              transtros
                              nostros
                              fervere \&
                         \stanza \ledsidenote{\color{gray}{\textbf{Non. 84,32\vspace{-0,5cm}}}}Proinde
                              huc
                              Dryante
                              regem
                              prognatum
                              patre
                              , & 
                     
                              Lycurgam
                              cette
                              ! \&
                         \stanza \ledsidenote{\color{gray}{\textbf{Fest. 193 M.\vspace{-0,5cm}}}}
                     Vos
                              qui
                              astatis
                              obstinati
                              .
                              .
                              . \&
                         \stanza \ledsidenote{\color{gray}{\textbf{Non. 191,34\vspace{-0,5cm}}}}
                     Se
                              quasi
                              amnis
                              celeris
                              rapit
                              ,
                              sed
                              tamen
                              inflexu
                              flectitur
                              . \&
                         \stanza \ledsidenote{\color{gray}{\textbf{Non. 334,32\vspace{-0,5cm}}}}
                     
                              Iam
                              solis
                              æstu
                              candor
                              cum
                              liquesceret \&
                     
                  \endnumbering
		\end{Leftside}
                  \begin{Rightside}
			\beginnumbering
			\numberstanzafalse
                     
                         \stanza 
                     Tes troupes qui gardent les hautes frontières ...  \&
                         \stanza 
                     Elles portent sur elle bien haut des serpents à crinières. \&
                         \stanza 
                     Partout où elles pénètrent, elles piétinent tous les champs. \&
                         \stanza Vous, qui vous occupez des gardes du corps & royal, avancez immédiatement vers les lieux & 
                     feuillus, où les arbres ont poussé sans avoir été semés. \&
                         \stanza  Vous autres, attirez-les en hauteur dans les  & hauts pâturages ... & 
                     Où les bipèdes ailés abandonnent la lumière du jour pour le filet. \&
                         \stanza Continuez, & 
                      Bacchantes munies d'un thyrse, avec l'habillement bachique \&
                         \stanza 
                     mélodie suave \&
                         \stanza 
                     Nous ignorons le chemin, toi-même, tu le sais ... \&
                         \stanza De tel sorte qu'ils nous laissent aller vers & 
                     leurs bois sacrés, joyeuse, pendant la chasse, ils nous parent de
                              châtiments sauvages. \&
                         \stanza 
                     En robes, bandeaux ornementaux, robes de couleur safran, vêtements
                              doux du deuil. \&
                         \stanza 
                     Dès lors, l'appréhension, la crainte des arrivants libres, redoublent
                              en nous. \&
                         \stanza En effet, nous les avons vues qu'elles jouaient entre elles le long du
                              ruisseau, qu'elles pui- & 
                     saient l'eau de la source dans des coupes. \&
                         \stanza 
                     Il avait des chaussures de femme aux pieds, et était enveloppé d'une
                              robe de laine fine. \&
                         \stanza 
                     Dis comment vous vous êtes emparés de lui : par la force ou par la
                              ruse ? \&
                         \stanza 
                     comme un troupeau qui se dirige vers la mort suivant des mains sans
                              fers \&
                         \stanza Conduisez ici & 
                     ces femmes silencieuses avec les quadrupèdes aux langages
                              expressifs \&
                         \stanza 
                     Prends garde, je te prie, de ne pas faire rivaliser ta colère avec
                              celle de Liber. \&
                         \stanza 
                     De peur qu’il ne [ provoque] la colère de mon esprit et l’impétueuse
                              acrimonie de mon âme  \&
                         \stanza Les dieux haïssent les hommes injustes & 
                     — Est-ce lui, ou moi qui suis déloyal ? \&
                         \stanza 
                     Pour que je voie les oeuvres de Vulcain qui fait des flammes
                              éclatantes  \&
                         \stanza 
                     agiter nos poutres longues et larges \&
                         \stanza Ainsi donc, montre ici le roi Lycurgue descendant par son père
                              Dryas. & 
                      \&
                         \stanza 
                     Vous qui demeurez là obstinément ... \&
                         \stanza 
                     Comme si un fleuve violent l’emportait, mais pourtant, celui-ci se
                              plie au contour [de la rive]. \&
                         \stanza 
                     Dès lors, alors que la neige fond par la chaleur du soleil \&
                     
                  \endnumbering
		\end{Rightside}
               \end{pairs}
	\Columns
            
         
         
            \section{Quintus Ennius}
            Texte latin établi et organisé selon les choix de reconstruction et d'édition de \cite{EnniusLoeb}.\par
            
               \subsection*{Achilles}
               \begin{abstract}
                  Sources hellènes : Homère l'\textit{Ἰλιάς}; et les
                     versions du mythe d'Achille d'Aristarque de Tégée, Iophon, Astydamas, Carcinus,
                     Cléophon (?), Evaretus, et Diogène de Sinope. \par
                  Argument: furieux d’avoir été privé de son butin de guerre, Achille se retire
                     sous sa tente et refuse de poursuivre le combat. Agamemnon, subissant de
                     nombreuses défaites, décide de lui envoyer une ambassade afin de convaincre le
                     héros de retourner combattre auprès des Grecs. La tragédie s’achevait
                     certainement par la mort de Patrocle qui suscitera le retour d’Achille parmi
                     les guerriers grecs.\par
               \end{abstract}
               \begin{pairs}
                  \begin{Leftside}
			\beginnumbering
			\setcounter{stanzaL}{0}
                     
                         \stanza \ledsidenote{\color{gray}{\textbf{Cicéron, Verr., II, 1, 18\vspace{-0,5cm}}}}
                     form
                              Ita
                              magni
                              fluctus
                              eiciebantur
                              .
                              .
                              . \&
                         \stanza \ledsidenote{\color{gray}{\textbf{Fest. 305 M.\vspace{-0,5cm}}}}.
                              .
                              .
                              per
                              ego
                              deum
                              sublimas
                              subices & 
                     Umidas
                              ,
                              unde
                              oritur
                              imber
                              sonitu
                              sævo
                              [
                              et
                              spiritu
                              . \&
                         \stanza \ledsidenote{\color{gray}{\textbf{Fest. 282.9 M.\vspace{-0,5cm}}}}
                     
                              Prolato
                              ære
                              astitit \&
                         \stanza \ledsidenote{\color{gray}{\textbf{Non.147,22\vspace{-0,5cm}}}}
                     .
                              .
                              .
                              nam
                              consiliis
                              obvarant
                              ,
                              quibus
                              tam
                              concedit
                              hic
                              ordo
                              . \&
                         \stanza \ledsidenote{\color{gray}{\textbf{Non.166,22\vspace{-0,5cm}}}}Quo
                              nunc
                              incerta
                              re
                              atque
                              inorata
                              gradum & 
                     regredere
                              conare \&
                         \stanza \ledsidenote{\color{gray}{\textbf{Non.277,21\vspace{-0,5cm}}}}
                     Serva
                              cives
                              ,
                              defende
                              hostes
                              ,
                              cum
                              potes
                              [
                              defendere \&
                         \stanza \ledsidenote{\color{gray}{\textbf{Non.472,50\vspace{-0,5cm}}}}
                     Interea
                              mortales
                              inter
                              sese
                              pugnant
                              ,
                              proeliant
                              . \&
                         \stanza \ledsidenote{\color{gray}{\textbf{Isodore, Different., 218\vspace{-0,5cm}}}}summam
                              tu
                              tibi & pro
                              mala
                              vita
                              famam
                              extolles
                              ,
                              [
                              et
                              ]
                              pro
                              bona
                              partam
                              gloriam
                              . & 
                     
                              Male
                              volentes
                              [
                              enim
                              ]
                              famam
                              tollunt
                              ,
                              bene
                              volentes
                              gloriam
                              . \&
                     
                  \endnumbering
		\end{Leftside}
                  \begin{Rightside}
			\beginnumbering
			\numberstanzafalse
                     
                         \stanza 
                      Ainsi les grands flots étaient jetés ...  \&
                         \stanza  ... Moi, à travers les sublimes marche- & 
                     pieds humides des dieux, de là les pluies naissent d’un souffle et
                              d’un retentissement impétueux. \&
                         \stanza 
                     Il s’est tenu debout avec l’airain porté en avant \&
                         \stanza 
                     ... car ils font obstacle aux résolutions, si par elles l’ordre
                              s’éloigne d’ici. \&
                         \stanza Maintenant, là où l’événement est incertain et méconnu, tu tentes de
                              revenir sur & 
                     tes pas \&
                         \stanza 
                     Sauve les citoyens, repousse les ennemies puisque tu peux les
                              repousser \&
                         \stanza 
                     Pendant que les mortels combattent entre eux, ils livrent bataillent.
                            \&
                         \stanza Tu élèveras le plus haut sommet pour toi- & même, la renommée en faveur d’une mauvaise vie [ et] la gloire
                              engendrée pour une bonne. & 
                     [ En réalité], ceux qui veulent le mal élèvent la renommée, ceux qui
                              veulent le bien la gloire. \&
                     
                  \endnumbering
		\end{Rightside}
               \end{pairs}
	\Columns
            
            
               \subsection*{Alcmæon}
               \begin{abstract}
                  Sources hellènes : Homère l'\textit{Ὀδύσσεια}; Euripide
                        \textit{Ἀλκμαίων ὁ διὰ Κορίνθου} et \textit{Ἀλκμαίων ὁ διὰ Ψωφῖδος}; ainsi que les versions
                     fragmentaires de Timothée d'Athènes, Astydamas II, Théodecte de Phasélis,
                     Evaretus et Nicomaque d'Alexandrie.\par
                   Argument: Alcméon, tue sa mère afin de venger son père, Amphiaraos: celui-ci
                     avait été contraint par sa femme de participer à la guerre contre des Épigones,
                     le condamnant à une mort certaine. À la suite de ce matricide, Alcméon sera
                     alors poursuivi par les Furies : celles-ci pousseront le héros à remettre en
                     question son acte.\par
               \end{abstract}
               \begin{pairs}
                  \begin{Leftside}
			\beginnumbering
			\setcounter{stanzaL}{0}
                     
                         \stanza \ledsidenote{\color{gray}{\textbf{Cicéron, De Orat., III, 58, 218\vspace{-0,5cm}}}}Multis
                              sum
                              modis
                              circumventus
                              ,
                              morbo
                              ,
                              exilio
                              atque
                              inopia
                              ; & Tum
                              pavor
                              sapientiam
                              omnem
                              mi
                              exanimato
                              expectorat
                              . & †
                              Alter
                              †
                              terribilem
                              minatur
                              vitæ
                              cruciatum
                              et
                              necem
                              ; & quæ
                              nemo
                              est
                              tam
                              firmo
                              ingenio
                              et
                              tanta
                              [
                              confidentia
                              , & 
                     quin
                              refugiat
                              timido
                              sanguen
                              atque
                              exalbescat
                              metu
                              . \&
                         \stanza \ledsidenote{\color{gray}{\textbf{Cicéron, Acad., II, 17\vspace{-0,5cm}}}}
                     Sed
                              mihi
                              ne
                              utiquam
                              cor
                              consentit
                              cum
                              oculorum
                              aspectu
                              .
                              . \&
                         \stanza \ledsidenote{\color{gray}{\textbf{Cicéron, Acad., II, 89\vspace{-0,5cm}}}}.
                              .
                              .
                              .
                              unde
                              hæc
                              flamma
                              oritur
                              ? & {
                              .
                              .
                              .
                              } & Incedunt
                              ,
                              incedunt
                              :
                              adsunt
                              adsunt
                              ,
                              me
                              [
                              expetunt & .
                              {
                              .
                              .
                              .
                              } & Fer
                              mi
                              auxilium
                              ,
                              pestem
                              abige
                              a
                              me
                              , & flammiferam
                              hanc
                              vim
                              ,
                              quæ
                              me
                              excruciat
                              ; & Cæruleæ
                              incinctæ
                              igni
                              incedunt
                              , & circumstant
                              cum
                              ardentibus
                              tædis
                              . & {
                              .
                              .
                              .
                              } & Intendit
                              crinitus
                              Apollo & Arcum
                              auratum
                              ,
                              luna
                              innixus
                              , & 
                     Diana
                              facem
                              iacit
                              a
                              læva
                              . \&
                         \stanza \ledsidenote{\color{gray}{\textbf{Non.127,13\vspace{-0,5cm}}}}
                     factum
                              est
                              iam
                              diu \&
                     
                  \endnumbering
		\end{Leftside}
                  \begin{Rightside}
			\beginnumbering
			\numberstanzafalse
                     
                         \stanza Par bien des manières, je suis entouré par la maladie, l’exil et le
                              dénuement. & Alors l’effroi bannit de mon esprit toute prudence qui m’épuise. & Quelqu’un menace la vie d’un terrible supplice et de meurtre; & tel que personne n’a un esprit si ferme et tant de confiance, & 
                     que son sang ne recule pas devant l’inquiétude et qu'il ne pâlit pas
                              face à l’anxiété. \&
                         \stanza 
                     Mais mon cœur n’est nullement d’accord avec la vision de mes yeux.
                            \&
                         \stanza ... d’où naît cette flamme ? & \{...\} & Elles arrivent, elles arrivent, elles sont là, elles sont là, elles
                              cherchent à m’attendre & \{...\} & Rapporte-moi de l’aide, éloigne de moi ce & fléau, cette puissance enflammée qui me torture; de couleur azur,
                              elles s’avancent & ceintes par le feu,  & elles se tiennent de toutes & parts avec des torches ardentes. & \{...\} & Apollon à la longue chevelure bande & 
                     l’arc doré, s’appuyant sur le croissant Diane lance une torche de la
                              main gauche \&
                         \stanza 
                     Cela a été fait il y a déjà longtemps. \&
                     
                  \endnumbering
		\end{Rightside}
               \end{pairs}
	\Columns
            
            
               \subsection*{Alexandrus}
               \begin{abstract}
                  Sources hellènes : Homère l'\textit{Ὀδύσσεια}; \textit{Ἀλέξανδρος} de Sophocle ainsi que la version
                     d'Euripide, \textit{Ἀλκμαίων ὁ διὰ Ψωφῖδος}, et Nicomaque
                     d'Alexandrie \textit{τροίας}.\par
                  Argument: ce drame se concentre sur l’enfance de Paris: son père commande son
                     assassinat à la suite du rêve prémonitoire que fait sa femme, Hécube, qui le
                     lie à la destruction de Troie . Abandonné sur le mont Ida par les serviteurs du
                     palais, l’enfant est recueilli par des bergers qui le nomment Alexandre. Devenu
                     adulte il participe aux jeux organisés par Priam, et y vainc ses frères.
                     Ceux-ci n’acceptent pas d’avoir été battus par un berger; l’un d’eux, Déiphobe
                     entreprend même de le tuer. Cassandre intervient, et révèle la véritable
                     identité du vainqueur ainsi que la destruction future de Troie. \par
               \end{abstract}
               \begin{pairs}
                  \begin{Leftside}
			\beginnumbering
			\setcounter{stanzaL}{0}
                     
                         \stanza \ledsidenote{\color{gray}{\textbf{Varron, De ling. lat.,VI, 83, M.\vspace{-0,5cm}}}}Iam
                              dudum
                              ab
                              ludis
                              animus
                              atque
                              aures
                              [
                              avent & 
                     avide
                              expectantes
                              nuntium
                              . \&
                         \stanza \ledsidenote{\color{gray}{\textbf{Varron, De ling. lat.,VII, 82, M.\vspace{-0,5cm}}}}
                     Qua
                              propter
                              Parim
                              pastores
                              nunc
                              Alexandrum
                              vocant
                              .
                              .
                              . \&
                         \stanza \ledsidenote{\color{gray}{\textbf{Fest.217 M\vspace{-0,5cm}}}}
                     †
                              amidio
                              †
                              purus
                              putus
                              . \&
                         \stanza \ledsidenote{\color{gray}{\textbf{Fest.317 M\vspace{-0,5cm}}}}Hominem
                              appellat
                              :
                              quid
                              lascivis
                            & 
                     stolide
                              ?
                              Non
                              intellegit
                              . \&
                         \stanza \ledsidenote{\color{gray}{\textbf{Fest.165 M\vspace{-0,5cm}}}}
                     Volans
                              de
                              cælo
                              cum
                              corona
                              et
                              tæniis \&
                         \stanza \ledsidenote{\color{gray}{\textbf{Macrobe, Sat.,VI, 1, 61\vspace{-0,5cm}}}}Multi
                              alii
                              adventant
                              ,
                              paupertas
                              quorum & 
                     obscurat
                              nomina
                              . \&
                         \stanza \ledsidenote{\color{gray}{\textbf{Macrobe, Sat.,VI, 2, 18\vspace{-0,5cm}}}}O
                              lux
                              Troiæ
                              ,
                              germane
                              Hector
                              ; & quid
                              ita
                              cum
                              tuo
                              lacerato
                              corpore
                              , & 
                     miser
                              es
                              ,
                              aut
                              qui
                              te
                              sic
                              respectantibus
                              tractavere
                              nobis
                              ? \&
                         \stanza \ledsidenote{\color{gray}{\textbf{Macrobe, Sat.,VI, 2, 25\vspace{-0,5cm}}}}Nam
                              maximo
                              saltu
                              superavit
                              gravidus
                              armatis
                              equus
                              , & 
                     Suo
                              qui
                              partu
                              ardua
                              perdat
                              Pergama
                              . \&
                     
                  \endnumbering
		\end{Leftside}
                  \begin{Rightside}
			\beginnumbering
			\numberstanzafalse
                     
                         \stanza Depuis longtemps, l’âme et les oreilles désirent avidement et
                              attendent une nouvelle & 
                     des jeux. \&
                         \stanza 
                     Pour cette raison, les bergers ont désormais appelé Paris, Alexandre .
                              . \&
                         \stanza 
                     ? absolument pur \&
                         \stanza Il s’adresse à l’homme : pourquoi  & 
                      badinez-vous sottement? Il ne comprend pas. \&
                         \stanza 
                     Volant en haut du ciel avec une couronne et des bandelettes  \&
                         \stanza Beaucoup d’autres approchent, dont la  & 
                     pauvreté obscurcit leurs noms. \&
                         \stanza  Ô lumière de Troie, ô authentique Hector; & pourquoi es-tu ainsi misérable avec ton corps lacéré ? & 
                     ou bien, quels hommes t’ont traité ainsi quand nous avions les yeux
                              tournés ?  \&
                         \stanza De fait, le cheval, lesté avec des armes, abattu par son très grand
                              saut qui détruit & 
                     les forteresses de Pergame avec sa descendance. \&
                     
                  \endnumbering
		\end{Rightside}
               \end{pairs}
	\Columns
            
            
               \subsection*{Andromacha Æchmalotis}
               \begin{abstract}
                  Sources hellènes : Homère l'\textit{Ἰλιάς}; Euripide
                        \textit{Ἀνδρομάχη}, \textit{Ἑκάβη}
                     et \textit{Τρῳάδες}; Antiphon \textit{Ἀνδρομάχη}.\par
                  Argument: l’intrigue de cette tragédie se situe au moment de la chute de Troie
                     : la pièce développe les lamentations d’Andromaque qui atteignent leur
                     apothéose lors du meurtre d’Astyanax. \par
               \end{abstract}
               \begin{pairs}
                  \begin{Leftside}
			\beginnumbering
			\setcounter{stanzaL}{0}
                     
                         \stanza \ledsidenote{\color{gray}{\textbf{Cicéron, Tusc., I, 105\vspace{-0,5cm}}}}Vidi
                              ,
                              videre
                              quod
                              me
                              passa
                              ægerrume
                              , & 
                     Hectorem
                              curru
                              quadriiugo
                              raptarier \&
                         \stanza \ledsidenote{\color{gray}{\textbf{Cicéron, Tusc., I, 105\vspace{-0,5cm}}}}
                              Ex
                              opibus
                              summis
                              opis
                              egens
                              ,
                              Hector
                              ,
                              [
                              tuæ & {
                              .
                              .
                              .
                              } & Quid
                              petam
                              præsidi
                              aut
                              exequar
                              ?
                              quove
                              {quove}
                              [
                              nunc & auxilio
                              aut
                              exili
                              aut
                              fugæ
                              freta
                              sim
                              ? & Arce
                              et
                              urbe
                              orba
                              sum
                              .
                              Quo
                              accedam
                              ?
                              [
                              Quo
                              applicem
                              ? & Cui
                              nec
                              aræ
                              patriæ
                              domi
                              stant
                              ,
                              fractæ
                              et
                              [
                              disiectæ
                              iacent
                              ; & fana
                              flamma
                              deflagrata
                              ,
                              tosti
                              alti
                              stant
                              parietes & Deformati
                              atque
                              abiete
                              crispa
                              .
                              .
                              . & {
                              .
                              .
                              .
                              } & O
                              pater
                              ,
                              o
                              patria
                              ,
                              o
                              Priami
                              domus
                              , & Sæptum
                              altisono
                              cardine
                              templum
                              ! & Vidi
                              ego
                              te
                              adstante
                              ope
                              barbarica & tectis
                              cælatis
                              laqueatis
                              , & auro
                              ebore
                              instructam
                              regifice. & {
                              .
                              .
                              .
                              } & Haec
                              omnia
                              vidi
                              inflammari & ,
                              Priamo
                              vi
                              vitam
                              evitari, & 
                     Iovis
                              aram
                              sanguine
                              turpari \&
                         \stanza \ledsidenote{\color{gray}{\textbf{Varron, De ling., VII, 6, M\vspace{-0,5cm}}}}
                     Acherusia
                              templa
                              alta
                              Orci
                              salvete
                              infera
                              .
                              .
                              . \&
                         \stanza \ledsidenote{\color{gray}{\textbf{Varron, De ling., VII, 82, M\vspace{-0,5cm}}}}
                     Andromachæ
                              nomen
                              qui
                              indidit
                              ,
                              recte
                              [
                              indidit
                              aut
                              Alexandrum
                              .
                              .
                              . \&
                         \stanza \ledsidenote{\color{gray}{\textbf{Fest. 384.16\vspace{-0,5cm}}}}
                     di
                              .
                              .
                              .
                              on
                              est
                              :
                              Nam
                              mussare
                              si
                              .
                              .
                              . \&
                         \stanza \ledsidenote{\color{gray}{\textbf{Non. 76, 1\vspace{-0,5cm}}}}
                     Quid
                              fit
                              ?
                              seditio
                              tabetne
                              {tabetne}
                              ,
                              an
                              numeros
                              augificat
                              suos
                              ? \&
                         \stanza \ledsidenote{\color{gray}{\textbf{292, 8\vspace{-0,5cm}}}}
                     Quantis
                              cum
                              ærumnis
                              illum
                              exanclavi
                              diem \&
                         \stanza \ledsidenote{\color{gray}{\textbf{Non. 402,3\vspace{-0,5cm}}}}.
                              .
                              .
                              annos
                              multos
                              longinque
                              domo & 
                     bellum
                              gerentes
                              summum
                              summa
                              industria \&
                         \stanza \ledsidenote{\color{gray}{\textbf{Non. 504, 16\vspace{-0,5cm}}}}Nam
                              ubi
                              introducta
                              est
                              ,
                              puerumque
                              {puerumque}
                              ,
                              ut
                              [
                              laverent
                              ,
                              locant & 
                     In
                              clupeo \&
                         \stanza \ledsidenote{\color{gray}{\textbf{Non. 505,25\vspace{-0,5cm}}}}
                     Nam
                              neque
                              irati
                              neque
                              blandi
                              quicquam
                              [
                              sincere
                              sonunt \&
                         \stanza \ledsidenote{\color{gray}{\textbf{515,27\vspace{-0,5cm}}}}.
                              .
                              .
                              sed
                              quasi
                              ferrum
                              aut
                              lapis & 
                     durat
                              ,
                              rarenter
                              gemitum
                              †
                              conatur
                              trabem
                              † \&
                         \stanza \ledsidenote{\color{gray}{\textbf{Macrobe,Sat. VI, 5, 10\vspace{-0,5cm}}}}
                     rapit
                              ex
                              alto
                              naves
                              velivolas \&
                     
                  \endnumbering
		\end{Leftside}
                  \begin{Rightside}
			\beginnumbering
			\numberstanzafalse
                     
                         \stanza 
                      \&
                         \stanza J’ai vu que j’ai enduré de voir avec la plus grande souffrance & 
                     Hector être emporté par un quadrige \&
                         \stanza d’après les plus grands secours, ayant besoin de ton secours,
                              Hector & \{...\} & Quelle protection dois-je solliciter ou poursuivre ? Ou à présent, en
                              quelle aide, l’exil,  & ou la fuite, puis-je compter ? & Je suis orpheline de ville et de citadelle. Où dois-je aller ? Sur qui
                              puis-je m’adosser ?  & Moi, à qui ni les hôtels paternels ni le foyer ne subsistent:
                              dispersés et brisés, ils sont en ruines;  & les temples consumés par les flammes, les hauts murs brûlés et altérés
                              subsistent grâce & au sapin tordu... & \{...\} & Ô père, ô patrie, ô maison de Priam, ô temple  & enceint d’une ligne sublime qui résonne & fort. Je t’ai vu quand le soutien se tenait  & aux côtés des étrangers avec tes toits sculptés et lambrissés, pourvu
                              royalement avec de l’or et de l’ivoire. & \{...\} & J’ai vu toutes ces choses être incendiées, & la vie de Priam être ôtée par la force, & 
                     l’autel de Jupiter être souillé par le sang \&
                         \stanza 
                     Profonds sanctuaires achérusien d’Orcus, salutations...  \&
                         \stanza 
                     Celui qui a donné le nom d’Andromaque l’a donné convenablement, ou
                              Alexandre … \&
                         \stanza 
                     les dieux … [ ?] est, en vérité taire si …  \&
                         \stanza 
                     Qu’arrive-t-il ? Est-ce que le soulèvement se désagrège ? Ou est-ce
                              qu’il accroît leurs effectifs ? \&
                         \stanza 
                     Avec quelles grandes peines ai-je enduré ce jour \&
                         \stanza  ... depuis de nombreuses années, loin du & 
                     foyer, puisqu’ ils accomplissent une guerre importante avec la plus
                              grande application  \&
                         \stanza En réalité, lorsqu’elle a été conduite à l’intérieur, et comme ils
                              lavaient l’enfant, ils & 
                     le placent sur le bouclier \&
                         \stanza 
                     De fait ni les impétueux, ni les flatteurs, ne clament quelque chose
                              sincèrement \&
                         \stanza ... mais, comme la pierre endurcit le fer,  & 
                     il se prépare à gémir rarement à la massue  \&
                         \stanza 
                     il emporte les voiliers en provenance du large \&
                     
                  \endnumbering
		\end{Rightside}
               \end{pairs}
	\Columns
            
            
               \subsection*{Andromeda}
               \begin{abstract}
                  Sources hellènes : les versions d'\textit{Ἀνδρομέδα} de
                     Sophocle, Euripide, Lycophron, Phrynichos.\par
                   Argument: cassiopée, reine de l’Éthiopie, se targue d’avoir une fille plus
                     belle que les Néréides : son excès d’orgueil offense Neptune qui inonde les
                     côtes éthiopiennes et envoie un monstre marin. Désespéré, le roi consulte
                     l’oracle d’Ammon qui révèle que seul le sacrifice d’Andromède au monstre pourra
                     faire cesser le fléau. Celle-ci est alors enchaînée à un rocher.\par
                   La critique \footcite{voir\cite{EnniusLoeb}, p.41} suppose
                     qu’Ennius suit précisément l’intrigue du drame d’Euripide: la pièce
                     commencerait alors qu’Andromède se lamente attendant sa mort sur un rocher ;
                     puis arrive Persée tombant éperdument amoureux de la belle, il décide de la
                     sauver en combattant le monstre qui met en péril son bonheur.\par
               \end{abstract}
               \begin{pairs}
                  \begin{Leftside}
			\beginnumbering
			\setcounter{stanzaL}{0}
                     
                         \stanza \ledsidenote{\color{gray}{\textbf{Varron, De ling.,VII, 6\vspace{-0,5cm}}}}quæ
                              cava
                              cæli & 
                     signitenentibus
                              conficis
                              bigis \&
                         \stanza \ledsidenote{\color{gray}{\textbf{Fest.217 M\vspace{-0,5cm}}}}
                     Liberum
                              quæsendum
                              causa
                              familiæ
                              matrem
                              tuæ
                              . \&
                         \stanza \ledsidenote{\color{gray}{\textbf{Non. 169, 21\vspace{-0,5cm}}}}
                     Scrupeo
                              investita
                              saxo
                              ,
                              atque
                              ostreis
                              squamæ
                              scabrent
                              . \&
                         \stanza \ledsidenote{\color{gray}{\textbf{Fest.375 M\vspace{-0,5cm}}}}.
                              .
                              .
                              circum
                              sese
                              urvat
                              ad
                              pedes, & 
                     
                              a
                              terra
                              quadringentos
                              caput \&
                         \stanza \ledsidenote{\color{gray}{\textbf{Varron, Non. 20, 23\vspace{-0,5cm}}}}
                     .
                              corpus
                              contemplatur
                              ,
                              unde
                              corporaret
                              [
                              vulnere \&
                         \stanza \ledsidenote{\color{gray}{\textbf{Non. 165, 11\vspace{-0,5cm}}}}
                     .
                              .
                              .
                              rursus
                              prosus
                              reciprocat
                              fluctus
                              feram \&
                         \stanza \ledsidenote{\color{gray}{\textbf{Non. 183, 17\vspace{-0,5cm}}}}
                              .
                              .
                              .
                              .
                              alia
                              fluctus
                              differt
                              dissupat
                              , & 
                     
                              visceratim
                              membra
                              ,
                              maria
                              salsa
                              spumant
                              sanguine
                              . \&
                         \stanza \ledsidenote{\color{gray}{\textbf{Varron, De ling.,VII, 6\vspace{-0,5cm}}}}
                     A
                              filiis
                              propter
                              te
                              objecta
                              sum
                              innocens
                              [
                              Nerei
                              . \&
                     
                  \endnumbering
		\end{Leftside}
                  \begin{Rightside}
			\beginnumbering
			\numberstanzafalse
                     
                         \stanza toi qui parcours les profondeurs du ciel & 
                     avec ton char étoilé \&
                         \stanza 
                     Pour demander des enfants à la mère de ta famille.  \&
                         \stanza 
                     Revêtus par une roche revêche, et les écailles sont hérissées par des
                              huîtres.  \&
                         \stanza ... autour de lui, il trace le sillon d’enceinte & 
                     de la capitale jusqu’à quatre cents pieds du sol \&
                         \stanza 
                     il considère le corps, là où il pourrait le tuer par une plaie  \&
                         \stanza 
                     Le flot fait osciller la bête d’avant en arrière \&
                         \stanza Le flot disperse en lambeaux les autres mem- & 
                     bres, les océans salés écument du sang. \&
                         \stanza 
                     À cause de toi, j’ai été exposée, innocente, aux filles de Nérée. \&
                     
                  \endnumbering
		\end{Rightside}
               \end{pairs}
	\Columns
            
            
               \subsection*{Herctor lytra}
               \begin{abstract}
                   Sources hellènes : Homère l'\textit{Ἰλιάς}; Eschyle
                        \textit{Φρύγες ἢ Ἕκτορος Λύτρα}
                  \par
                   Argument: \textit{Le départ d'Hector} se concentre sur de la rançon
                     qu’offre Priam à Achille en échange de la dépouille de son fils Hector. Les
                     fragments ne permettent pas de décrire précisément le fil de la tragédie. \par
               \end{abstract}
               \begin{pairs}
                  \begin{Leftside}
			\beginnumbering
			\setcounter{stanzaL}{0}
                     
                         \stanza \ledsidenote{\color{gray}{\textbf{Fest.270 M\vspace{-0,5cm}}}}
                     quæ
                              mea
                              comminus
                              machæra
                              atque
                              hasta
                              †
                              hospius
                              manu† \&
                         \stanza \ledsidenote{\color{gray}{\textbf{Diomède, I p. 336P\vspace{-0,5cm}}}}
                     sublime
                              iter
                              quadrupedantes
                              flammam
                              [
                              halitantes \&
                         \stanza \ledsidenote{\color{gray}{\textbf{Fest.270 M\vspace{-0,5cm}}}}Nos
                              quiescere
                              æquum
                              est
                              ?
                              nomus
                              ambo & 
                     Ulixem \&
                         \stanza \ledsidenote{\color{gray}{\textbf{Non. 111,15\vspace{-0,5cm}}}}At
                              ego
                              ,
                              omnipotens & 
                     
                              ,
                              ted
                              exposco
                              ,
                              ut
                              hoc
                              consilium
                              Achivis
                              auxilio
                              fuat
                              . \&
                         \stanza \ledsidenote{\color{gray}{\textbf{Non. 222,33\vspace{-0,5cm}}}}
                     .
                              .
                              .
                              inferum
                              vastos
                              specus \&
                         \stanza \ledsidenote{\color{gray}{\textbf{Non. 355, 15\vspace{-0,5cm}}}}Hector
                              vi
                              summa
                              armatos
                              educit
                              foras & 
                     ,
                              castrisque
                              {castrisque}
                              castra
                              ultro
                              iam
                              conferre
                              occupat \&
                         \stanza \ledsidenote{\color{gray}{\textbf{Non. 399\vspace{-0,5cm}}}}Melius
                              est
                              virtute
                              ius
                              :
                              nam
                              sæpe
                              virtutem
                              mali
                              , & 
                     nanciscuntur
                              ;
                              ius
                              atque
                              æcum
                              se
                              a
                              malis
                              spernit
                              procul
                              . \&
                         \stanza \ledsidenote{\color{gray}{\textbf{Non. 407,22\vspace{-0,5cm}}}}†
                              ducet
                              quadrupedum
                              iugo
                              invitam
                              doma
                              infrena & 
                     et
                              iuge
                              valida
                              quorum
                              tenacia
                              infrenari
                              minis
                              †
                              . \&
                         \stanza \ledsidenote{\color{gray}{\textbf{Non. 467,23 M\vspace{-0,5cm}}}}
                     Constitit
                              ,
                              credo
                              ,
                              Scamander
                              ;
                              arbores
                              vento
                              vacant \&
                         \stanza \ledsidenote{\color{gray}{\textbf{Non. 469,25\vspace{-0,5cm}}}}
                     
                              Qui
                              cupiant
                              dare
                              arma
                              Achilli
                              †
                              ut
                              ipse
                              †
                              ,
                              cunctent \&
                         \stanza \ledsidenote{\color{gray}{\textbf{Non. 472,27\vspace{-0,5cm}}}}.
                              .
                              .
                              per
                              vos
                              et
                              vostrorum & 
                     imperium
                              et
                              fidem
                              ,
                              Myrmidonum
                              vigiles
                              ,
                              commiserescite
                              ! \&
                         \stanza \ledsidenote{\color{gray}{\textbf{Non. 489, 27\vspace{-0,5cm}}}}
                     Quid
                              hoc
                              hic
                              clamoris
                              ?
                              quid
                              tumulti
                              est
                              ?
                              nomen
                              qui
                              usurpat
                              meum? \&
                         \stanza \ledsidenote{\color{gray}{\textbf{Non. 490, 7\vspace{-0,5cm}}}}
                     Quid
                              in
                              castris
                              strepiti
                              est
                              ? \&
                         \stanza \ledsidenote{\color{gray}{\textbf{Non. 504,32\vspace{-0,5cm}}}}
                     Æs
                              sonit
                              ,
                              franguntur
                              hastæ
                              ,
                              terra
                              sudat
                              sanguine \&
                         \stanza \ledsidenote{\color{gray}{\textbf{Non. 511,8\vspace{-0,5cm}}}}
                     Sæviter
                              fortunam
                              ferro
                              cernunt
                              de
                              victoria
                              . \&
                         \stanza \ledsidenote{\color{gray}{\textbf{Non. 513,4\vspace{-0,5cm}}}}ecce
                              autem
                              caligo
                              oborta
                              est
                              ,
                              omnem
                              prospectum
                              abstulit & 
                     derepente
                              ;
                              contulit
                              sese
                              in
                              pedes \&
                     
                  \endnumbering
		\end{Leftside}
                  \begin{Rightside}
			\beginnumbering
			\numberstanzafalse
                     
                         \stanza 
                     que sans délai le coutelas et le sabre te payent en retour par ma
                              main \&
                         \stanza 
                      Les chevaux émanent du feu vers leur route céleste \&
                         \stanza Quand est-ce que nous nous reposons ? Nous reconnaissons ensemble
                              Ulysse. & 
                      \&
                         \stanza Mais moi, tout-puissant, je te demande que & 
                      que cette mesure soit une aide pour tous les Grecs. \&
                         \stanza 
                     les immenses cavernes du chtonien \&
                         \stanza Hector chasse les hommes armés dehors & 
                     par une puissance absolue, et déjà il s’occupe de réunir les camps
                              au-delà des camps \&
                         \stanza La justice est mieux que la bravoure : de fait, les malfaiteurs
                              acquièrent souvent la & 
                     bravoure ; et la justice impartiale s’éloigne loin des
                              malfaiteurs. \&
                         \stanza Elle conduira le cheval par le joug, dompte, malgré elle, les
                              impétueux, et les  & 
                     vigoureux, parmi lesquels, rétifs au joug, sont domptés par les
                              menaces. \&
                         \stanza 
                     Je crois que le Scamandre s’est stabilisé : les arbres sont à l’abri
                              du vent. \&
                         \stanza 
                     Ceux-ci désirent que lui-même donne les armes d’Achille si bien qu’ils
                              hésitent \&
                         \stanza ... par vous, par votre autorité et votre & 
                     fidélité, patrouilles des Myrmidons, ayez pitié ! \&
                         \stanza 
                     Pourquoi ce cri ici ? Pourquoi y a-t-il du bruit ? Qui usurpe mon nom
                              ?  \&
                         \stanza 
                     Pourquoi y a-t-il du vacarme dans les camps ?  \&
                         \stanza 
                     Le bronze sonne, les lances sont rompues, la terre ruisselle de
                              sang \&
                         \stanza 
                     Ils distinguent avec rigueur la chance par le fer de la victoire. \&
                         \stanza  Mais voilà que le brouillard est apparu, il a emporté toute
                              perspective ; il engage le  & 
                     combat. \&
                     
                  \endnumbering
		\end{Rightside}
               \end{pairs}
	\Columns
            
            
               \subsection*{Hecuba}
               \begin{abstract}
                  Sources hellènes : Homère l'\textit{Ἰλιάς}; Euripide
                        \textit{Ἑκάβη} et \textit{Τρῳάδες}.\par
                  Argument: l’action se situe après la chute de Troie, et se concentre sur le
                     sort subi par la reine Hécube. Ainsi, la tragédie rapporte ses lamentations et
                     sa capture par Ulysse.\par
               \end{abstract}
               \begin{pairs}
                  \begin{Leftside}
			\beginnumbering
			\setcounter{stanzaL}{0}
                     
                         \stanza \ledsidenote{\color{gray}{\textbf{Varron, De ling. lat., VII, 6M.\vspace{-0,5cm}}}}
                     O
                              magna
                              templa
                              cælitum
                              commixta
                              [
                              stellis
                              splendidis
                              ! \&
                         \stanza \ledsidenote{\color{gray}{\textbf{Gel. X, 4\vspace{-0,5cm}}}}Hæc
                              tu
                              etsi
                              perverse
                              dices
                              ,
                              facile
                              Achivos
                              flexeris
                              ; & nam
                              opulenti
                              cum
                              locuntur
                              pariter
                              atque
                              ignobiles
                              , & 
                     eadem
                              dicta
                              eademque
                              {eademque}
                              oratio
                              æqua
                              non
                              æque
                              valet \&
                         \stanza \ledsidenote{\color{gray}{\textbf{Non. 115, 33\vspace{-0,5cm}}}}
                     vide
                              hunc
                              ,
                              meæ
                              in
                              quem
                              lacrumæ
                              guttatim
                              cadunt \&
                         \stanza \ledsidenote{\color{gray}{\textbf{Non. 116, 28\vspace{-0,5cm}}}}
                     Juppiter
                              tibi
                              summe
                              tandem
                              male
                              re
                              gesta
                              gratulor
                              . \&
                         \stanza \ledsidenote{\color{gray}{\textbf{Non. 153, 28\vspace{-0,5cm}}}}Set
                              numquam
                              scripstis
                              ,
                              qui
                              parentem
                              [
                              aut
                              hospitem & 
                     
                              necasset
                              quo
                              quis
                              cruciatu
                              perbiteret \&
                         \stanza \ledsidenote{\color{gray}{\textbf{Non. 223, 23\vspace{-0,5cm}}}}
                     
                              .
                              .
                              .
                              undantem
                              salum \&
                         \stanza \ledsidenote{\color{gray}{\textbf{Non. 224, 6\vspace{-0,5cm}}}}
                     
                              Heume
                              miseram
                              !
                              interii
                              :
                              pergunt
                              lavere
                              sanguen
                              sanguine
                              . \&
                         \stanza \ledsidenote{\color{gray}{\textbf{Non. 342, 25\vspace{-0,5cm}}}}
                     Quæ
                              tibi
                              in
                              concubio
                              verecunde
                              et
                              modice
                              morem
                              gerit
                              . \&
                         \stanza \ledsidenote{\color{gray}{\textbf{Non. 474, 30\vspace{-0,5cm}}}}.
                              .
                              .
                              miserete
                              anuis
                              : & 
                     
                              date
                              ferrum
                              ,
                              qui
                              me
                              anima
                              privem \&
                         \stanza \ledsidenote{\color{gray}{\textbf{Non. 494,1\vspace{-0,5cm}}}}Senex
                              sum
                              :
                              utinam
                              mortem
                              oppetam
                              ,
                              [
                              prius
                              quam
                              evenat & 
                     quod
                              in
                              pauperie
                              mea
                              senex
                              graviter
                              gemam
                              ! \&
                     
                  \endnumbering
		\end{Leftside}
                  \begin{Rightside}
			\beginnumbering
			\numberstanzafalse
                     
                         \stanza 
                     Ô grands sanctuaires des cieux, mêlés [aux étoiles étincelantes! \&
                         \stanza Bien que, tu diras cela de travers, tu auras incliné aisément les
                              grecques; & car les hommes opulents parlent de la même manière que ceux de basse
                              naissance, & 
                     avec les mêmes mots et les mêmes paroles, qui n’ont pas la même
                              valeur \&
                         \stanza 
                     Regarde celui sur qui mes larmes tombent goutte à goutte \&
                         \stanza 
                      Ô Jupiter souverain, en fin de compte, je te loue pour l’événement
                              funeste accompli. \&
                         \stanza Mais vous n’avez jamais écrit par quel supplice devait périr celui qui
                              avait tué un  & 
                     parent ou un hôte \&
                         \stanza 
                     ... la mer agitée  \&
                         \stanza 
                     Hélas, comme je suis misérable ! J’agonise! Ils continuent de laver le
                              sang par le sang. \&
                         \stanza 
                     Laquelle accomplit pendant l’union charnelle tes désirs avec retenue
                              et tempérance \&
                         \stanza Ayez pitié d’une vieille femme:  & 
                     donnez-moi le glaive qui m’ôtera la vie \&
                         \stanza Je suis une vieille femme : si seulement je pouvais affronter la mort
                              avant que cela  & 
                     n’arrive. parce que moi, en tant que vieille femme, je me lamenterai
                              dans la pauvreté  \&
                     
                  \endnumbering
		\end{Rightside}
               \end{pairs}
	\Columns
            
            
               \subsection*{Iphigenia}
               \begin{abstract}
                   Sources hellènes : Homère l'\textit{Ἰλιάς}; Euripide
                        \textit{Ἰφιγένεια ἡ ἐν Αὐλίδι}, \\textit{Ἰφιγένεια ἐν Ταύροις}; Eschyle \textit{Ὀρέστεια}, \textit{Ἰφιγένεια}?; Sophocle \textit{Ἰφιγένεια}?..\par
                  Argument: cette tragédie relate le sacrifice d’Iphigénie, que son père commande
                     afin d'obtenir les vents favorables à la navigation. Elle se constitue d’un
                     passage où Achille critique les astrologues, de dialogues entre Agamemnon et
                     une servante, et d'une altercation entre Agamemnon et Ménélas. \par
               \end{abstract}
               \begin{pairs}
                  \begin{Leftside}
			\beginnumbering
			\setcounter{stanzaL}{0}
                     
                         \stanza \ledsidenote{\color{gray}{\textbf{Cicéron, De rep.,I, 13, 30\vspace{-0,5cm}}}}—Achilles— & 
Astrologorum
                              signa
                              in
                              cælo
                              quid
                              sit
                              observationis
                              , & cum
                              Capra
                              aut
                              Nepa
                              aut
                              exoritur
                              nomen
                              aliquod
                              belvarum
                              : & 
                     quod
                              est
                              ante
                              pedes
                              ,
                              nemo
                              spectat
                              ;
                              cæli
                              scrutantur
                              plagas
                              . \&

                         \stanza \ledsidenote{\color{gray}{\textbf{Varron, De ling. lat.,VII, 73M.\vspace{-0,5cm}}}}Quid
                              noctis
                              videtur
                              ?
                              —
                              in
                              altisono & cæli
                              clipeo
                              Temo
                              superat & stellas
                              sublimen
                              agens
                              etiam
                              atque
                              etiam & 
                     noctis
                              iter. \&

                         \stanza \ledsidenote{\color{gray}{\textbf{Cicéron, Tusc.,III, 34, 57\vspace{-0,5cm}}}}—Chorus— & 
Otio
                              qui
                              nescit
                              uti
                              , & plus
                              negoti
                              habet
                              ,
                              quam
                              cum
                              quis
                              negotium
                              in
                              negotio
                              . & Nam
                              cui
                              ,
                              quod
                              agat
                              institutum
                              est
                              ,
                              non
                              ullo
                              negotio; & id
                              agit
                              ,
                              id
                              studet
                              ,
                              ibi
                              mentem
                              atque
                              animum
                              delectat
                              suum
                              . & Otioso
                              in
                              otio
                              animus
                              nescit
                              quid
                              velit
                              . & 
                              Hoc
                              idem
                              est
                              :
                              em
                              neque
                              domi
                              nunc
                              nos
                              nec
                              militiæ
                              sumus
                              : & 
                              imus
                              huc
                              ,
                              hinc
                              illuc
                              ;
                              cum
                              illuc
                              ventum
                              est
                              ,
                              ire
                              illinc
                              lubet
                              . & 
                     Incerte
                              errat
                              animus
                              ,
                              præter
                              propter
                              vitam
                              vivitur
                              . \&
                         \stanza \ledsidenote{\color{gray}{\textbf{Fest. 201 M.\vspace{-0,5cm}}}}—Iphigenia— & 

                     
                              Acherontem
                              obibo
                              ,
                              ubi
                              Mortis
                              thesauri
                              objacent \&
                         \stanza \ledsidenote{\color{gray}{\textbf{Fest. 249 M.\vspace{-0,5cm}}}}procede
                              :
                              gradum
                              proferre
                              pedum & 
                     nitere
                              :
                              cessas
                              ,
                              o
                              fide \&
                         \stanza \ledsidenote{\color{gray}{\textbf{Rufinianus, 11, 205\vspace{-0,5cm}}}}
                     Menelaus
                              me
                              objurgat
                              ?
                              id
                              meis
                              rebus
                              [
                              regimen
                              restitat
                              . \&
                         \stanza \ledsidenote{\color{gray}{\textbf{Fest. 201 M.\vspace{-0,5cm}}}}
                     quæ
                              nunc
                              abs
                              te
                              viduæ
                              et
                              vastæ
                              virgines
                              sunt \&
                     
                  \endnumbering
		\end{Leftside}
                  \begin{Rightside}
			\beginnumbering
			\numberstanzafalse
                     
                         \stanza —Achille— & 
Qu’en est-il de l’observation des signes des astrologues dans le
                              ciel, & quand la constellation de la chèvre ou celle du scorpion tire leur nom
                              de semblable bête & 
                     ce qui est devant les pieds, personne ne regarde; ils explorent les
                              étendues du ciel. \&
                         \stanza Quel moment de la nuit apparait ? — dans & dans la voûte éclatante du ciel, la Grande & Ourse surpasse les étoiles, faisant subli- & 
                     mement maintes et maintes fois le chemin de la nuit. \&
                         \stanza —Chœur— & 
Qui ne sait pas utiliser son repos a plus de  & travail que celui qui a le travail pour occupation. &  Car parce qu’il se fait un plan d'occupations, il n’a aucun
                              travail; &  il le fait, il s’y applique, il ravit alors son esprit et son âme.  & Dans le repos oisif, l’âme ne sait pas ce qu’elle veut. & C'est la même chose : Voilà, à présent, nous ne sommes ni dans notre
                              foyer ni en campagne :  & nous allons là, de part et d’autre; quand nous sommes là, nous
                              souhaitons aller ici. & 
                      L’âme erre d’une manière incertaine; elle existe, à proximité,
                              indépendamment de la vie. \&
                         \stanza —Iphigenie— & 

                     J’irai vers l’Acheron, là où se trouvent les trésors de la mort.  \&
                         \stanza avancez-vous, engagez la marche de vos & 
                     pieds avance-toi en avant, engage la marche de tes pieds, fais un
                              effort, avance-toi, ô fidèle \&
                         \stanza 
                     Ménélas me blâme, il réprouve cette conduite pour mon affaire. \&
                         \stanza 
                     celles-ci, maintenant privées de toi, sont des vierges dévastées \&
                     
                  \endnumbering
		\end{Rightside}
               \end{pairs}
	\Columns
            
            
               \subsection*{Medea exul}
               \begin{abstract}
                  Sources hellènes : les tragédies homonymes d’Euripide, Néophron, Mélanthius,
                     Morsimus, Dicæogenes, Carcinos le Jeune, Théodoride, Diogène de Sinope. L'œuvre
                     d'Ennius, par sa proximité avec la tragédie d'Euripide, s'apparente à une
                     traduction.\par
                   Argument: tel que la \textit{Μήδεια} d’Euripide, le drame
                     se concentre sur la vengeance de Médée délaissée par son époux pour la
                     princesse de Corinthe.\par
               \end{abstract}
               \begin{pairs}
                  \begin{Leftside}
			\beginnumbering
			\setcounter{stanzaL}{0}
                     
                         \stanza \ledsidenote{\color{gray}{\textbf{Fest. 201M\vspace{-0,5cm}}}}Utinam
                              ne
                              in
                              nemore
                              Pelio
                              securibus & cæsa
                              accidisset
                              abiegna
                              ad
                              terram
                              trabes
                              ; & neve
                              inde
                              navis
                              inchoandi
                              exordium & cœpisset
                              ,
                              quæ
                              nunc
                              nominatur
                              nomine & argo
                              ;
                              quia
                              Argivi
                              in
                              ea
                              delecti
                              viri & vecti
                              petebant
                              pellem
                              inauratam
                              arietis & colchis
                              ,
                              imperio
                              regis
                              Peliæ
                              ,
                              per
                              dolum
                              . & Nam
                              numquam
                              era
                              errans
                              mea
                              domo
                              ecferret
                              pedem
                              : & 
                     Medea
                              ,
                              animo
                              ægra
                              ,
                              amore
                              sævo
                              saucia
                              . \&
                         \stanza \ledsidenote{\color{gray}{\textbf{Cicéron, Ad famil., VII 6\vspace{-0,5cm}}}}
                              Quæ
                              Corinthum
                              arcem
                              altam
                              habetis
                              ,
                              matronæ
                              opulentæ
                              ,
                              optumates & {
                              .
                              .
                              .
                              } & 
                              Multi
                              suam
                              rem
                              bene
                              gessere
                              et
                              publicam
                              patria
                              procul
                              ; & multi
                              ,
                              qui
                              domi
                              ætatem
                              agerent
                              ,
                              propterea
                              sunt
                              inprobati. & {
                              .
                              .
                              .
                              } & 
                     qui
                              ipse
                              sibi
                              sapiens
                              prodesse
                              non
                              quit
                              ,
                              nequiquam
                              sapit \&
                         \stanza \ledsidenote{\color{gray}{\textbf{Cicéron, Tusc., III, 26, 63\vspace{-0,5cm}}}}—Nutrix— & 
Cupido
                              cepit
                              miseram
                              nunc
                              me
                              proloqui & 
                     Cælo
                              atque
                              terræ
                              Medeai
                              miserias
                              . \&

                         \stanza \ledsidenote{\color{gray}{\textbf{Cicéron, Tusc., IV, 32, 69\vspace{-0,5cm}}}}—Jason— & 

                     Tu
                              me
                              amoris
                              magis
                              quam
                              honoris
                              servavisti
                              gratia \&
                         \stanza \ledsidenote{\color{gray}{\textbf{Varron, De ling. lat., VI, 81 M\vspace{-0,5cm}}}}—Medea— & 
nam
                              ter
                              sub
                              armis
                              malim
                              vitam
                              cernere
                              , & 
                     quam
                              semel
                              modo
                              parere
                              . \&
                         \stanza \ledsidenote{\color{gray}{\textbf{Non. 470. 6\vspace{-0,5cm}}}}asta
                              atque
                              Athenas
                              anticum
                              opulentum
                              [
                              oppidum & 
                     contempla
                              et
                              templum
                              Cereris
                              ad
                              lævam
                              aspice \&
                         \stanza \ledsidenote{\color{gray}{\textbf{Probus, Verg. Ecl., 6 31\vspace{-0,5cm}}}}Iuppiter
                              tuque
                              {tuque}
                              adeo
                              summe
                              Sol
                              qui
                              omnis
                              inspicis
                              , & quique
                              {quique}
                              lumine
                              tuo
                              mare
                              terram
                              cælum
                              [
                              contines
                              , & 
                     inspice
                              hoc
                              facinus
                              ,
                              prius
                              quam
                              fiat
                              :
                              prohibessis
                              scelus
                              . \&
                         \stanza \ledsidenote{\color{gray}{\textbf{Non. 39. 1\vspace{-0,5cm}}}}antiqua
                              erilis
                              fida
                              custos
                              corporis & 
                     
                              quid
                              sic
                              te
                              extra
                              ædis
                              exanimatam
                              eliminas
                              ? \&
                         \stanza \ledsidenote{\color{gray}{\textbf{Non. 39. 1\vspace{-0,5cm}}}}.
                              .
                              .
                              saluete
                              ,
                              optima
                              corpora
                              , & 
                     cette
                              manus
                              vestras
                              measque
                              {measque}
                              accipite \&

                         \stanza \ledsidenote{\color{gray}{\textbf{Non. 170. 10\vspace{-0,5cm}}}}
                     sol
                              ,
                              qui
                              candentem
                              in
                              cælo
                              sublimat
                              facem \&
                         \stanza \ledsidenote{\color{gray}{\textbf{Non. 297. 16\vspace{-0,5cm}}}}
                     utinam
                              ne
                              umquam
                              ,
                              Medea
                              Colchis
                              ,
                              [
                              cupido
                              corde
                              pedem
                              extulisses \&
                         \stanza \ledsidenote{\color{gray}{\textbf{Non. 467. 15\vspace{-0,5cm}}}}
                     .
                              .
                              .
                              fructus
                              verborum
                              aures
                              aucupant \&
                     
                  \endnumbering
		\end{Leftside}
                  \begin{Rightside}
			\beginnumbering
			\numberstanzafalse
                     
                         \stanza Si seulement, le sapin abattu, dans la fo- & rêt du mont Pélion, n’avait pas été coupé  & sur le sol en poutre par des haches; et qu’à  & partir de là, le bateau n’eût commencé le  & début de l’entreprise; celui-ci maintenant  & est nommé Argo parce que recrutés viennent & d’Argos transportés par le bateau, ils convoi- & taient la Toison d'Or du bélier aux habitants de la Colchide, par le
                              commandement & 
                     du roi Pélias, grâce à la ruse. Car jamais ma souveraine errante ne
                              sortait un pied de sa maison : Médée, malade dans son âme, blessée
                              dans son amour. \&
                         \stanza Vous qui avez la haute citadelle de Corinthe, dames riches et
                              nobles &  \{...\} & Beaucoup ont convenablement exécuté leur affaire publique et loin de
                              la patrie; & beaucoup de ceux qui font leur vie dans leur maison sont décriés à
                              cause de cela. & \{...\} & 
                     Celui qui est lui-même sage ne peut être profitable s’il a vainement
                              de la sagesse pour lui-même. \&
                         \stanza —Nourrice— & 
Maintenant, le désir m’a saisie, miséra ble, d’exposer au ciel et à la
                              terre les malheurs & 
                     de Médée \&
                         \stanza —Jason— & 

                     Tu m’as sauvé plus au profit de l’amour que de l’honneur. \&
                         \stanza —Médée— & 
En réalité, j’aimerais mieux trois fois dis- & 
                     poser ma vie sous les armes que de seulement me soumettre une
                              fois. \&
                         \stanza Lève-toi et admire Athènes, somptueuse et antique acropole, puis
                              examine à gau-  & 
                     che le temple de Cérès. \&
                         \stanza Et toi, grand Jupiter, de surcroît, soleil qui observe toutes choses,  &  et toi qui préserves par ta lumière la mer, la terre et le ciel, & 
                     observe cet acte avant qu’il ne se produise : tu peux empêcher un
                              crime. \&
                         \stanza fidèle et ancienne garde du corps de la & 
                     maîtresse, qu’est-ce qui te fait ainsi sortir, essoufflée, à
                              l’extérieur de la maison \&
                         \stanza  ... adieu, êtres chers, & 
                     donnez-moi votre main prenez la mienne \&
                         \stanza 
                     le soleil qui élève son flambeau ardent dans le ciel \&
                         \stanza 
                     puisses-tu un jour, Médée de Colchide, ne plus faire avancer tes pas
                              avec ton coeur passionné \&
                         \stanza 
                     ... les oreilles épient les fruits des mots \&
                     
                  \endnumbering
		\end{Rightside}
               \end{pairs}
	\Columns
            

            
               \subsection*{Phœnix}
               \begin{abstract}
                  Sources hellènes : Homère l'\textit{Ἰλιάς}, ainsi que des
                     tragédies homonymes de Sophocle, Euripide, Ion et Astydamas.\par
                   Argument: suivant les conseils de sa mère, Phénix cherche à obtenir les
                     faveurs de la concubine de son père. Quand celui-ci apprend les faits, il entre
                     dans une colère immodérée et rend son fils aveugle afin qu’il ne puisse plus
                     être charmé par la concubine. Cette tragédie se concentrait sur le conflit qui
                     oppose Phénix et son père Amyntor. \par
               \end{abstract}
               \begin{pairs}
                  \begin{Leftside}
			\beginnumbering
			\setcounter{stanzaL}{0}
                     
                         \stanza \ledsidenote{\color{gray}{\textbf{Gel. VI, 258\vspace{-0,5cm}}}}Sed
                              virum
                              vera
                              virtute
                              vivere
                              animatum
                              addecet
                              , & fortiterque
                              {fortiterque}
                              ,
                              innoxium
                              astare
                              adversum
                              adversarios
                              . & Ea
                              libertas
                              est
                              qui
                              pectus
                              purum
                              et
                              firmum
                              gestitat
                              ; & 
                     
                              aliæ
                              res
                              obnoxiosæ
                              nocte
                              in
                              obscura
                              latent
                              . \&

                         \stanza \ledsidenote{\color{gray}{\textbf{Non. 512, 8\vspace{-0,5cm}}}}
                     quam
                              tibi
                              ex
                              ore
                              orationem
                              duriter
                              dictis
                              dedit \&
                         \stanza \ledsidenote{\color{gray}{\textbf{Non. 91, 8\vspace{-0,5cm}}}}
                     stultus
                              est
                              qui
                              †
                              cupida
                              †
                              cupiens
                              cupienter
                              cupit
                              . \&
                         \stanza \ledsidenote{\color{gray}{\textbf{Non. 245, 25\vspace{-0,5cm}}}}
                     †
                              tum
                              tu
                              isti
                              credere
                              †
                              atque
                              exerce
                              linguam
                              ut
                              argutarier
                              possis
                              . \&
                         \stanza \ledsidenote{\color{gray}{\textbf{Non. 507, 22\vspace{-0,5cm}}}}
                     
                              plus
                              miser
                              sim
                              ,
                              si
                              scelestum
                              faxim
                              quod
                              [
                              dicam
                              fore \&
                         \stanza \ledsidenote{\color{gray}{\textbf{Non. 511, 5\vspace{-0,5cm}}}}
                     sæviter
                              suspicionem
                              ferre
                              falsam
                              futtilum
                              est \&
                         \stanza \ledsidenote{\color{gray}{\textbf{Non. 514, 13\vspace{-0,5cm}}}}
                     Ut
                              quod
                              factum
                              est
                              futtile
                              ,
                              amici
                              ,
                              vos
                              feratis
                              fortiter
                              . \&
                         \stanza \ledsidenote{\color{gray}{\textbf{Non. 518, 5\vspace{-0,5cm}}}}
                     Ibi
                              tum
                              derepente
                              ex
                              alto
                              in
                              altum
                              despexit
                              mare
                              . \&
                     
                  \endnumbering
		\end{Leftside}
                  \begin{Rightside}
			\beginnumbering
			\numberstanzafalse
                     
                         \stanza Mais il convient à un homme de vivre animé par une véritable vertu,  & et qu’il soit sans reproche pour se dresser vaillamment contre les
                              adversaires. & C’est cette liberté qui porte un coeur pur et solide;  & 
                     les autres faits reprochables sont cachés dans la nuit obscure. \&
                         \stanza 
                     que le discours qu’il t’a livré de sa bouche avec des paroles
                              rustres \&
                         \stanza 
                     est sot celui qui envieux convoite avidement les passions. \&
                         \stanza 
                      † alors tu crois en cela †, et exerces ta langue afin de pouvoir
                              bavarder. \&
                         \stanza 
                     je serais plus malheureux si je faisais le crime que j’annonce
                              arriver \&
                         \stanza 
                     il est inutile de colporter ardemment une fausse suspicion \&
                         \stanza 
                     Étant donné que ce qui a été fait vainement, vous, mais amis, vous le
                              rapportez vigoureusement. \&
                         \stanza 
                     Là alors, il a regardé en direction de la haute mer dans sa
                              profondeur. \&
                     
                  \endnumbering
		\end{Rightside}
               \end{pairs}
	\Columns
            
            
               \subsection*{Telamo}
               \begin{abstract}
                  Sources hellènes : Il n’y a pas de tragédie connue ayant le même titre; le
                     mythe est toutefois développé par Sophocle dans \textit{Τεῦκρος}, et quelques éléments dans \textit{Αἴας}\par
                  Argument: cette pièce expose le dépit et la contrariété de Télamon qui voit son
                     fils Teucros revenir au foyer sans avoir vengé l’injustice relative à la mort
                     d’Ajax. Les fragments de la pièce exposent une altercation entre Télamon et
                     Teucros, une réflexion de Télamon sur la bienveillance des dieux, et la
                     description de l’affliction d’une femme.\par
               \end{abstract}
               \begin{pairs}
                  \begin{Leftside}
			\beginnumbering
			\setcounter{stanzaL}{0}
                     
                         \stanza \ledsidenote{\color{gray}{\textbf{Cicéron, De Nat. D., 3, 79\vspace{-0,5cm}}}}
                     
                              Nam
                              si
                              curent
                              bene
                              bonis
                              sit
                              ,
                              male
                              malis
                              ;
                              quod
                              nunc
                              abest \&
                         \stanza \ledsidenote{\color{gray}{\textbf{Cicéron, De Div, II, 58, 132\vspace{-0,5cm}}}}autinertes
                              aut
                              insani
                              aut
                              quibus
                              egestas
                              imperat
                              , & qui
                              sibi
                              semitam
                              non
                              sapiunt
                              ,
                              alteri
                              monstrant
                              viam
                              ; & quibus
                              divitias
                              pollicentur
                              ,
                              ab
                              iis
                              drachumam
                              ipsi
                              petunt
                              . & 
                     De
                              his
                              divitiis
                              ,
                              sibi
                              deducant
                              drachumam
                              ,
                              reddant
                              cetera
                              . \&
                         \stanza \ledsidenote{\color{gray}{\textbf{Cicéron, De Div, II, 50, 104\vspace{-0,5cm}}}}Ego
                              deum
                              genus
                              esse
                              semper
                              dixi
                              et
                              dicam
                              cælitum
                              , & 
                     sed
                              eos
                              non
                              curare
                              opinor
                              ,
                              quid
                              agat
                              humanum
                              genus \&
                         \stanza \ledsidenote{\color{gray}{\textbf{Fest 198 M.\vspace{-0,5cm}}}}
                     
                              Scibas
                              natum
                              ingenuum
                              Aiacem
                              ,
                              cui
                              tu
                              [
                              obsidionem
                              paras
                              . \&
                         \stanza \ledsidenote{\color{gray}{\textbf{Diomède, I, 378P\vspace{-0,5cm}}}}
                     Abnuebunt \&
                         \stanza \ledsidenote{\color{gray}{\textbf{Non. 85,3\vspace{-0,5cm}}}}nam
                              ita
                              mihi
                              Telamonis
                              patris
                              ,
                              avi
                              [
                              Æaci
                              et
                              proavi
                              Iovis & 
                     gratia
                              †
                              ea
                              est
                              †
                              atque
                              hoc
                              lumen
                              candidum
                              claret
                              mihi \&
                         \stanza \ledsidenote{\color{gray}{\textbf{Non., 160,4\vspace{-0,5cm}}}}
                     deum
                              me
                              sentit
                              facere
                              pietas
                              ,
                              civium
                              [
                              porcet
                              pudor \&
                         \stanza \ledsidenote{\color{gray}{\textbf{Non., 172,20\vspace{-0,5cm}}}}
                     strata
                              terræ
                              lavere
                              lacrimis
                              vestem
                              [
                              squalam
                              et
                              sordidam \&
                         \stanza \ledsidenote{\color{gray}{\textbf{Non. 475,25\vspace{-0,5cm}}}}
                     
                              eandem
                              me
                              in
                              suspicionem
                              sceleris
                              partivit
                              pater \&
                         \stanza \ledsidenote{\color{gray}{\textbf{Non., 506,1\vspace{-0,5cm}}}}
                     More
                              antiquo
                              audibo
                              atque
                              auris
                              tibi
                              [
                              contra
                              utendas
                              dabo
                              . \&
                     
                  \endnumbering
		\end{Leftside}
                  \begin{Rightside}
			\beginnumbering
			\numberstanzafalse
                     
                         \stanza 
                     Car s’ils prennent soin que tout aille bien pour les bons, et mal pour
                              les mauvais; cela fait actuellement défaut \&
                         \stanza ou ils sont paralysés, ou ils sont fous, ou la pauvreté les gouverne
                              eux qui ne connaissent & pas le chemin pour eux-mêmes, indiquent à autrui le trajet; &  ceux à qui ils présentent leurs richesses, ils leur demandent
                              eux-mêmes un drachme. & 
                     De ces richesses, qu’ils déduisent un drachme pour eux, du reste
                              qu’ils le restituent. \&
                         \stanza Moi, j’ai toujours dit qu’il y a une race de dieux, et je dirai qu’ils
                              sont célestes, mais  & 
                     je pense qu’ils ne se soucient pas de ce que la race humaine fait. \&
                         \stanza 
                     Vous savez qu’Ajax provient d’une noble naissance, à laquelle tu fais
                              obstacle. \&
                         \stanza 
                     Ils nieront  \&
                         \stanza car ainsi la grâce de mon père Télamon, de mon grand-père Éaque, et de
                              mon arrière- & 
                     grand-père Jupiter, † est là †, et cette lumière radieuse
                              m’illumine \&
                         \stanza 
                     Ma piété des dieux considère que je dois le faire, mon respect des
                              citoyens m’empêche de le faire  \&
                         \stanza 
                     étendue sur le sol pour baigner de larmes son habit sale et
                              répugnant \&
                         \stanza 
                     Mon père m’a aussi attribué le soupçon du crime \&
                         \stanza 
                     Selon l’usage d’autrefois, j’écouterai et je te donnerai mes oreilles
                              pour que tu les emploies. \&
                     
                  \endnumbering
		\end{Rightside}
               \end{pairs}
	\Columns
            
            
               \subsection*{Thyestes}
               \begin{abstract}
                  Sources hellènes : les tragédies homonymes d’Euripide, Sophocle, Agathon,
                     Diogène Apollodore, Carcinos le Jeune, Chérémon d'Athènes, Cléophon, Diogène de
                     Sinope.\par
                  Argument: la tragédie a pour sujet le repas de réconciliation servi à Thyeste
                     par Atrée; or, celui-ci, convoitant plus la vengeance que le pardon, sert à son
                     frère un ragoût cuisiné avec la chair de ses enfants. Les fragments révèlent
                     qu’Ennius souligne le rôle du destin dans les épreuves que subit Thyeste, ainsi
                     que celui d’Apollon. Thyeste doit faire\par
               \end{abstract}
               \begin{pairs}
                  \begin{Leftside}
			\beginnumbering
			\setcounter{stanzaL}{0}
                     
                         \stanza \ledsidenote{\color{gray}{\textbf{Cicéron, Tusc. I,44,107\vspace{-0,5cm}}}}—Thyeste— & 
ipse
                              summis
                              saxis
                              fixus
                              asperis
                              ,
                              evisceratus
                              , & 
                              latere
                              pendens
                              ,
                              saxa
                              spargens
                              tabo
                              ,
                              sanie
                              et
                              sanguine
                              atro
                              , & 
                              {
                              .
                              .
                              .
                              } & neque
                              sepulcrum
                              quo
                              recipiat
                              ,
                              habeat
                              portum
                              corporis
                              , & 
                     ubi
                              remissa
                              humana
                              vita
                              corpus
                              requiescat
                              malis
                              . \&
                         \stanza \ledsidenote{\color{gray}{\textbf{Cicéron, De Orat., 55,185\vspace{-0,5cm}}}}
                     
                              quemnam
                              te
                              esse
                              dicam
                              ,
                              qui
                              tarda
                              in
                              [
                              senectute \&
                         \stanza \ledsidenote{\color{gray}{\textbf{Fest. 306 M.\vspace{-0,5cm}}}}
                     aspice
                              hoc
                              sublime
                              candens
                              quem
                              vocant
                              omnes
                              Iovem \&
                         \stanza \ledsidenote{\color{gray}{\textbf{Non. 90,15\vspace{-0,5cm}}}}
                     Eheu
                              !
                              mea
                              fortuna
                              ,
                              ut
                              omnia
                              in
                              me
                              conglomeras
                              mala \&
                         \stanza \ledsidenote{\color{gray}{\textbf{Non. 97,32\vspace{-0,5cm}}}}
                     set
                              me
                              Apollo
                              ipsus
                              delectat
                              :
                              ductat
                              Delphicus \&
                         \stanza \ledsidenote{\color{gray}{\textbf{Non. 97,32\vspace{-0,5cm}}}}
                     sin
                              flaccebunt
                              conditiones
                              ,
                              repudiato
                              et
                              [
                              ]
                              reddito \&
                         \stanza \ledsidenote{\color{gray}{\textbf{Non. 255,25\vspace{-0,5cm}}}}
                     sed
                              sonitus
                              auris
                              meas
                              pedum
                              pulsu
                              increpat \&
                         \stanza \ledsidenote{\color{gray}{\textbf{Non. 261,15\vspace{-0,5cm}}}}
                     
                              impetrem
                              facile
                              ab
                              animo
                              ut
                              cernat
                              vitalem
                              †
                              babium
                              † \&
                         \stanza \ledsidenote{\color{gray}{\textbf{Non. 268,12\vspace{-0,5cm}}}}
                     Quam
                              mihi
                              maximum
                              hodie
                              hic
                              contigerit
                              malum
                              ? \&
                         \stanza \ledsidenote{\color{gray}{\textbf{Non. 369,25\vspace{-0,5cm}}}}
                     ibi
                              quid
                              agat
                              secum
                              {secum}
                              cogitat
                              :
                              parat
                              ,
                              putat
                              . \&
                     
                  \endnumbering
		\end{Leftside}
                  \begin{Rightside}
			\beginnumbering
			\numberstanzafalse
                     
                         \stanza —Thyeste— & 
lui-même fixé sur les hauts rochers rugueux, éventré, & gisant sur le côté, répandant sur les rochers un fluide purulant, et
                              du sang noir, & \{...\} & ni sépulcre pour qu’il puisse se reposer, pour qu’il trouve le port de
                              son corps,  & 
                      lorsque, la vie humaine est rejetée, le corps se repose des
                              malheurs. \&
                         \stanza 
                     que puis-je donc dire que tu es, toi qui languis dans la
                              vieillesse \&
                         \stanza 
                     observe cette chose qui resplendit dans le ciel, ce que tous appellent
                              Jupiter \&
                         \stanza 
                     Ah! Ma destinée, comme tu m’as amassé dans tous les malheurs  \&
                         \stanza 
                     mais l’Apollon lui-même m’attire : Delphicus me dirige  \&
                         \stanza 
                     si au contraire les dispositions font défaut, tu rejetteras et
                              renverras \&
                         \stanza 
                     mais un bruit retentit dans mes oreilles, par le trépignement des
                              pieds  \&
                         \stanza 
                     je peux obtenir aisément de l’esprit qu’il discerne le vital †babium†
                            \&
                         \stanza 
                      Comment aujourd’hui, ici, m’est-il arrivé le plus grand malheur ? \&
                         \stanza 
                     là, il délibère avec lui-même sur ce qu’il doit faire : il envisage,
                              suppose \&
                     
                  \endnumbering
		\end{Rightside}
               \end{pairs}
	\Columns
            
         
         
            \section{Marcus Pacuvius}
            Texte latin établi et organisé selon les choix de reconstruction et d'édition de \cite{SchierlPaca}.\par
            
               \subsection*{Antiopa}
               \begin{abstract}
                  Sources hellènes : Homère l'\textit{Ὀδύσσεια}, ainsi que
                     la tragédie homonyme d’Euripide (traduite parfois mot à mot).\par
                  Argument: la tragédie a pour sujet la réunion d'Antiope, sous le joug Dircé,
                     avec ses fils, les jumeaux Amphion et Zéthos. Ayant pris connaissance de leur
                     filiation, Amphion et Zéthos décident de venger leur mère et parviennent à
                     prendre le trône thébain, lieu où Dircé et Lycos tenaient leur mère
                     captive.\par
               \end{abstract}
               \begin{pairs}
                  \begin{Leftside}
			\beginnumbering
			\setcounter{stanzaL}{0}
                     
                         \stanza \ledsidenote{\color{gray}{\textbf{Cicéron, De div., II, 64, 133\vspace{-0,5cm}}}}—Amphio— —Chorus— —Amphio— & 
Quadrupes
                              tardigrada
                              agrestis
                              humilis
                              aspera
                              , & capite
                              brevi
                              ,
                              cervice
                              anguina
                              ,
                              aspectu
                              truci
                              , & eviscerata
                              inanima
                              cum
                              animali
                              sono
                              . & Non
                              intellegimus
                              ,
                              nisi
                              si
                              aperte
                              dixeris
                              : & 
                              ita
                              sæptuose
                              dictio
                              abs
                              te
                              datur
                              , & 
                              quod
                              coniectura
                              sapiens
                              ægre
                              contuit
                              ; & 
                     
                              testudo \&
                         \stanza \ledsidenote{\color{gray}{\textbf{Gel. XIII 8,4\vspace{-0,5cm}}}}
                     
                              odi
                              ego
                              homines
                              ignava
                              opera
                              et
                              philosopha
                              sententia \&
                         \stanza \ledsidenote{\color{gray}{\textbf{Diomedes I p.336P\vspace{-0,5cm}}}}
                     
                              loca
                              horrida
                              initas \&
                         \stanza \ledsidenote{\color{gray}{\textbf{Non. 447, 16\vspace{-0,5cm}}}}tu
                              pascere & 
                     
                              cornifrontes
                              soles
                              armentas
                              .
                              .
                              .
                            \&
                         \stanza \ledsidenote{\color{gray}{\textbf{Varron, De re rustica, I, 2, 5\vspace{-0,5cm}}}}.
                              .
                              .
                              sol
                              si
                              perpetuo
                              siet
                              , & flammeo
                              vapore
                              torrens
                              terræ
                              fetum
                              exusserit
                              ; & 
                     
                              omnia
                              nisi
                              interveniat
                              sol
                              ,
                              pruina
                              obriguerint
                              . \&
                         \stanza \ledsidenote{\color{gray}{\textbf{Schol. Persii\vspace{-0,5cm}}}}illuvie
                              corporis & 
                     et
                              coma
                              promissa
                              impexa
                              conglomerata
                              atque
                              horrida \&
                         \stanza \ledsidenote{\color{gray}{\textbf{Charisius, I, 78 P\vspace{-0,5cm}}}}
                     .
                              .
                              .
                              perdita
                              inluvie
                              atque
                              insomnia \&
                         \stanza \ledsidenote{\color{gray}{\textbf{Prisc. II 468,27\vspace{-0,5cm}}}}
                     qua
                              te
                              adplicasti
                              tamen
                              ærumnis
                              obruta
                              ? \&
                         \stanza \ledsidenote{\color{gray}{\textbf{Servius, II, 64, 133\vspace{-0,5cm}}}}.
                              .
                              .
                              cervicum & 
                     floros
                              dispendite
                              crines \&
                         \stanza \ledsidenote{\color{gray}{\textbf{Non. 73, 15\vspace{-0,5cm}}}}
                     Nonne
                              hinc
                              vos
                              propere
                              a
                              stabulis
                              amolimini
                              ? \&
                         \stanza \ledsidenote{\color{gray}{\textbf{Varron, VI, 83\vspace{-0,5cm}}}}
                     
                              exorto
                              jubare
                              noctis
                              decurso
                              itinere \&
                         \stanza \ledsidenote{\color{gray}{\textbf{Non. 238, 15\vspace{-0,5cm}}}}
                     Sed
                              cum
                              animum
                              adtendi
                              ad
                              quærendum
                              ,
                              quid
                              siet \&
                         \stanza \ledsidenote{\color{gray}{\textbf{Non. 63, 30\vspace{-0,5cm}}}}
                     Salvete
                              ,
                              gemini
                              ,
                              mea
                              propages
                              sanguinis
                              ! \&
                         \stanza \ledsidenote{\color{gray}{\textbf{Non. 139, 23\vspace{-0,5cm}}}}
                     Minitabiliterque
                              {Minitabiliterque}
                              increpare
                              dictis
                              saevis
                              incipit
                              . \&
                         \stanza \ledsidenote{\color{gray}{\textbf{Non. 447, 16\vspace{-0,5cm}}}}
                     frendere
                              noctes
                              ,
                              misera
                              quas
                              perpessa
                              sum \&
                         \stanza \ledsidenote{\color{gray}{\textbf{Non. 447, 16\vspace{-0,5cm}}}}
                     
                              fruges
                              frendendo
                              sola
                              saxi
                              robore \&
                     
                  \endnumbering
		\end{Leftside}
                  \begin{Rightside}
			\beginnumbering
			\numberstanzafalse
                     
                         \stanza —Amphion— —Choeur— —Amphion— & 
Le quadrupède, à la démarche lente, animal agreste, modeste, bourru,
                              capturez ra- & pidement les serpents par le coup, au regard farouche éventrés,
                              inanimés, avec un &  être sonore & Nous ne comprenons pas excepté si c’est  & exprimé clairement : ainsi tu donnes le dis- & cours d’une manière obscure, qu’un sage observe péniblement
                              l’interprétation ;  & 
                     tortue \&
                         \stanza 
                      Je déteste les hommes indolents dans leur tache et le philosophe dans
                              sa pensée \&
                         \stanza 
                     les lieux sauvages atteints \&
                         \stanza tu as l’habitude de nourrir le troupeau & 
                      aux cornes frontales... \&
                         \stanza  ... si le soleil rayonnait perpétuellement, il  & brûlerait la semence de la terre par une émanation flamboyante ardente
                              ;  & 
                     si le soleil n’intervenait pas, toutes choses seraient figées par le
                              givre \&
                         \stanza avec la saleté du corps et une longue che- & 
                      velure hirsute et sauvage, amassée \&
                         \stanza 
                      ... ruinée par la saleté et l’insomnie \&
                         \stanza 
                     Sur quoi t’es-tu appuyée, alors que tu étais submergée par les
                              malheurs ? \&
                         \stanza ... vous répartissez les cheveux de la tête & 
                      \&
                         \stanza 
                      Ne devez-vous pas vous éloigner précipitamment d’ici, de la
                              bergerie? \&
                         \stanza 
                     quand l’étoile du matin se lève, alors que le trajet de la nuit est
                              parcouru  \&
                         \stanza 
                     Mais quand j’ai tendu mon esprit pour comprendre ce qu’il ferait \&
                         \stanza 
                     Salutations, frères jumeaux, ma descendance de sang! \&
                         \stanza 
                     Il commence à invectiver sévèrement avec des paroles violentes. \&
                         \stanza 
                     je grinçais des dents quand j’ai enduré mes nuits misérables \&
                         \stanza 
                     en broyant les céréales par la seule force de la pierre \&
                     
                  \endnumbering
		\end{Rightside}
               \end{pairs}
	\Columns
            
            
               \subsection*{Armorum Judicium}
               \begin{abstract}
                  \textit{}Homère l'\textit{Ὀδύσσεια} et
                        l'\textit{Ἰλιάς}, Arctinos de Milet \textit{Αἰθιοπίς}; Eschyle \textit{Ὅπλων
                        κρίσισ}, les versions d'\textit{Αἴας} de
                     Sophocle, Théodecte de Phasélis et Carcinos le Jeune.\par
                   Argument: cette tragédie fait part du conflit entre Ulysse et Ajax concernant
                     le partage des armes d’Achille.\par
                  La tragédie débuterait par les jeux funéraires tenus en l’honneur d’Achille qui
                     se concluent par la victoire d’Ulysse à ces jeux. Ajax exprime alors son
                     irritation et réclame les armes qui lui étaient promises. Puis, la pièce devait
                     certainement représenter la folie d’Ajax ainsi que son suicide.\par
               \end{abstract}
               \begin{pairs}
                  \begin{Leftside}
			\beginnumbering
			\setcounter{stanzaL}{0}
                     
                         \stanza \ledsidenote{\color{gray}{\textbf{Charisius, II, 175P\vspace{-0,5cm}}}}
                     
                              .
                              .
                              .
                              seque
                              {seque}
                              ad
                              ludos
                              iam
                              inde
                              abhinc
                              exerceant \&
                         \stanza \ledsidenote{\color{gray}{\textbf{Non 126, 23\vspace{-0,5cm}}}}
                     
                              quod
                              ego
                              inaudivi
                              accipite
                              ,
                              et
                              quid
                              sit
                              facto
                              opus
                              decernite \&
                         \stanza \ledsidenote{\color{gray}{\textbf{Non 89, 11\vspace{-0,5cm}}}}.
                              .
                              .
                              qui
                              sese
                              †
                              adfinem
                              †
                              esse
                              ad
                              causandum
                              volunt
                              , & 
                     de
                              virtute
                              is
                              ego
                              cernundi
                              do
                              potestatem
                              omnibus \&
                         \stanza \ledsidenote{\color{gray}{\textbf{Non 416, 1\vspace{-0,5cm}}}}
                     qui
                              viget
                              ,
                              vescatur
                              armis
                              ,
                              ut
                              percipiat
                              præmium \&
                         \stanza \ledsidenote{\color{gray}{\textbf{Non 759\vspace{-0,5cm}}}}
                     
                              an
                              quis
                              est
                              qui
                              te
                              esse
                              dignum
                              quicum
                              {quicum}
                              certetur
                              putet
                              ? \&
                         \stanza \ledsidenote{\color{gray}{\textbf{Fest 281\vspace{-0,5cm}}}}
                     tuque
                              {tuque}
                              te
                              desider
                              e
                              residem
                              †
                              nos
                              hic
                              esse
                              in
                              ma† \&
                         \stanza \ledsidenote{\color{gray}{\textbf{Non 507,25\vspace{-0,5cm}}}}
                     .
                              .
                              dic
                              quid
                              faciam
                              :
                              quod
                              me
                              moneris
                              ,
                              effectum
                              dabo
                              . \&
                         \stanza \ledsidenote{\color{gray}{\textbf{Non 475,18\vspace{-0,5cm}}}}
                     proloqui
                              nunc
                              pænitebunt
                              libere
                              in
                              grato
                              ex
                              loco
                              . \&
                         \stanza \ledsidenote{\color{gray}{\textbf{Non 261, 13\vspace{-0,5cm}}}}.
                              .
                              .
                              et
                              aecum
                              et
                              rectum
                              est
                              id
                              quod
                              postulas
                              : & 
                     
                              jurati
                              cernant
                              ! \&
                         \stanza \ledsidenote{\color{gray}{\textbf{Fest. 278\vspace{-0,5cm}}}}
                     .
                              .
                              .
                              si
                              non
                              est
                              ingratum
                              reapse
                              quod
                              feci
                              bene
                              ! \&
                         \stanza \ledsidenote{\color{gray}{\textbf{Suetonius, Cæs., 84 \vspace{-0,5cm}}}}
                     .
                              .
                              .
                              men
                              {men}
                              servasse
                              ,
                              ut
                              essent
                              qui
                              me
                              perderent
                              ? \&
                         \stanza \ledsidenote{\color{gray}{\textbf{Non 534,32\vspace{-0,5cm}}}}
                     .
                              .
                              .
                              feroci
                              ingenio
                              ,
                              torvus
                              ,
                              prægrandi
                              gradu
                              . \&
                         \stanza \ledsidenote{\color{gray}{\textbf{Fest., 355\vspace{-0,5cm}}}}
                     cum
                              recordor
                              eius
                              ferocem
                              et
                              torvam
                              confidentiam \&
                         \stanza \ledsidenote{\color{gray}{\textbf{Non 126,23\vspace{-0,5cm}}}}nam
                              canis
                              ,
                              quando
                              est
                              percussa
                              lapide
                              ,
                              non
                              tam
                              illum
                              adpetit & 
                     qui
                              sese
                              icit
                              ,
                              quam
                              illum
                              eumpse
                              {eumpse}
                              lapidem
                              ,
                              qui
                              ipsa
                              icta
                              est
                              petit \&
                         \stanza \ledsidenote{\color{gray}{\textbf{Non 246,4\vspace{-0,5cm}}}}
                     
                              Pro
                              imperio
                              agendum
                              est
                              .
                              Quis
                              vetat
                              ,
                              quisve
                              {quisve}
                              attolat \&
                     
                  \endnumbering
		\end{Leftside}
                  \begin{Rightside}
			\beginnumbering
			\numberstanzafalse
                     
                         \stanza 
                      . . . . dès lors, ils doivent s’entraîner immédiatement aux jeux. \&
                         \stanza 
                      entends ce que moi j’ai entendu et décide du travail qui doit être
                              fait \&
                         \stanza Je donne à tous ceux qui désirent être à la frontière pour se mettre
                              en avant, la & 
                     possibilité de combattre pour la vertu \&
                         \stanza 
                     que celui qui prédomine jouisse des armes, de telle sorte qu’il les
                              recevra comme récompense \&
                         \stanza 
                     Est-ce qu’il y a quelqu’un qui t’estimera être digne de combattre
                              contre lui ? \&
                         \stanza 
                     Et toi, tu restes indolent, † nous sommes ici dans † \&
                         \stanza 
                     Dis ce que je dois faire, ce que tu m’engages à effectuer, je
                              l’accorderai \&
                         \stanza 
                     à présent, ils regretteront d’exposer ouvertement à voix haute, depuis
                              leur place ingrate \&
                         \stanza  Et ce que tu demandes est équitable et juste : & 
                      que les juges statuent ! \&
                         \stanza 
                     ... si ce que j’ai bien fait n’est effectivement aucunement ingrat
                              ! \&
                         \stanza 
                     ... est-ce que je les ai sauvés pour qu’ils se révèlent être ceux qui
                              m’anéantiront ? \&
                         \stanza 
                     . . . par son caractère fougueux, farouche, avec un pas imposant  \&
                         \stanza 
                     lorsque je me rappelle sa fougue et son assurance farouche \&
                         \stanza de fait, quand une chienne est frappée par une pierre, elle n’attaque
                              pas autant celui & 
                     qui l’a seulement lancé, qu’elle attaque cette pierre même qui a été
                              lancée sur elle-même. \&
                         \stanza 
                     Il faut agir pour le pouvoir. Qui le défend ou qui l’occasionne ? \&
                     
                  \endnumbering
		\end{Rightside}
               \end{pairs}
	\Columns
            
            
               \subsection*{Atalanta}
               \begin{abstract}
                  Sources hellènes : Eschyle et Aristias aurait composé une tragédie nommée
                        \textit{`Αταλάντη}.\par
                  Argument: cette tragédie se concentre sur le fils d’Atalante, Parthénopée.
                     Abandonné à la naissance sur le mont Parthénius, il est recueilli par le berger
                     Télèphe — fils d’Héraclès il fût lui aussi abandonné et recueilli par des
                     bergers. Adolescent, Parthénius recherche ses parents; l’anagnorisis
                     adviendrait lors d’une compétition où la mère concourt contre le fils. \par
               \end{abstract}
               \begin{pairs}
                  \begin{Leftside}
			\beginnumbering
			\setcounter{stanzaL}{0}
                     
                         \stanza \ledsidenote{\color{gray}{\textbf{Paul., Epitomes. 83. L\vspace{-0,5cm}}}}
                     nunc
                              primum
                              opacat
                              flore
                              lanugo
                              genas \&
                         \stanza \ledsidenote{\color{gray}{\textbf{Non.794-795\vspace{-0,5cm}}}}
                     parentum
                              incertum
                              investigandum
                              gratia \&
                         \stanza \ledsidenote{\color{gray}{\textbf{Non.787\vspace{-0,5cm}}}}
                     dolet
                              pigetque
                              {pigetque}
                              magisque
                              {magisque}
                              me
                              conatum
                              hoc
                              nequiquam
                              itiner \&
                         \stanza \ledsidenote{\color{gray}{\textbf{Non.378,6\vspace{-0,5cm}}}}etsi
                              metuo
                              picta
                              de
                              plaga
                              palam & 
                      \&
                         \stanza \ledsidenote{\color{gray}{\textbf{Non.262,10\vspace{-0,5cm}}}}
                              gradere
                              atque
                              atrocem
                              coerce
                              confidentiam & 
                      \&
                         \stanza \ledsidenote{\color{gray}{\textbf{Non.65,1\vspace{-0,5cm}}}}
                              .
                              .
                              .
                              .
                              extremum
                              intra
                              camterem
                              ipsum
                              prægradat & 
                     Parthenopaeum \&
                         \stanza \ledsidenote{\color{gray}{\textbf{Non.692\vspace{-0,5cm}}}}
                     
                              hic
                              sollicita
                              ,
                              studio
                              obstupida
                              ,
                              suspenso
                              animo
                              civitas
                              ! \&
                         \stanza \ledsidenote{\color{gray}{\textbf{Non.322,17\vspace{-0,5cm}}}}
                     Quæ
                              ægritudo
                              insolens
                              mentem
                              adtemptat
                              tuam
                              ? \&
                         \stanza \ledsidenote{\color{gray}{\textbf{Non.226,32\vspace{-0,5cm}}}}.
                              .
                              .
                              dubito
                              quam
                              insistam
                              viam & 
                     aut
                              quod
                              primordium
                              capissam
                              ad
                              stirpem
                              exquirendum
                              .
                              .
                              . \&
                         \stanza \ledsidenote{\color{gray}{\textbf{Non.116,18\vspace{-0,5cm}}}}
                     
                              habeo
                              ego
                              istam
                              qui
                              distinguam
                              inter
                              vos
                              geminitudinem
                              . \&
                         \stanza \ledsidenote{\color{gray}{\textbf{Non.85,4\vspace{-0,5cm}}}}is
                              vestrorum
                              uter
                              sit
                              ,
                              cui
                              signum
                              datum
                              est
                              ; & 
                     
                              cette \&
                         \stanza \ledsidenote{\color{gray}{\textbf{Fest., 375\vspace{-0,5cm}}}}
                     suspensum
                              in
                              lævo
                              bracchio
                              ostendo
                              ungulum \&
                         \stanza \ledsidenote{\color{gray}{\textbf{Non.182,1\vspace{-0,5cm}}}}
                     
                              Quid
                              istuc
                              est
                              ?
                              Vultum
                              alligat
                              quæ
                              tristitas
                              ? \&
                         \stanza \ledsidenote{\color{gray}{\textbf{Non.160,1\vspace{-0,5cm}}}}
                     
                              Mi
                              gnate
                              ,
                              ut
                              vereor
                              !
                              eloqui
                              ,
                              porcet
                              pudor \&
                         \stanza \ledsidenote{\color{gray}{\textbf{Non.512,27\vspace{-0,5cm}}}}
                     concertare
                              ac
                              dissentire
                              partim
                              ac
                              da
                              rursum
                              æquiter \&
                         \stanza \ledsidenote{\color{gray}{\textbf{Non. 505,3\vspace{-0,5cm}}}}nam
                              quod
                              conabar
                              ,
                              cum
                              interventum
                              est
                              dicere
                              , & 
                     
                              nunc
                              expedibo \&
                         \stanza \ledsidenote{\color{gray}{\textbf{Non.486,3\vspace{-0,5cm}}}}ubi
                              ego
                              me
                              gravidam
                              sentio
                              adgravescere & 
                     
                              propinquitate
                              parti
                              .
                              .
                              . \&
                         \stanza \ledsidenote{\color{gray}{\textbf{Non.382,7\vspace{-0,5cm}}}}
                     cum
                              incultos
                              pervestigans
                              rimarem
                              sinus \&
                         \stanza \ledsidenote{\color{gray}{\textbf{Non.355,27\vspace{-0,5cm}}}}
                     
                              semper
                              sat
                              agere
                              ut
                              ne
                              in
                              amore
                              animum
                              occupes \&
                         \stanza \ledsidenote{\color{gray}{\textbf{Non.109,28\vspace{-0,5cm}}}}
                     
                              quas
                              famulitas
                              ,
                              vis
                              ,
                              egestas
                              ,
                              fama
                              ,
                              formido
                              ,
                              pavor \&
                         \stanza \ledsidenote{\color{gray}{\textbf{Schol. Leid, I,18\vspace{-0,5cm}}}}Tegea
                              Arcadiæ
                              civitas
                              †
                              calumina
                              † & 
                     antiquum
                              oppidum \&
                         \stanza \ledsidenote{\color{gray}{\textbf{Fest 334\vspace{-0,5cm}}}}
                     
                              triplicem
                              virili
                              sexu
                              partum
                              procreat \&
                         \stanza \ledsidenote{\color{gray}{\textbf{Non.450,30\vspace{-0,5cm}}}}mortem
                              ostentant
                              ,
                              regno
                              expellunt
                              , & 
                     consanguineam
                              esse
                              abdicant \&
                         \stanza \ledsidenote{\color{gray}{\textbf{Non.481,31\vspace{-0,5cm}}}}
                     †
                              regi
                              ut
                              memorabis
                              †
                              nunc
                              regnum
                              potitur
                              transmissu
                              patris \&
                         \stanza \ledsidenote{\color{gray}{\textbf{Non.258,1\vspace{-0,5cm}}}}
                              omnes
                              ,
                              qui
                              tamquam
                              nos
                              serviunt & 
                     sub
                              regno
                              ,
                              callent
                              domiti
                              imperium
                              metuere \&
                     
                  \endnumbering
		\end{Leftside}
                  \begin{Rightside}
			\beginnumbering
			\numberstanzafalse
                     
                         \stanza 
                      à présent à la fleur de l’âge, un duvet couvre ses joues \&
                         \stanza 
                      afin de retrouver sa parenté incertaine \&
                         \stanza 
                      il/elle est affligé.e et regrette, plus que moi, la tentative vaine
                              vers ce chemin \&
                         \stanza Bien que je redoute ce qui concerne la  & 
                     couverture de lit brodé \&
                         \stanza avance-toi et contiens ton outrecuidance & 
                     inflexible \&
                         \stanza . . . . dans le dernier virage, il/elle dépasse Parthénopée en
                              personne & 
                      \&
                         \stanza 
                      cité, hébétée par le dévouement, par une âme subordonnée, soulève-toi
                              maintenant ! \&
                         \stanza 
                      Quel chagrin inaccoutumé assaille ton esprit ? \&
                         \stanza je doute de la route que je dois suivre, ou  & 
                     du principe que je dois saisir pour rechercher ma descendance  \&
                         \stanza 
                      moi je sais comment je peux distinguer celui-ci parmi vous,
                              jumeaux \&
                         \stanza  celui parmi vous qui est celui à qui le signe a été donné;  & 
                     dites-moi \&
                         \stanza 
                     je montre l’anneau accroché sur mon bras gauche \&
                         \stanza 
                     Qu’est-ce que cela ? Quelle tristesse se fixe sur ton visage ? \&
                         \stanza 
                      Ô mon fils, comme je suis inquiète ! La retenue empêche de parler à
                              voix haute  \&
                         \stanza 
                      de nouveau, donne également le parti d’être en désaccord, et même
                              d’être en conflit \&
                         \stanza  car, je vais expliquer maintenant, ce que j’ai tenté de dire quand il
                              m’a interrompu & 
                      \&
                         \stanza lorsque, enceinte, je me sens m’alourdir par & 
                     la proximité de l’accouchement ...  \&
                         \stanza 
                     Comme, en recherchant, j’ai scruté les recoins en friche \&
                         \stanza 
                     toujours suffisamment occupé, afin que tu ne monopolises pas ton âme
                              avec l’amour \&
                         \stanza 
                      celles-ci que la servitude, la force, l’indigence, la réputation, la
                              crainte, l’effroi \&
                         \stanza  Tégée, ville d’Acardie, † calumina † une & 
                     ancienne ville fortifiée \&
                         \stanza 
                     elle engendre un triple enfantement de sexe masculin \&
                         \stanza ils exposent la mort, ils bannissent du & 
                     royaume, ils nient être parent \&
                         \stanza 
                      † comme vous le mentionnerez au roi †, le royaume est maintenant
                              conquis par l’héritage paternel \&
                         \stanza tous ceux qui comme nous servent sous le  & 
                     royaume savent parfaitement redouter le maître \&
                     
                  \endnumbering
		\end{Rightside}
               \end{pairs}
	\Columns
            
            
               \subsection*{Chryses}
               \begin{abstract}
                  Sources hellènes : Homère l'\textit{Ἰλιάς}; Sophocle
                        \textit{Χρύσης}, et certainement une version homonyme
                     écrite par Silenos, Euripide \textit{Ἰφιγένεια ἐν
                     Ταύροις};\par
                   Argument: la tragédie rapporte les péripéties d’Oreste, Pylade, et Iphigénie
                     —toutefois, son nom n’apparaît pas dans les fragments— qui, fuyant la Tauride
                     où ils ont volé la statue d’Artémis, cherchent refuge auprès de Chrysès,
                     prêtre-roi de Sminthe. Dans un premier temps, après avoir consulté les
                     prophètes et les anciens, Chrysès souhaite les livrer à Thoas, roi de Tauride;
                     toutefois, il change d’avis quand sa mère, Chryséis, lui révèle sa filiation
                     commune avec les fugitifs : il serait l’enfant d’Agamemnon tout comme Oreste et
                     Iphigénie.\par
               \end{abstract}
               \begin{pairs}
                  \begin{Leftside}
			\beginnumbering
			\setcounter{stanzaL}{0}
                     
                         \stanza \ledsidenote{\color{gray}{\textbf{Non.484,11\vspace{-0,5cm}}}}
                     æsti
                              forte
                              ex
                              arido \&
                         \stanza \ledsidenote{\color{gray}{\textbf{Non.488,14\vspace{-0,5cm}}}}.
                              .
                              .
                              interea
                              loci & 
                     
                              flucti
                              flaccescunt
                              ,
                              silescunt
                              venti
                              mollitur
                              mare \&
                         \stanza \ledsidenote{\color{gray}{\textbf{Gel., IV, 17\vspace{-0,5cm}}}}
                     idæ
                              promunturium
                              quoius
                              lingua
                              in
                              altum
                              proicit \&
                         \stanza \ledsidenote{\color{gray}{\textbf{Non.467,15\vspace{-0,5cm}}}}.
                              .
                              .
                              incipio
                              saxum
                              temptans
                              scandere & 
                     
                              vorticem
                              ,
                              summum
                              ,
                              inde
                              in
                              omnis
                              partes
                              prospectum
                              aucupo \&
                         \stanza \ledsidenote{\color{gray}{\textbf{Non.508,28\vspace{-0,5cm}}}}
                     
                              si
                              qua
                              potestur
                              investigari
                              via \&
                         \stanza \ledsidenote{\color{gray}{\textbf{Fest. 343\vspace{-0,5cm}}}}
                     
                              est
                              ibi
                              sub
                              eo
                              saxo
                              penitus
                              strata
                              harena
                              ingens
                              specus \&
                         \stanza \ledsidenote{\color{gray}{\textbf{Non.415, 34\vspace{-0,5cm}}}}
                     
                              .
                              .
                              .
                              fugimus
                              qui
                              arte
                              hac
                              vescimur \&
                         \stanza \ledsidenote{\color{gray}{\textbf{Non.74,1\vspace{-0,5cm}}}}
                     
                              adiutamini
                              et
                              defendite \&
                         \stanza \ledsidenote{\color{gray}{\textbf{Non.89,21\vspace{-0,5cm}}}}
                     atque
                              eccos
                              unde
                              certiscent
                              .
                              .
                              . \&
                         \stanza \ledsidenote{\color{gray}{\textbf{Non.475,1\vspace{-0,5cm}}}}
                     inveni
                              ,
                              opino
                              ,
                              Orestes
                              uter
                              esset
                              tamen \&
                         \stanza \ledsidenote{\color{gray}{\textbf{Fest 373\vspace{-0,5cm}}}}
                     di
                              monerint
                              meliora
                              atque
                              amentiam
                              averruncassint
                              tuam
                              ! \&
                         \stanza \ledsidenote{\color{gray}{\textbf{Fest. 273\vspace{-0,5cm}}}}pro
                              merenda
                              gratia & simul
                              cum
                              videam
                              Graios
                              nihil
                              mediocriter & 
                     
                              redamptruare
                              opibusque
                              {opibusque}
                              summis
                              persequi \&
                         \stanza \ledsidenote{\color{gray}{\textbf{Non.39,31\vspace{-0,5cm}}}}
                     atque
                              ,
                              ut
                              promeruit
                              ,
                              pater
                              mihi
                              patriam
                              populavit
                              meam \&
                         \stanza \ledsidenote{\color{gray}{\textbf{Non.101,27\vspace{-0,5cm}}}}
                     perque
                              {perque}
                              nostram
                              egregiam
                              unanimitatem
                              quam
                              memoria
                              deiugat \&
                         \stanza \ledsidenote{\color{gray}{\textbf{Non.127,4\vspace{-0,5cm}}}}
                     
                              set
                              cesso
                              inimicitiam
                              integrare
                              .
                              .
                              . \&
                         \stanza \ledsidenote{\color{gray}{\textbf{Cicéron, De Orat., 46,155\vspace{-0,5cm}}}}cives
                              ,
                              antiqui
                              amici
                              maiorum
                              meum
                              , & consilium
                              socii
                              ,
                              augurium
                              atque
                              extum
                              interpretes
                              , & 
                              {
                              .
                              .
                              .
                              } & 
                     postquam
                              prodigium
                              horriferum
                              ,
                              portentum
                              pavos \&
                         \stanza \ledsidenote{\color{gray}{\textbf{Cicéron, De de div., I, 57,131\vspace{-0,5cm}}}}.
                              .
                              .
                              nam
                              isti
                              qui
                              linguam
                              avium
                              intellegunt & plusque
                              {plusque}
                              ex
                              alieno
                              iecore
                              sapiunt
                              quam
                              ex
                              suo
                              , & 
                     magis
                              audiendum
                              quam
                              auscultandum
                              censeo \&
                         \stanza \ledsidenote{\color{gray}{\textbf{Varron, De ling lat., VI, 83\vspace{-0,5cm}}}}Hoc
                              vide
                              ,
                              circum
                              supraque
                              {supraque}
                              quod
                              complexu
                              continet
                              terram
                              ; & 
                     Id
                              quod
                              nostri
                              caelum
                              memorant
                              ,
                              Grai
                              perhibent
                              æthera \&
                         \stanza \ledsidenote{\color{gray}{\textbf{Non 144,9\vspace{-0,5cm}}}}
                     Solisque
                              {Solisque}
                              exortu
                              capessit
                              candorem
                              ,
                              occasu
                              nigret \&
                         \stanza \ledsidenote{\color{gray}{\textbf{Cicéron, De de div., I, 68,131\vspace{-0,5cm}}}}quidquid
                              est
                              hoc
                              ,
                              omnia
                              animat
                              ,
                              format
                              ,
                              alit
                              ,
                              auget
                              ,
                              creat & sepelit
                              ,
                              recipitque
                              {recipitque}
                              in
                              sese
                              omnia
                              ,
                              omniumque
                              {omniumque}
                              idemst
                              {idemst}
                              pater
                              ; & 
                     indidemque
                              {indidemque}
                              eadem
                              æque
                              oriuntur
                              de
                              integro
                              atque
                              eodem
                              occidunt \&
                         \stanza \ledsidenote{\color{gray}{\textbf{Non 75,11\vspace{-0,5cm}}}}
                     Mater
                              est
                              terra
                              :
                              ea
                              parit
                              corpus
                              animam
                              æther
                              adiugat \&
                         \stanza \ledsidenote{\color{gray}{\textbf{Non 469,8\vspace{-0,5cm}}}}
                     
                              propemodum
                              animus
                              conjectura
                              de
                              errore
                              eius
                              augurat \&
                         \stanza \ledsidenote{\color{gray}{\textbf{Prisc., VI 710P\vspace{-0,5cm}}}}
                     
                              ossuum
                              inhumatum
                              æstuosam
                              auram
                              .
                              .
                              . \&
                     
                  \endnumbering
		\end{Leftside}
                  \begin{Rightside}
			\beginnumbering
			\numberstanzafalse
                     
                         \stanza 
                     par le hasard de la marée vers la terre aride  \&
                         \stanza  . . . pendant ce temps, en ce lieu, les vagues  & 
                      se dissipent, les vents se taisent, la mer s’apaise \&
                         \stanza 
                     le promontoire d’Ida, dont la langue se jette au large \&
                         \stanza  . . .je commence en essayant d’escalader le  & 
                     rocher vers le plus haut sommet, de là, je suis à l’affût de tous les
                              détails \&
                         \stanza 
                     S’il peut être recherché par quel chemin \&
                         \stanza 
                     dans ce lieu, au pied de ce rocher, il y a une gigantesque grotte
                              recouverte de sable \&
                         \stanza 
                     Nous avons fui, nous qui nous sommes servis de cette ruse \&
                         \stanza 
                     Secourez\{-nous\} et défendez\{-nous\} \&
                         \stanza 
                     Et voici d’où ils seront renseignés \&
                         \stanza 
                     je présume que j’ai, malgré tout, découvert lequel des deux Oreste il
                              est \&
                         \stanza 
                     que les dieux te préservent davantage, et qu’ils te détournent de ta
                              démence! \&
                         \stanza en même temps que je constate que les Grecs & ne font en rien la contrepartie avec modération, je constate qu’ ils
                              l’accomplissent & 
                      avec les plus grands moyens \&
                         \stanza 
                      et, comme j’ai mérité que mon père ait dépeuplé ma patrie \&
                         \stanza 
                     Et, par suite de notre excellent rapport que la mémoire divise  \&
                         \stanza 
                      mais je tarde à renouveler l’hostilité... \&
                         \stanza Citoyens, amis autrefois de mes ancêtres, & conseil augural allié, et interprètes d’haruspice, & \{...\} & 
                     après que le prodige effrayant, la calamité prédite \&
                         \stanza En vérité, à l’égard de celui qui comprend la langue des oiseaux, et
                              qui a davantage &  le jugement du foie d’autrui que du sien, j’estime qu’il vaut mieux
                              l'entendre que & 
                     l'écouter. \&
                         \stanza Regarde autour de toi et au-dessus ce qui contient la terre par une
                              étreinte; & 
                     ce que les nôtres nomment la voûte céleste, les Grecs l’appellent
                              l’éther  \&
                         \stanza 
                     Et au lever du soleil il prend une blancheur éclatante, au coucher il
                              devient obscur \&

                         \stanza quoi que ce soit, tout ce qu’il anime, forme, nourrit, accroît, crée,
                              enterre, et il accueille  & en lui-même toutes ces choses, et également, il est le père de toutes
                              ces choses; et, delà, & 
                     ces mêmes choses prennent également leurs sources de nouveau et elles
                              périssent de la même manière. \&
                         \stanza 
                      La terre est la mère : celle-ci fait naître le corps, l’éther ajoute
                              l’âme \&
                         \stanza 
                     Je pressens presque mon âme va inférer de son erreur \&
                         \stanza 
                     os sans sépulture, à l’émanation brûlante ... \&
                     
                  \endnumbering
		\end{Rightside}
               \end{pairs}
	\Columns
            
            
               \subsection*{Duloreste}
               \begin{abstract}
                   Sources hellènes : Eschyle \textit{Χοηφόρες}; Sophocle
                        \textit{ Ἠλέκτρα}, Euripide \textit{Ἰφιγένεια ἐν Ταύροις}.Toutefois, cette version du mythe semble très
                     différente de ces prédécesseurs hellènes.\par
                  Argument: les fragments, provenant essentiellement de Nonius, ne donne aucune
                     indiction de contexte, il est donc difficile définir précisément l’intrigue. La
                     trame de l'\textit{Oreste esclave}} semble être la vengeance d’Oreste et
                     le meurtre de Clytemenestre; un fragment révèle qu’Oreste utilise une ruse,
                     fragment XIII, pour mener à bien son dessein. Enfin, un fragment (XVIII) fait
                     mention du personnage Œax, peu présent dans les sources antiques.\par
               \end{abstract}
               \begin{pairs}
                  \begin{Leftside}
			\beginnumbering
			\setcounter{stanzaL}{0}
                     
                         \stanza \ledsidenote{\color{gray}{\textbf{Non.490,16\vspace{-0,5cm}}}}
                     delphos
                              venum
                              pecus
                              egi
                              ,
                              unde
                              ad
                              stabula
                              hæc
                              itiner
                              contuli \&
                         \stanza \ledsidenote{\color{gray}{\textbf{Non.522,7\vspace{-0,5cm}}}}
                     gnatam
                              despondit
                              ,
                              nuptiis
                              hanc
                              dat
                              diem \&
                         \stanza \ledsidenote{\color{gray}{\textbf{Non.505,1\vspace{-0,5cm}}}}hymenæum
                              fremunt & 
                     æquales
                              ,
                              aula
                              resonit
                              crepitu
                              musico \&
                         \stanza \ledsidenote{\color{gray}{\textbf{Non.497,16\vspace{-0,5cm}}}}
                     Nonne
                              officium
                              fungar
                              vulgi
                              ,
                              atque
                              ægre
                              malefactum
                              feram
                              ? \&
                         \stanza \ledsidenote{\color{gray}{\textbf{Prisc., VI, 668\vspace{-0,5cm}}}}
                     pater
                              Achivos
                              in
                              Capherei
                              saxis
                              pleros
                              perdidit \&
                         \stanza \ledsidenote{\color{gray}{\textbf{Non. 184,3\vspace{-0,5cm}}}}.
                              .
                              .
                              primum
                              hoc
                              abs
                              te
                              oro
                              ,
                              ni
                              inexorabilem & 
                     Faxis
                              ,
                              ni
                              turpassis
                              vanitudine
                              ætatem
                              tuam \&
                         \stanza \ledsidenote{\color{gray}{\textbf{Non. 160,11\vspace{-0,5cm}}}}
                     oro
                              ,
                              †
                              mi
                              †
                              ne
                              flectas
                              fandi
                              me
                              prolixitudine \&
                         \stanza \ledsidenote{\color{gray}{\textbf{Non. 125, 4\vspace{-0,5cm}}}}
                     Si
                              quis
                              hac
                              me
                              oratione
                              incilet
                              ,
                              quid
                              respondeam
                              ? \&
                         \stanza \ledsidenote{\color{gray}{\textbf{Non. 179,14\vspace{-0,5cm}}}}nam
                              te
                              in
                              tenebrica
                              sæpe
                              lacerabo
                              fame & 
                     clausum
                              et
                              fatigans
                              artus
                              torto
                              distraham \&
                         \stanza \ledsidenote{\color{gray}{\textbf{Non. 237,11\vspace{-0,5cm}}}}
                     aut
                              hic
                              est
                              ,
                              aut
                              hic
                              adfore
                              actutum
                              autumno \&
                         \stanza \ledsidenote{\color{gray}{\textbf{\vspace{-0,5cm}}}}quid
                              quod
                              iam
                              ei
                              mihi & 
                              piget
                              paternum
                              nomen
                              ,
                              maternum
                              pudet & 
                     profari \&
                         \stanza \ledsidenote{\color{gray}{\textbf{Non. 13,28\vspace{-0,5cm}}}}
                     
                              non
                              decet
                              animum
                              ægritudine
                              in
                              re
                              crepera
                              confici \&
                         \stanza \ledsidenote{\color{gray}{\textbf{Non.137,7\vspace{-0,5cm}}}}
                     
                              utinam
                              nunc
                              matrescam
                              ingenio
                              ,
                              ut
                              meum
                              patrem
                              ulcisci
                              queam
                              ! \&
                         \stanza \ledsidenote{\color{gray}{\textbf{Non. 90,12\vspace{-0,5cm}}}}.
                              .
                              .
                              extemplo
                              ,
                              Ægisthi
                              fidem & 
                     
                              nuncupantes
                              conciebunt
                              populum
                              .
                              .
                              . \&
                         \stanza \ledsidenote{\color{gray}{\textbf{Non. 181, 21\vspace{-0,5cm}}}}
                     
                              heu
                              ,
                              non
                              tyrannum
                              novi
                              temeritudinem
                              ? \&
                         \stanza \ledsidenote{\color{gray}{\textbf{Non. 491,24\vspace{-0,5cm}}}}
                     .
                              .
                              .
                              quidnam
                              autem
                              hoc
                              soniti
                              est
                              ,
                              quod
                              stridunt
                              fores
                              ? \&
                         \stanza \ledsidenote{\color{gray}{\textbf{Fest. 330\vspace{-0,5cm}}}}
                     amplus
                              ,
                              rubicundo
                              colore
                              et
                              spectu
                              protervo
                              ferox \&
                         \stanza \ledsidenote{\color{gray}{\textbf{Non. 352,1\vspace{-0,5cm}}}}
                     
                              Hicine
                              {Hicine}
                              is
                              est
                              ,
                              quem
                              fama
                              Graia
                              ante
                              omnis
                              nobilitat
                              viros
                              ? \&
                         \stanza \ledsidenote{\color{gray}{\textbf{Non. 355,17\vspace{-0,5cm}}}}
                     Is
                              quis
                              est
                              ?
                              —
                              Qui
                              te
                              ,
                              nisi
                              illum
                              tu
                              occupas
                              ,
                              leto
                              dabit \&
                         \stanza \ledsidenote{\color{gray}{\textbf{Non. 401\vspace{-0,5cm}}}}Unde
                              exoritur
                              ?
                              quo
                              præsidio
                              fretus
                              ?
                              auxiliis
                              quibus
                              ? & 
                     
                              Quo
                              consilio
                              consternatur
                              ?
                              Qua
                              vi
                              ?
                              Cuius
                              copiis
                              ? \&
                         \stanza \ledsidenote{\color{gray}{\textbf{Non. 477,26\vspace{-0,5cm}}}}
                     
                              illum
                              quæro
                              ,
                              qui
                              adjutatur \&
                         \stanza \ledsidenote{\color{gray}{\textbf{Non.510,22\vspace{-0,5cm}}}}nunc
                              ne
                              illum
                              expectes
                              ,
                              quando
                              amico
                              amiciter & 
                     fecisti \&
                         \stanza \ledsidenote{\color{gray}{\textbf{Non. 38,31\vspace{-0,5cm}}}}
                     
                              .
                              .
                              .
                              ubi
                              illic
                              est
                              ?
                              me
                              miseram
                              !
                              quonam
                              clam
                              {
                              grados
                              }
                              eliminat? \&
                         \stanza \ledsidenote{\color{gray}{\textbf{Non.7,3\vspace{-0,5cm}}}}.
                              .
                              .
                              {
                              ni
                              }
                              me
                              calvitur
                              suspicio
                              , & 
                     hoc
                              est
                              illud
                              quod
                              fore
                              occulte
                              Œax
                              prædixit
                              .
                              .
                              . \&
                         \stanza \ledsidenote{\color{gray}{\textbf{Non.307,10\vspace{-0,5cm}}}}.
                              .
                              .
                              ut
                              si
                              ita
                              sunt
                              promerita
                              vestra
                              ,
                              æquiperare
                              ut
                              queam & 
                     vereor
                              ,
                              nisi
                              numquam
                              fatiscar
                              facere
                              quod
                              quibo
                              boni \&
                         \stanza \ledsidenote{\color{gray}{\textbf{Non.342,1\vspace{-0,5cm}}}}
                     Macte
                              esto
                              virtute
                              ,
                              operaque
                              {operaque}
                              !
                              omenque
                              {omenque}
                              adprobo \&
                         \stanza \ledsidenote{\color{gray}{\textbf{Non.111,6\vspace{-0,5cm}}}}responsa
                              explanat
                              :
                              mandat
                              ne
                              matri
                              fuat & 
                     cognoscendi
                              umquam
                              aut
                              contuendi
                              copia \&
                         \stanza \ledsidenote{\color{gray}{\textbf{Non.123,30\vspace{-0,5cm}}}}
                     set
                              med
                              incertat
                              dictio
                              —
                              quare
                              ?
                              expedi
                              ! \&
                         \stanza \ledsidenote{\color{gray}{\textbf{Non.260,10\vspace{-0,5cm}}}}nihil
                              conjectura
                              quivi
                              interpretarier
                              , & 
                     quorsum
                              †
                              flexivice
                              contenderet \&
                         \stanza \ledsidenote{\color{gray}{\textbf{Non. 146,19\vspace{-0,5cm}}}}
                     
                              vel
                              cum
                              illum
                              videas
                              sollicitum
                              orbitudine \&
                         \stanza \ledsidenote{\color{gray}{\textbf{Non. 115,11\vspace{-0,5cm}}}}
                     nec
                              grandiri
                              frugum
                              fetum
                              posse
                              nec
                              mitiscere \&
                         \stanza \ledsidenote{\color{gray}{\textbf{Non.23,13\vspace{-0,5cm}}}}
                     animum
                              quæ \&
                     
                  \endnumbering
		\end{Leftside}
                  \begin{Rightside}
			\beginnumbering
			\numberstanzafalse
                     
                         \stanza 
                     je suis allé vendre mon bétail à Delphes, de là je me suis réfugié
                              dans ces demeures \&
                         \stanza 
                     il/elle a fiancé sa fille, donne ce jour pour le mariage  \&
                         \stanza les compagnons font retentir l’hyménée, la & 
                     cour résonne par le crépitement musical \&
                         \stanza 
                      est-ce que je dois endurer, tel que l’obligeance du peuple, et
                              supporter péniblement le méfait \&
                         \stanza 
                      le fondateur a anéanti la majeure partie des Grecs sur les rochers de
                              Capharée \&
                         \stanza . . . ce que je te dis en avant toute chose : ne sois pas sans pitié
                              pour moi, ne souille pas & 
                      ton âge par vanité  \&
                         \stanza 
                     j’implore à moi-même que tu ne sois pas fléchi par l’étendue de mon
                              expression \&
                         \stanza 
                     si quelqu’un me réprimande avec ce discours, que dois-je répondre ?
                            \&
                         \stanza  car, je vais te faire souffrir de faim dans un  & 
                     sombre cachot, enfermé et accablé, je vais écarteler tes membres avec
                              une corde de torture \&
                         \stanza 
                      soit il est ici, soit, je suppose, qu’il y sera sous peu \&
                         \stanza hélas, pourquoi est-ce que maintenant, il  & est regrettable que je prononce le nom de mon père, il est honteux que
                              je prononce  & 
                     celui de ma mère \&
                         \stanza 
                      il ne convient pas que l’âme soit accablée par le chagrin dans une
                              situation incertaine \&
                         \stanza 
                      si seulement je pouvais devenir maintenant aussi ingénieux que ma
                              mère, pour que je puisse venger mon père ! \&
                         \stanza  ... aussitôt, jurant fidélité à Égisthe, ils se & 
                     réuniront en un peuple  \&
                         \stanza 
                     hélas, je ne connais pas l’irréflexion du tyran \&
                         \stanza 
                     ... mais, quel est donc ce bruit qui fait grincer les portes ? \&
                         \stanza 
                      imposant par sa couleur rubiconde, et fougueux par son regard
                              impudent \&
                         \stanza 
                      est-ce celui que la réputation grecque rend fameux devant chaque
                              homme? \&
                         \stanza 
                      Qui est-il ? — Celui qui, si tu ne prends pas les devants sur lui, te
                              donnera la mort \&
                         \stanza D’où sort-il ? En quelle protection avoir confiance ? Quels secours ?
                              Vers quel plan doit-on se & 
                      dresser ? Quelle force ? Vers quelles troupes ?  \&
                         \stanza 
                      je cherche celui qu’il aide \&
                         \stanza  maintenant, ne l’attends pas, puisqu’amicalement tu l’as fait ton
                              ami & 
                      \&
                         \stanza 
                     où est cet homme ? Comme je suis misérable ! Où donc s’est-il échappé
                              à pas de loup ? \&
                         \stanza  si ma suspicion ne me trompe pas, c’est ce & 
                      qu’Œax a prédit qui sera dissimulé \&
                         \stanza  . . . si bien que je crains, s’il en est ainsi de vos bienfaits, que
                              je ne puisse pas les égaler, & 
                     à supposer toujours je m’évertuerai à faire ce qui peut être bon \&
                         \stanza 
                     Aie du courage et de l’application! — Et j’approuve le présage \&
                         \stanza  Il expose ses décisions : il mandate qu’il & 
                     n’y ait pas un jour la possibilité que la mère le reconnaisse ou le
                              voit \&
                         \stanza 
                     mais ce discours me met dans l’incertitude. — Pourquoi ? Explique-toi
                              ! \&
                         \stanza  rien que je ne peux interpréter par la conject- & 
                     ure, vers quoi † je pourrais soutenir d’une manière équivoque \&
                         \stanza 
                      ou, comme tu le vois angoissé par la perte de ses enfants \&
                         \stanza 
                      et le semis des céréales ne peut ni se développer ni mûrir \&
                         \stanza 
                      l’âme qui \&
                     
                  \endnumbering
		\end{Rightside}
               \end{pairs}
	\Columns
            
            
               \subsection*{Hermione}
               \begin{abstract}
                   Sources hellènes : Sophocle \textit{Ἑρμιόνη}, Euripide
                        \textit{Ἀνδρομάχη}, Antiphon \textit{Ἀνδρομάχη}\par
                  Argument: fille d’Hélène et Ménélas, Hermione était promise à Oreste.
                     Seulement, après avoir fait la guerre aux Troyens, Ménélas décide d’accorder la
                     main de sa fille à Néoptolème. Oreste défit alors son rival, fort probablement
                     dans un combat singulier, afin d’acquérir ce qui lui avait été promis. \par
               \end{abstract}
               \begin{pairs}
                  \begin{Leftside}
			\beginnumbering
			\setcounter{stanzaL}{0}
                     
                         \stanza \ledsidenote{\color{gray}{\textbf{Non.77,33\vspace{-0,5cm}}}}
                     {
      et
      }
      obnoxium
      esse
      aut
      brutum
      aut
      elinguem
      putes \&
                         \stanza \ledsidenote{\color{gray}{\textbf{Varron, De ling.lat.\vspace{-0,5cm}}}}.
      .
      .
      regni
      alieni
      cupiditas & 
                     pellexit \&
                         \stanza \ledsidenote{\color{gray}{\textbf{Non.316,18\vspace{-0,5cm}}}}
                      par
      fortitudo
      ,
      gemina
      confidentia \&
                         \stanza \ledsidenote{\color{gray}{\textbf{Non.96,1\vspace{-0,5cm}}}}
                     Nam
      solus
      Danais
      hic
      domutionem
      dedit \&
                         \stanza \ledsidenote{\color{gray}{\textbf{Non.237,3\vspace{-0,5cm}}}}
                     Quid
      benefacta
      mei
      patris
      ,
      cuius
      opera
      te
      esse
      ultum
      autumant
      ? \&
                         \stanza \ledsidenote{\color{gray}{\textbf{Non.116,15\vspace{-0,5cm}}}}.
      .
      .
      quo
      tamen
      ipsa
      orbitas & grandævitasque
      {grandævitasque}
      Pelei
      penuriam & 
                      stirpis
      †
      subaxet
      †
      .
      .
      . \&
                         \stanza \ledsidenote{\color{gray}{\textbf{Non.20,18\vspace{-0,5cm}}}}
                     sermonem
      hic
      nostrum
      ex
      occulto
      clepsit
      ,
      quantum
      intellego \&
                         \stanza \ledsidenote{\color{gray}{\textbf{Non.234,25\vspace{-0,5cm}}}}
                      quod
      ego
      in
      acie
      crebro
      objectans
      vitam
      bellando
      aptus
      sum \&
                         \stanza \ledsidenote{\color{gray}{\textbf{Non.280,27\vspace{-0,5cm}}}}
                      prius
      data
      est
      quam
      tibi
      dari
      dicta
      ,
      aut
      quam
      reditum
      est
      Pergamo \&
                         \stanza \ledsidenote{\color{gray}{\textbf{Non.393,2\vspace{-0,5cm}}}}habet
      hoc
      senectus
      ipsa
      in
      se
      {
      se
      }
      ,
      cum
      pigra
      est
      , & 
                     spisse
      ut
      videantur
      omnia
      {
      ei
      }
      confieri \&
                         \stanza \ledsidenote{\color{gray}{\textbf{Non.496,30\vspace{-0,5cm}}}}.
      .
      .
      Tyndareo
      fieri
      contumeliam
      , & 
                     cuius
      a
      te
      veretur
      maxume \&
                         \stanza \ledsidenote{\color{gray}{\textbf{Non.73,13\vspace{-0,5cm}}}}
                     non
      tu
      te
      e
      conspectu
      hinc
      amolire
      .
      .
      . \&
                         \stanza \ledsidenote{\color{gray}{\textbf{Non.72,33\vspace{-0,5cm}}}}
                     tristia
      atque
      animi
      intoleranda
      anxitudine \&
                         \stanza \ledsidenote{\color{gray}{\textbf{Non.470,19\vspace{-0,5cm}}}}
                     cum
      neque
      me
      aspicere
      æquales
      dignarent
      meæ \&
                         \stanza \ledsidenote{\color{gray}{\textbf{Non.178,14\vspace{-0,5cm}}}}
                     .
      .
      .
      sub
      judicio
      quæ
      omnes
      tetinerim
      grados \&
                         \stanza \ledsidenote{\color{gray}{\textbf{Non.97,1\vspace{-0,5cm}}}}
                      quantamque
      {quantamque}
      ex
      discorditate
      cladem
      inportem
      familiæ
      ? \&
                         \stanza \ledsidenote{\color{gray}{\textbf{Non.132,29\vspace{-0,5cm}}}}
                      lamentas
      fletus
      facere
      conpendi
      licet \&
                         \stanza \ledsidenote{\color{gray}{\textbf{Diomedes, Ip. 395P\vspace{-0,5cm}}}}
                      paucis
      absolvit
      ,
      ne
      moraret
      diutius \&
                         \stanza \ledsidenote{\color{gray}{\textbf{Non.87,26\vspace{-0,5cm}}}}
                     currum
      liquit
      ,
      clamide
      contorta
      astu
      clupeat
      bracchium \&
                         \stanza \ledsidenote{\color{gray}{\textbf{Servius, Æen., V, 40\vspace{-0,5cm}}}}
                      Ibo
      atque
      edicam
      ,
      frequentes
      ut
      eant
      gratatum
      hospiti \&
                         \stanza \ledsidenote{\color{gray}{\textbf{Non.88,20\vspace{-0,5cm}}}}concorditatem
      hospitio
      adjunctam
      perpetem & 
                     probitate
      conservetis \&
                         \stanza \ledsidenote{\color{gray}{\textbf{Non.113,32\vspace{-0,5cm}}}}
                       hoc
      flexanima
      atque
      omnium
      regina
      rerum
      oratio
      ! \&
                         \stanza \ledsidenote{\color{gray}{\textbf{Non.356\vspace{-0,5cm}}}}
                     .
      .
      .
      aut
      non
      cernam
      ,
      nisi
      tagam \&
                         \stanza \ledsidenote{\color{gray}{\textbf{Fest. 277M.\vspace{-0,5cm}}}}
                      quas
      gloria
      et
      {
      .
      .
      .
      refutant
      .
      .
      .
      va
      }
      rietas
      humanum \&
                     
                  \endnumbering
		\end{Leftside}
                  \begin{Rightside}
			\beginnumbering
			\numberstanzafalse
                     
                         \stanza 
                      \&
                        \stanza 
                      et tu peux croire qu’il/je est/suis muet ou stupide, ou nébuleux  \&
                        \stanza la convoitise du royaume d’autrui a séduit   & 
                      \&
                        \stanza 
                     bravoure semblable, audace jumelle  \&
                        \stanza 
                     car il est le seul qui a permis le retour aux Danaïdes  \&
                        \stanza 
                     pourquoi disent-ils les bienfaits de mon père, par le soin duquel tu as été vengé \&
                        \stanza toutefois, parce que la perte même de ses & enfants et le grand âge de Pélée le †contrai- & 
                     gnent† 
                        à l’absence de descendance \&
                        \stanza 
                     celui-ci a dérobé notre conversation depuis une cachette, pour autant que je le comprends  \&
                        \stanza 
                     ce que moi, dans une bataille répétée, exposant ma vie en combattant, j’ai gagné  \&
                        \stanza 
                     Elle a été donnée, avant qu’elle se soit promis de se donner à toi, ou avant le retour de Pergame \&
                        \stanza la vieillesse même a ça en elle, quand elle & 
                     est 
                        indolente afin que tous voient lentement ce qui s’est produit \&
                        \stanza . . . qu’il se produise l’affront à Tyndare, & 
                     celui
                        dont vous craignez le plus  \&
                        \stanza 
                     Toi, ne t’éloigne pas d’ici, de ma vue  \& 
                        \stanza 
                     les chagrins et les sentiments insupportables de l’anxiété \&
                        \stanza 
                     comme mes congénères ne me jugeaient pas digne d’être aperçu  \&
                        \stanza 
                     . . . moi qui dois tenir tous les grades sous  le jugement  \&
                        \stanza 
                     et combien de discorde suscité-je à ma famille par le fléau ? \&
                        \stanza 
                     il est permis de faire l’économie des lamentations et des pleurs \&
                        \stanza 
                     il s’est acquitté de peu de choses de peur qu’il ne s’attarde plus longtemps  \&
                        \stanza 
                     il laisse son char, avec astuce il se fait un bouclier au bras grâce à sa clamyde enroulée \&
                        \stanza 
                     j’irai et j’ordonnerai au plus grand nombre qu’ils aillent louer le voyageur  \&
                        \stanza puissiez-vous conserver la concorde perpétuelle
                             dans l’hospitalité, liée à l’intégrité  & 
                      \&
                        \stanza 
                     cette éloquence, fougueuse, et reine de tous les biens !  \&
                        \stanza 
                     . . . ou je ne discerne rien excepté si je le touche  \&
                        \stanza 
                     que la gloire et la diversité réfutent l’humain \&
                        
                     
                  \endnumbering
		\end{Rightside}
               \end{pairs}
	\Columns
            

         

         
            Lycurgus
            
               \textit{}
                  \textit{}.\par
               \par
            
            \begin{pairs}
               \begin{Leftside}
			\beginnumbering
			\setcounter{stanzaL}{0}
                  
                      \stanza \ledsidenote{\color{gray}{\textbf{Non.???\vspace{-0,5cm}}}}
                      \&


                  
               \endnumbering
		\end{Leftside}
               \begin{Rightside}
			\beginnumbering
			\numberstanzafalse
                  
                      \stanza 
                      \&
                  
               \endnumbering
		\end{Rightside}
            \end{pairs}
	\Columns
         



      \end{document}