\documentclass[12pt,onecolumn,twoside,a4paper]{memoir}
\usepackage[no-math]{fontspec}
\usepackage{ebgaramond}
\usepackage{graphicx}
\usepackage{amssymb}
\usepackage{xcolor}
\usepackage{polyglossia}% typographie française
\setdefaultlanguage{french}
\usepackage{microtype}
\usepackage{xspace}% gérer les espaces
\usepackage{amssymb}% symboles utiles (math, etc.)
\usepackage{ulem} % souligner
\usepackage{alltt} % environnement "télétype"
\usepackage{setspace} % réglage des interlignes etc.
\usepackage{fancyhdr} % hauts et pieds de page
\usepackage{multicol} % texte sur plusieurs colonnes
\usepackage{fancybox} % boîtes améliorées
\usepackage{array} % tableaux
\usepackage{multirow} % les tableaux améliorés
\usepackage{url} % écrire les url
\usepackage{appendix} % annexes améliorées
 \usepackage{geometry}
\usepackage{csquotes}
\usepackage{hyperref}
\usepackage{ragged2e}
\usepackage[backend=biber, bibstyle=verbose, backref=false, hyperref=false, citestyle=authortitle-ibid]{biblatex}
\DeclareFieldFormat{title}{\textit{#1}}
\DeclareFieldFormat{booktitle}{\textit{#1}}
\DeclareFieldFormat{journaltitle}{\textit{#1}}
\addbibresource{Biblio.bib}
%%%%%%%%
%%%%%%%%
%%%%%%%%%%%%%%%%
\usepackage{calc} 
\usepackage{fourier-orns} 
%%%% voir http://jacques-andre.fr/fontex/Fourier-orn.pdf
%%%%%%%
\pagestyle{fancyplain} \renewcommand{\chaptermark}[1]{% 
	\markboth{\chaptername\ \thechapter.\ #1}% 
	{\chaptername\ \thechapter.\ #1}} \renewcommand{\sectionmark}[1]{% 
	\markright{\thesection\ #1}}
% 
\lhead[\fancyplain{}{\bfseries\thepage}]%
{\fancyplain{}{\bfseries\nouppercase{\leftmark}}} 
\rhead[\fancyplain{}{\bfseries\nouppercase{\rightmark}}]%
{\fancyplain{}{\bfseries\thepage}} \fancyfoot{}
%%%%%%%%%%%%
%%%%%%%%%%%%
\newcommand{\phipaireblanche}
{\newpage{\pagestyle{empty}\cleardoublepage}} \newcommand{\phietc}{\textit{etc.}\xspace}
\newcommand{\phigg}[1]{\og #1 \fg}
%%%%%%%
\makeatletter
\newcommand{\finirphiimpaire}{\clearpage\if@twoside \ifodd\c@page
	\hbox{}\newpage\if@twocolumn\hbox{}\newpage\fi\fi\fi} \makeatother
\newcommand{\phiimpaireblanche}{% 
	\newpage{\pagestyle{empty}\finirphiimpaire}}
\renewcommand{\baselinestretch}{1,2}%%% interligne
\makeatother
%%%%%%%
%%%%%%
%Style Chapitre établi par Vincent Zoonekynd : http://zoonek.free.fr/LaTeX/LaTeX_samples_chapter/0.html
\makeatletter
\setlength\midchapskip{7pt}
\makechapterstyle{VZ21}{
	\renewcommand\chapnamefont{\Large\scshape}
	\renewcommand\chapnumfont{\Large\scshape\centering}
	\renewcommand\chaptitlefont{\huge\bfseries\centering}
	\renewcommand\printchaptertitle[1]{%
		\setlength\tabcolsep{7pt}% used as indentation on both sides
		\settowidth\@tempdimc{\chaptitlefont ##1}%
		\setlength\@tempdimc{\textwidth-\@tempdimc-2\tabcolsep}%
		\chaptitlefont
		\ifdim\@tempdimc > 0pt\relax% one line
		\begin{tabular}{c}
			\toprule  ##1\\ \bottomrule
		\end{tabular}
		\else% two+ lines
		\begin{tabular}{%
				>{\chaptitlefont\arraybackslash}p{\textwidth-2\tabcolsep}}
			\toprule ##1\\ \bottomrule
		\end{tabular}
		\fi
	}
}
%%%%%%%%%%%%%%
%%%%% Pour la traduction %%%%%%%%
\usepackage[ widthliketwocolumns,
nocritical,
noeledsec,
noend,
nofamiliar,
noledgroup,
series={}
]{reledmac}
\usepackage{reledpar}
%%%%%%%
\setcounter{stanzaindentsrepetition}{1}
\setstanzaindents{0,0,0}
\AtEveryStopStanza{\vspace{1\baselineskip}}
\numberstanzatrue
\renewcommand{\thestanzaL}{\MakeUppercase{\roman{stanzaL}}}
%%%%%
\firstlinenum*{100000}
%%%%%%%
\setlength{\Lcolwidth}{.430\textwidth}
\setlength{\Rcolwidth}{.450\textwidth}
\columnsposition{C}
\setlength{\beforecolumnseparator}{0.035\textwidth}
\setlength{\aftercolumnseparator}{0.0001\textwidth}
\sidenotemargin{left}
%%%%% 
%%%%%%%%%%%%%%%%%
%%%%% 
\AtBeginDocument{\renewcommand{\abstractname}{}}
%%%%%%%%%%%%%%%%%
                \title{\textit{Corpus sélectionné des fragments de tragédies de l'époque républicaine}}
                \author{}
                \date{\today}
                \begin{document}
            
         
            \section{Livius Andronicus}
            Texte latin établi et organisé selon les choix de reconstruction et d'édition de \cite{TrRF_I_2012}.\par
            
               \subsection*{Égisthe}
               \begin{abstract}
                  Source hellène : Sophocle \textit{Αἴγισθος}, Eschyle
                        \textit{Όρεστεια}.\par
                   L'intrigue est proche de l'\textit{Άγαμέμνων} d'Eschyle
                     et de la version latine de Sénèque: elle se concentre donc sur l'assassinat
                     d'Agamemnon par Égisthe et Clytemnestre.\par
               \end{abstract}
               \begin{pairs}
                  \begin{Leftside}
			\beginnumbering
			\setcounter{stanzaL}{0}
                     
                         \stanza Nam
                              ut
                              Pergama &
                              accensa
                              et
                              præda
                              per
                              participes
                              æquiter &
                     partita
                              est. \&
                         \stanza Tum
                              autem
                              lasciuum
                              Nerei
                              simum
                              pecus &
                     ludens
                              ad
                              cantum
                              classem
                              lustratur... \&
                         \stanza 
                     Nemo
                              hæce
                              uostrum
                              ruminetur
                              mulieri. \&
                         \stanza 
                     ...Sollemnitusque
                              {Sollemnitusque}
                              deo
                              litat
                              laudem
                              lubens. \&
                         \stanza ... in
                              sedes
                              conlocat
                              se
                              regias: &
                     Clytemnestra
                              iuxtim
                              ,
                              tertias
                              natæ
                              occupant. \&
                         \stanza 
                     Ipsus
                              se
                              in
                              terram
                              saucius
                              fligit
                              cadens. \&
                         \stanza Quin
                              quod
                              parere
                              mihi
                              uos
                              maiestas
                              mea &
                     procat
                              ,
                              toleratis
                              temploque
                              {temploque}
                              hanc
                              deducitis
                              ? \&
                         \stanza 
                     Iamne
                              {Iamne}
                              oculos
                              specie
                              lætauisti
                              optabili
                              ? \&
                     
                  \endnumbering
		\end{Leftside}
                  \begin{Rightside}
			\beginnumbering
			\numberstanzafalse
                     
                         \stanza En réalité, quand Pergame fût &enflammée on eût partagé équitablement &
                     le butin entre les participants. \&
                         \stanza Alors, voici la horde enjouée de Nérée, avec son bec aplati, &
                     qui tourne autour de la flotte en s’amusant des chants [des
                              marins]... \&
                         \stanza 
                     Que personne parmi vous ne rumine [ses pensées] à la femme. \&
                         \stanza 
                     Et la solennité extatique offre un éloge au dieu. \&
                         \stanza . . .il s’assied sur le trône royal : &
                     à côté Clytemnestre, ses filles occupent les troisièmes [places]. \&
                         \stanza 
                     Le blessé se heurte lui-même en tombant à terre. \&
                         \stanza Comment? Vous ne supportez pas que ma majesté, &
                     vous demande de m’obéir ? Pourquoi n’emmenez-vous pas celle-ci au
                              sanctuaire ? \&
                         \stanza 
                     N’as-tu pas désormais contenté tes yeux par cette vision convoitée
                              ? \&
                     
                  \endnumbering
		\end{Rightside}
               \end{pairs}
	\Columns
            
         
         
            \section{Nævius}
            Texte latin établi et organisé selon les choix de reconstruction et d'édition de \cite{SpaltenNaevius}.\par
            
               \subsection*{Danæ}
               \begin{abstract}
                  Sources hellènes : Eschyle \textit{Δικτυλκοί}; Sophocle
                        \textit{Ακρισιος}, \textit{Δαναη},
                     et \textit{Λαρισαῖοι}; Euripide \textit{Δαναη}.\par
                  Argument: cette tragédie aurait pour sujet la maternité de Danæ ; l’intrigue
                     pourrait inclure l’intrigue amoureuse entre Jupiter et Danæ, mais aussi son
                     accouchement et ses conséquences.\par
               \end{abstract}
               \begin{pairs}
                  \begin{Leftside}
			\beginnumbering
			\setcounter{stanzaL}{0}
                     
                         \stanza 
                     
                              Contemplo
                              placide
                              formam
                              et
                              faciem
                              virginis. \&
                         \stanza 
                     Omnes
                              formidant
                              homines
                              eius
                              valentiam
                              . \&
                         \stanza 
                     
                              Excidit
                              orationis
                              omnis
                              confidentia
                              . \&
                         \stanza 
                     Manubiæ
                              subpetant
                              pro
                              me \&
                         \stanza 
                     Suo
                              sonitu
                              claro
                              fulgorivit
                              Juppiter \&
                         \stanza 
                     .
                              .
                              .
                              .
                              quæ
                              quondam
                              fulmine
                              icit
                              Iuppiter \&
                         \stanza 
                     Eam
                              nunc
                              esse
                              inventam
                              probris
                              compotem
                              Scis
                              . \&
                         \stanza 
                     Desubito
                              famam
                              tollunt
                              ,
                              si
                              quam
                              solam
                              videre
                              in
                              via
                              . \&
                         \stanza 
                     Quin
                              ,
                              ut
                              quisque
                              est
                              meritus
                              ,
                              præsens
                              pretium
                              pro
                              factis
                              ferat. \&
                         \stanza 
                     .
                              .
                              .
                              indigne
                              exigor
                              patria
                              innocens
                              . \&
                         \stanza 
                     .
                              .
                              amnis
                              niveo
                              eo
                              fonte
                              lavere
                              me
                              memini
                              manum
                              . \&
                     
                  \endnumbering
		\end{Leftside}
                  \begin{Rightside}
			\beginnumbering
			\numberstanzafalse
                     
                         \stanza 
                     Je considère paisiblement le corps et le visage de la vierge. \&
                         \stanza 
                     Tous les hommes redoutent sa force physique. \&
                         \stanza 
                     La confiance en toute parole est perdue  \&
                         \stanza 
                     Les éclairs qu’ils soient à ma disposition \&
                         \stanza 
                     Jupiter a lancé des éclairs avec le bruyant fracas qui est le
                              sien. \&
                         \stanza 
                     ... autrefois, Jupiter les a frappées avec la foudre \&
                         \stanza 
                     Maintenant, tu sais que celle qui a obtenu le déshonneur a été
                              découverte. \&
                         \stanza 
                     Ils portent la rumeur subitement, toutes les fois qu’elle est aperçue
                              seule dans la rue. \&
                         \stanza 
                     Eh bien, chacun a mérité qu'il emporte le salaire présent en vertu de
                              ce qu'il a fait. \&
                         \stanza 
                     ... innocente, je suis jugée indignement par ma patrie. \&
                         \stanza 
                     .. je me souviens de ma main trempé ici par l'eau jaillissante d'un
                              fleuve blanc comme la neige. \&
                     
                  \endnumbering
		\end{Rightside}
               \end{pairs}
	\Columns
            
            
               \subsection*{Lycurgus}
               \begin{abstract}
                  Sources hellènes : Homère l'\textit{Ἰλιάς}; Eschylle
                        \textit{Λυκουργεια}.\par
                  Argument : Après avoir été averti de la venue d’un étranger accompagné de
                     bacchantes, Lycurge décide de chasser ces agitateurs en envoyant ses troupes à
                     leur poursuite. Des bacchantes sont tuées, d’autres capturées puis emprisonnées
                     avec Bacchus. Quand le dieu, resté anonyme, comparait au palais, il montre son
                     pouvoir, et punit ses persécuteurs.\par
               \end{abstract}
               \begin{pairs}
                  \begin{Leftside}
			\beginnumbering
			\setcounter{stanzaL}{0}
                     
                         \stanza 
                     Tuos
                              qui
                              celsos
                              terminos
                              tutant
                              .
                              .
                              . \&
                         \stanza 
                     
                              Alte
                              iubatos
                              angues
                              in
                              sese
                              gerunt
                              . \&
                         \stanza 
                     Quaque
                              incedunt
                              ,
                              omnes
                              arvas
                              obterunt
                              . \&
                         \stanza 
                              Vos
                              ,
                              qui
                              regalis
                              corporis
                              custodias &agitatis
                              ,
                              ite
                              actutum
                              in
                              frundiferos
                              locos
                              , &
                     ingenio
                              arbusta
                              ubi
                              nata
                              sunt
                              ,
                              non
                              obsitu
                              . \&
                         \stanza Alii &Sublime
                              in
                              altos
                              saltus
                              inlicite
                              .
                              .
                              . &
                     Ubis
                              bipedes
                              volucres
                              lino
                              linquant
                              lumina
                              . \&
                         \stanza 
                              Pergite
                              , &
                     Thyrsigeræ
                              Bacchæ
                              ,
                              Bacchico
                              cum
                              schemate
                              . \&
                         \stanza 
                     suavisonum
                              melos \&
                         \stanza 
                     Ignotæ
                              iteris
                              sumus
                              ,
                              tute
                              scis
                              .
                              .
                              . \&
                         \stanza Ut
                              in
                              venatu
                              vitulantes
                              ex
                              suis &
                     lucis
                              nos
                              mittant
                              pœnis
                              decoratas
                              feris
                              . \&
                         \stanza 
                     Pallis
                              patagiis
                              crocotis
                              malacis
                              mortualibus
                              . \&
                         \stanza 
                     Iam
                              ibi
                              nos
                              duplicat
                              advenientis
                              liberi
                              timos
                              pavos
                              . \&
                         \stanza Nam
                              ludere
                              ut
                              lætantis
                              inter
                              se
                              vidimus
                              præter
                              amnem &
                     creterris
                              sumere
                              aquam
                              ex
                              fonte
                              . \&
                         \stanza 
                     Diabathra
                              in
                              pedibus
                              habebat
                              ,
                              erat
                              amictus
                              epicroco
                              . \&
                         \stanza 
                     Dic
                              quo
                              pacto
                              eum
                              potiti
                              :
                              pugnan
                              {pugnan}
                              an
                              dolis
                              ? \&
                         \stanza 
                     sine
                              ferro
                              ut
                              pecua
                              manibus
                              ad
                              mortem
                              meant \&
                         \stanza Ducite &
                     eo
                              cum
                              argutis
                              linguis
                              mutas
                              quadrupedis
                              . \&
                         \stanza 
                     Cave
                              {sis}
                              tuam
                              contendas
                              iram
                              contra
                              cum
                              ira
                              Liberi
                              . \&
                         \stanza 
                     Ne
                              ille
                              mei
                              feri
                              ingeni
                              iram
                              atque
                              animi
                              acrem
                              acrimoniam \&
                         \stanza Oderunt
                              di
                              homines
                              iniuros
                              . &
                     —
                              Egone
                              {Egone}
                              an
                              ille
                              iniurie
                              Facimus
                              ? \&
                         \stanza 
                     Ut
                              videam
                              Vulcani
                              opera
                              hæc
                              flammis
                              fieri
                              flora \&
                         \stanza 
                     Late
                              longeque
                              longeque
                              transtros
                              nostros
                              fervere \&
                         \stanza Proinde
                              huc
                              Dryante
                              regem
                              prognatum
                              patre
                              , &
                     
                              Lycurgam
                              cette
                              ! \&
                         \stanza 
                     Vos
                              qui
                              astatis
                              obstinati
                              .
                              .
                              . \&
                         \stanza 
                     Se
                              quasi
                              amnis
                              celeris
                              rapit
                              ,
                              sed
                              tamen
                              inflexu
                              flectitur
                              . \&
                         \stanza 
                     
                              Iam
                              solis
                              æstu
                              candor
                              cum
                              liquesceret \&
                     
                  \endnumbering
		\end{Leftside}
                  \begin{Rightside}
			\beginnumbering
			\numberstanzafalse
                     
                         \stanza 
                     Tes troupes qui gardent les hautes frontières ...  \&
                         \stanza 
                     Elles portent sur elle bien haut des serpents à crinières. \&
                         \stanza 
                     Partout où elles pénètrent, elles piétinent tous les champs. \&
                         \stanza Vous, qui vous occupez des gardes du corps &royal, avancez immédiatement vers les lieux &
                     feuillus, où les arbres ont poussé sans avoir été semés. \&
                         \stanza  Vous autres, attirez-les en hauteur dans les  &hauts pâturages ... &
                     Où les bipèdes ailés abandonnent la lumière du jour pour le filet. \&
                         \stanza Continuez, &
                      Bacchantes munies d'un thyrse, avec l'habillement bachique \&
                         \stanza 
                     mélodie suave \&
                         \stanza 
                     Nous ignorons le chemin, toi-même, tu le sais ... \&
                         \stanza De tel sorte qu'ils nous laissent aller vers &
                     leurs bois sacrés, joyeuse, pendant la chasse, ils nous parent de
                              châtiments sauvages. \&
                         \stanza 
                     En robes, bandeaux ornementaux, robes de couleur safran, vêtements
                              doux du deuil. \&
                         \stanza 
                     Dès lors, l'appréhension, la crainte des arrivants libres, redoublent
                              en nous. \&
                         \stanza En effet, nous les avons vues qu'elles jouaient entre elles le long du
                              ruisseau, qu'elles pui- &
                     saient l'eau de la source dans des coupes. \&
                         \stanza 
                     Il avait des chaussures de femme aux pieds, et était enveloppé d'une
                              robe de laine fine. \&
                         \stanza 
                     Dis comment vous vous êtes emparés de lui : par la force ou par la
                              ruse ? \&
                         \stanza 
                     comme un troupeau qui se dirige vers la mort suivant des mains sans
                              fers \&
                         \stanza Conduisez ici &
                     ces femmes silencieuses avec les quadrupèdes aux langages
                              expressifs \&
                         \stanza 
                     Prends garde, je te prie, de ne pas faire rivaliser ta colère avec
                              celle de Liber. \&
                         \stanza 
                     De peur qu’il ne [ provoque] la colère de mon esprit et l’impétueuse
                              acrimonie de mon âme  \&
                         \stanza Les dieux haïssent les hommes injustes &
                     — Est-ce lui, ou moi qui suis déloyal ? \&
                         \stanza 
                     Pour que je voie les oeuvres de Vulcain qui fait des flammes
                              éclatantes  \&
                         \stanza 
                     agiter nos poutres longues et larges \&
                         \stanza Ainsi donc, montre ici le roi Lycurgue descendant par son père
                              Dryas. &
                      \&
                         \stanza 
                     Vous qui demeurez là obstinément ... \&
                         \stanza 
                     Comme si un fleuve violent l’emportait, mais pourtant, celui-ci se
                              plie au contour [de la rive]. \&
                         \stanza 
                     Dès lors, alors que la neige fond par la chaleur du soleil \&
                     
                  \endnumbering
		\end{Rightside}
               \end{pairs}
	\Columns
            
         
         
            \section{Quintus Ennius}
            Texte latin établi et organisé selon les choix de reconstruction et d'édition de \cite{EnniusLoeb}.\par
            
               \subsection*{Achilles}
               \begin{abstract}
                  Sources hellènes : Homère l'\textit{Ἰλιάς}; et les
                     versions du mythe d'Achille d'Aristarque de Tégée, Iophon, Astydamas, Carcinus,
                     Cléophon (?), Evaretus, et Diogène de Sinope. \par
                  Argument: furieux d’avoir été privé de son butin de guerre, Achille se retire
                     sous sa tente et refuse de poursuivre le combat. Agamemnon, subissant de
                     nombreuses défaites, décide de lui envoyer une ambassade afin de convaincre le
                     héros de retourner combattre auprès des Grecs. La tragédie s’achevait
                     certainement par la mort de Patrocle qui suscitera le retour d’Achille parmi
                     les guerriers grecs.\par
               \end{abstract}
               \begin{pairs}
                  \begin{Leftside}
			\beginnumbering
			\setcounter{stanzaL}{0}
                     
                         \stanza 
                     form
                              Ita
                              magni
                              fluctus
                              eiciebantur
                              .
                              .
                              . \&
                         \stanza .
                              .
                              .
                              per
                              ego
                              deum
                              sublimas
                              subices &
                     Umidas
                              ,
                              unde
                              oritur
                              imber
                              sonitu
                              sævo
                              [
                              et
                              spiritu
                              . \&
                         \stanza 
                     
                              Prolato
                              ære
                              astitit \&
                         \stanza 
                     .
                              .
                              .
                              nam
                              consiliis
                              obvarant
                              ,
                              quibus
                              tam
                              concedit
                              hic
                              ordo
                              . \&
                         \stanza Quo
                              nunc
                              incerta
                              re
                              atque
                              inorata
                              gradum &
                     regredere
                              conare \&
                         \stanza 
                     Serva
                              cives
                              ,
                              defende
                              hostes
                              ,
                              cum
                              potes
                              [
                              defendere \&
                         \stanza 
                     Interea
                              mortales
                              inter
                              sese
                              pugnant
                              ,
                              proeliant
                              . \&
                         \stanza summam
                              tu
                              tibi &pro
                              mala
                              vita
                              famam
                              extolles
                              ,
                              [
                              et
                              ]
                              pro
                              bona
                              partam
                              gloriam
                              . &
                     
                              Male
                              volentes
                              [
                              enim
                              ]
                              famam
                              tollunt
                              ,
                              bene
                              volentes
                              gloriam
                              . \&
                     
                  \endnumbering
		\end{Leftside}
                  \begin{Rightside}
			\beginnumbering
			\numberstanzafalse
                     
                         \stanza 
                      Ainsi les grands flots étaient jetés ...  \&
                         \stanza  ... Moi, à travers les sublimes marche- &
                     pieds humides des dieux, de là les pluies naissent d’un souffle et
                              d’un retentissement impétueux. \&
                         \stanza 
                     Il s’est tenu debout avec l’airain porté en avant \&
                         \stanza 
                     ... car ils font obstacle aux résolutions, si par elles l’ordre
                              s’éloigne d’ici. \&
                         \stanza Maintenant, là où l’événement est incertain et méconnu, tu tentes de
                              revenir sur &
                     tes pas \&
                         \stanza 
                     Sauve les citoyens, repousse les ennemies puisque tu peux les
                              repousser \&
                         \stanza 
                     Pendant que les mortels combattent entre eux, ils livrent bataillent.
                            \&
                         \stanza Tu élèveras le plus haut sommet pour toi- &même, la renommée en faveur d’une mauvaise vie [ et] la gloire
                              engendrée pour une bonne. &
                     [ En réalité], ceux qui veulent le mal élèvent la renommée, ceux qui
                              veulent le bien la gloire. \&
                     
                  \endnumbering
		\end{Rightside}
               \end{pairs}
	\Columns
            
            
               \subsection*{Alcmæon}
               \begin{abstract}
                  Sources hellènes : Homère l'\textit{Ὀδύσσεια}; Euripide
                        \textit{Ἀλκμαίων ὁ διὰ Κορίνθου} et \textit{Ἀλκμαίων ὁ διὰ Ψωφῖδος}; ainsi que les versions
                     fragmentaires de Timothée d'Athènes, Astydamas II, Théodecte de Phasélis,
                     Evaretus et Nicomaque d'Alexandrie.\par
                   Argument: Alcméon, tue sa mère afin de venger son père, Amphiaraos: celui-ci
                     avait été contraint par sa femme de participer à la guerre contre des Épigones,
                     le condamnant à une mort certaine. À la suite de ce matricide, Alcméon sera
                     alors poursuivi par les Furies : celles-ci pousseront le héros à remettre en
                     question son acte.\par
               \end{abstract}
               \begin{pairs}
                  \begin{Leftside}
			\beginnumbering
			\setcounter{stanzaL}{0}
                     
                         \stanza Multis
                              sum
                              modis
                              circumventus
                              ,
                              morbo
                              ,
                              exilio
                              atque
                              inopia
                              ; &Tum
                              pavor
                              sapientiam
                              omnem
                              mi
                              exanimato
                              expectorat
                              . &†
                              Alter
                              †
                              terribilem
                              minatur
                              vitæ
                              cruciatum
                              et
                              necem
                              ; &quæ
                              nemo
                              est
                              tam
                              firmo
                              ingenio
                              et
                              tanta
                              [
                              confidentia
                              , &
                     quin
                              refugiat
                              timido
                              sanguen
                              atque
                              exalbescat
                              metu
                              . \&
                         \stanza 
                     Sed
                              mihi
                              ne
                              utiquam
                              cor
                              consentit
                              cum
                              oculorum
                              aspectu
                              .
                              . \&
                         \stanza .
                              .
                              .
                              .
                              unde
                              hæc
                              flamma
                              oritur
                              ? &{
                              .
                              .
                              .
                              } &Incedunt
                              ,
                              incedunt
                              :
                              adsunt
                              adsunt
                              ,
                              me
                              [
                              expetunt &.
                              {
                              .
                              .
                              .
                              } &Fer
                              mi
                              auxilium
                              ,
                              pestem
                              abige
                              a
                              me
                              , &flammiferam
                              hanc
                              vim
                              ,
                              quæ
                              me
                              excruciat
                              ; &Cæruleæ
                              incinctæ
                              igni
                              incedunt
                              , &circumstant
                              cum
                              ardentibus
                              tædis
                              . &{
                              .
                              .
                              .
                              } &Intendit
                              crinitus
                              Apollo &Arcum
                              auratum
                              ,
                              luna
                              innixus
                              , &
                     Diana
                              facem
                              iacit
                              a
                              læva
                              . \&
                         \stanza 
                     factum
                              est
                              iam
                              diu \&
                     
                  \endnumbering
		\end{Leftside}
                  \begin{Rightside}
			\beginnumbering
			\numberstanzafalse
                     
                         \stanza Par bien des manières, je suis entouré par la maladie, l’exil et le
                              dénuement. &Alors l’effroi bannit de mon esprit toute prudence qui m’épuise. &Quelqu’un menace la vie d’un terrible supplice et de meurtre; &tel que personne n’a un esprit si ferme et tant de confiance, &
                     que son sang ne recule pas devant l’inquiétude et qu'il ne pâlit pas
                              face à l’anxiété. \&
                         \stanza 
                     Mais mon cœur n’est nullement d’accord avec la vision de mes yeux.
                            \&
                         \stanza ... d’où naît cette flamme ? &\{...\} &Elles arrivent, elles arrivent, elles sont là, elles sont là, elles
                              cherchent à m’attendre &\{...\} &Rapporte-moi de l’aide, éloigne de moi ce &fléau, cette puissance enflammée qui me torture; de couleur azur,
                              elles s’avancent &ceintes par le feu,  &elles se tiennent de toutes &parts avec des torches ardentes. &\{...\} &Apollon à la longue chevelure bande &
                     l’arc doré, s’appuyant sur le croissant Diane lance une torche de la
                              main gauche \&
                         \stanza 
                     Cela a été fait il y a déjà longtemps. \&
                     
                  \endnumbering
		\end{Rightside}
               \end{pairs}
	\Columns
            
            
               \subsection*{Alexandrus}
               \begin{abstract}
                  Sources hellènes : Homère l'\textit{Ὀδύσσεια}; \textit{Ἀλέξανδρος} de Sophocle ainsi que la version
                     d'Euripide, \textit{Ἀλκμαίων ὁ διὰ Ψωφῖδος}, et Nicomaque
                     d'Alexandrie \textit{τροίας}.\par
                  Argument: ce drame se concentre sur l’enfance de Paris: son père commande son
                     assassinat à la suite du rêve prémonitoire que fait sa femme, Hécube, qui le
                     lie à la destruction de Troie . Abandonné sur le mont Ida par les serviteurs du
                     palais, l’enfant est recueilli par des bergers qui le nomment Alexandre. Devenu
                     adulte il participe aux jeux organisés par Priam, et y vainc ses frères.
                     Ceux-ci n’acceptent pas d’avoir été battus par un berger; l’un d’eux, Déiphobe
                     entreprend même de le tuer. Cassandre intervient, et révèle la véritable
                     identité du vainqueur ainsi que la destruction future de Troie. \par
               \end{abstract}
               \begin{pairs}
                  \begin{Leftside}
			\beginnumbering
			\setcounter{stanzaL}{0}
                     
                         \stanza Iam
                              dudum
                              ab
                              ludis
                              animus
                              atque
                              aures
                              [
                              avent &
                     avide
                              expectantes
                              nuntium
                              . \&
                         \stanza 
                     Qua
                              propter
                              Parim
                              pastores
                              nunc
                              Alexandrum
                              vocant
                              .
                              .
                              . \&
                         \stanza 
                     †
                              amidio
                              †
                              purus
                              putus
                              . \&
                         \stanza Hominem
                              appellat
                              :
                              quid
                              lascivis
                            &
                     stolide
                              ?
                              Non
                              intellegit
                              . \&
                         \stanza 
                     Volans
                              de
                              cælo
                              cum
                              corona
                              et
                              tæniis \&
                         \stanza Multi
                              alii
                              adventant
                              ,
                              paupertas
                              quorum &
                     obscurat
                              nomina
                              . \&
                         \stanza O
                              lux
                              Troiæ
                              ,
                              germane
                              Hector
                              ; &quid
                              ita
                              cum
                              tuo
                              lacerato
                              corpore
                              , &
                     miser
                              es
                              ,
                              aut
                              qui
                              te
                              sic
                              respectantibus
                              tractavere
                              nobis
                              ? \&
                         \stanza Nam
                              maximo
                              saltu
                              superavit
                              gravidus
                              armatis
                              equus
                              , &
                     Suo
                              qui
                              partu
                              ardua
                              perdat
                              Pergama
                              . \&
                     
                  \endnumbering
		\end{Leftside}
                  \begin{Rightside}
			\beginnumbering
			\numberstanzafalse
                     
                         \stanza Depuis longtemps, l’âme et les oreilles désirent avidement et
                              attendent une nouvelle &
                     des jeux. \&
                         \stanza 
                     Pour cette raison, les bergers ont désormais appelé Paris, Alexandre .
                              . \&
                         \stanza 
                     ? absolument pur \&
                         \stanza Il s’adresse à l’homme : pourquoi  &
                      badinez-vous sottement? Il ne comprend pas. \&
                         \stanza 
                     Volant en haut du ciel avec une couronne et des bandelettes  \&
                         \stanza Beaucoup d’autres approchent, dont la  &
                     pauvreté obscurcit leurs noms. \&
                         \stanza  Ô lumière de Troie, ô authentique Hector; &pourquoi es-tu ainsi misérable avec ton corps lacéré ? &
                     ou bien, quels hommes t’ont traité ainsi quand nous avions les yeux
                              tournés ?  \&
                         \stanza De fait, le cheval, lesté avec des armes, abattu par son très grand
                              saut qui détruit &
                     les forteresses de Pergame avec sa descendance. \&
                     
                  \endnumbering
		\end{Rightside}
               \end{pairs}
	\Columns
            
            
               \subsection*{Andromacha Æchmalotis}
               \begin{abstract}
                  Sources hellènes : Homère l'\textit{Ἰλιάς}; Euripide
                        \textit{Ἀνδρομάχη}, \textit{Ἑκάβη}
                     et \textit{Τρῳάδες}; Antiphon \textit{Ἀνδρομάχη}.\par
                  Argument: l’intrigue de cette tragédie se situe au moment de la chute de Troie
                     : la pièce développe les lamentations d’Andromaque qui atteignent leur
                     apothéose lors du meurtre d’Astyanax. \par
               \end{abstract}
               \begin{pairs}
                  \begin{Leftside}
			\beginnumbering
			\setcounter{stanzaL}{0}
                     
                         \stanza Vidi
                              ,
                              videre
                              quod
                              me
                              passa
                              ægerrume
                              , &
                     Hectorem
                              curru
                              quadriiugo
                              raptarier \&
                         \stanza 
                              Ex
                              opibus
                              summis
                              opis
                              egens
                              ,
                              Hector
                              ,
                              [
                              tuæ &{
                              .
                              .
                              .
                              } &Quid
                              petam
                              præsidi
                              aut
                              exequar
                              ?
                              quove
                              {quove}
                              [
                              nunc &auxilio
                              aut
                              exili
                              aut
                              fugæ
                              freta
                              sim
                              ? &Arce
                              et
                              urbe
                              orba
                              sum
                              .
                              Quo
                              accedam
                              ?
                              [
                              Quo
                              applicem
                              ? &Cui
                              nec
                              aræ
                              patriæ
                              domi
                              stant
                              ,
                              fractæ
                              et
                              [
                              disiectæ
                              iacent
                              ; &fana
                              flamma
                              deflagrata
                              ,
                              tosti
                              alti
                              stant
                              parietes &Deformati
                              atque
                              abiete
                              crispa
                              .
                              .
                              . &{
                              .
                              .
                              .
                              } &O
                              pater
                              ,
                              o
                              patria
                              ,
                              o
                              Priami
                              domus
                              , &Sæptum
                              altisono
                              cardine
                              templum
                              ! &Vidi
                              ego
                              te
                              adstante
                              ope
                              barbarica &tectis
                              cælatis
                              laqueatis
                              , &auro
                              ebore
                              instructam
                              regifice. &{
                              .
                              .
                              .
                              } &Haec
                              omnia
                              vidi
                              inflammari &,
                              Priamo
                              vi
                              vitam
                              evitari, &
                     Iovis
                              aram
                              sanguine
                              turpari \&
                         \stanza 
                     Acherusia
                              templa
                              alta
                              Orci
                              salvete
                              infera
                              .
                              .
                              . \&
                         \stanza 
                     Andromachæ
                              nomen
                              qui
                              indidit
                              ,
                              recte
                              [
                              indidit
                              aut
                              Alexandrum
                              .
                              .
                              . \&
                         \stanza 
                     di
                              .
                              .
                              .
                              on
                              est
                              :
                              Nam
                              mussare
                              si
                              .
                              .
                              . \&
                         \stanza 
                     Quid
                              fit
                              ?
                              seditio
                              tabetne
                              {tabetne}
                              ,
                              an
                              numeros
                              augificat
                              suos
                              ? \&
                         \stanza 
                     Quantis
                              cum
                              ærumnis
                              illum
                              exanclavi
                              diem \&
                         \stanza .
                              .
                              .
                              annos
                              multos
                              longinque
                              domo &
                     bellum
                              gerentes
                              summum
                              summa
                              industria \&
                         \stanza Nam
                              ubi
                              introducta
                              est
                              ,
                              puerumque
                              {puerumque}
                              ,
                              ut
                              [
                              laverent
                              ,
                              locant &
                     In
                              clupeo \&
                         \stanza 
                     Nam
                              neque
                              irati
                              neque
                              blandi
                              quicquam
                              [
                              sincere
                              sonunt \&
                         \stanza .
                              .
                              .
                              sed
                              quasi
                              ferrum
                              aut
                              lapis &
                     durat
                              ,
                              rarenter
                              gemitum
                              †
                              conatur
                              trabem
                              † \&
                         \stanza 
                     rapit
                              ex
                              alto
                              naves
                              velivolas \&
                     
                  \endnumbering
		\end{Leftside}
                  \begin{Rightside}
			\beginnumbering
			\numberstanzafalse
                     
                         \stanza 
                      \&
                         \stanza J’ai vu que j’ai enduré de voir avec la plus grande souffrance &
                     Hector être emporté par un quadrige \&
                         \stanza d’après les plus grands secours, ayant besoin de ton secours,
                              Hector &\{...\} &Quelle protection dois-je solliciter ou poursuivre ? Ou à présent, en
                              quelle aide, l’exil,  &ou la fuite, puis-je compter ? &Je suis orpheline de ville et de citadelle. Où dois-je aller ? Sur qui
                              puis-je m’adosser ?  &Moi, à qui ni les hôtels paternels ni le foyer ne subsistent:
                              dispersés et brisés, ils sont en ruines;  &les temples consumés par les flammes, les hauts murs brûlés et altérés
                              subsistent grâce &au sapin tordu... &\{...\} &Ô père, ô patrie, ô maison de Priam, ô temple  &enceint d’une ligne sublime qui résonne &fort. Je t’ai vu quand le soutien se tenait  &aux côtés des étrangers avec tes toits sculptés et lambrissés, pourvu
                              royalement avec de l’or et de l’ivoire. &\{...\} &J’ai vu toutes ces choses être incendiées, &la vie de Priam être ôtée par la force, &
                     l’autel de Jupiter être souillé par le sang \&
                         \stanza 
                     Profonds sanctuaires achérusien d’Orcus, salutations...  \&
                         \stanza 
                     Celui qui a donné le nom d’Andromaque l’a donné convenablement, ou
                              Alexandre … \&
                         \stanza 
                     les dieux … [ ?] est, en vérité taire si …  \&
                         \stanza 
                     Qu’arrive-t-il ? Est-ce que le soulèvement se désagrège ? Ou est-ce
                              qu’il accroît leurs effectifs ? \&
                         \stanza 
                     Avec quelles grandes peines ai-je enduré ce jour \&
                         \stanza  ... depuis de nombreuses années, loin du &
                     foyer, puisqu’ ils accomplissent une guerre importante avec la plus
                              grande application  \&
                         \stanza En réalité, lorsqu’elle a été conduite à l’intérieur, et comme ils
                              lavaient l’enfant, ils &
                     le placent sur le bouclier \&
                         \stanza 
                     De fait ni les impétueux, ni les flatteurs, ne clament quelque chose
                              sincèrement \&
                         \stanza ... mais, comme la pierre endurcit le fer,  &
                     il se prépare à gémir rarement à la massue  \&
                         \stanza 
                     il emporte les voiliers en provenance du large \&
                     
                  \endnumbering
		\end{Rightside}
               \end{pairs}
	\Columns
            
            
               \subsection*{Andromeda}
               \begin{abstract}
                  Sources hellènes : les versions d'\textit{Ἀνδρομέδα} de
                     Sophocle, Euripide, Lycophron, Phrynichos.\par
                   Argument: cassiopée, reine de l’Éthiopie, se targue d’avoir une fille plus
                     belle que les Néréides : son excès d’orgueil offense Neptune qui inonde les
                     côtes éthiopiennes et envoie un monstre marin. Désespéré, le roi consulte
                     l’oracle d’Ammon qui révèle que seul le sacrifice d’Andromède au monstre pourra
                     faire cesser le fléau. Celle-ci est alors enchaînée à un rocher.\par
                   La critique \footcite{voir\cite{EnniusLoeb}, p.41} suppose
                     qu’Ennius suit précisément l’intrigue du drame d’Euripide: la pièce
                     commencerait alors qu’Andromède se lamente attendant sa mort sur un rocher ;
                     puis arrive Persée tombant éperdument amoureux de la belle, il décide de la
                     sauver en combattant le monstre qui met en péril son bonheur.\par
               \end{abstract}
               \begin{pairs}
                  \begin{Leftside}
			\beginnumbering
			\setcounter{stanzaL}{0}
                     
                         \stanza quæ
                              cava
                              cæli &
                     signitenentibus
                              conficis
                              bigis \&
                         \stanza 
                     Liberum
                              quæsendum
                              causa
                              familiæ
                              matrem
                              tuæ
                              . \&
                         \stanza 
                     Scrupeo
                              investita
                              saxo
                              ,
                              atque
                              ostreis
                              squamæ
                              scabrent
                              . \&
                         \stanza .
                              .
                              .
                              circum
                              sese
                              urvat
                              ad
                              pedes, &
                     
                              a
                              terra
                              quadringentos
                              caput \&
                         \stanza 
                     .
                              corpus
                              contemplatur
                              ,
                              unde
                              corporaret
                              [
                              vulnere \&
                         \stanza 
                     .
                              .
                              .
                              rursus
                              prosus
                              reciprocat
                              fluctus
                              feram \&
                         \stanza 
                              .
                              .
                              .
                              .
                              alia
                              fluctus
                              differt
                              dissupat
                              , &
                     
                              visceratim
                              membra
                              ,
                              maria
                              salsa
                              spumant
                              sanguine
                              . \&
                         \stanza 
                     A
                              filiis
                              propter
                              te
                              objecta
                              sum
                              innocens
                              [
                              Nerei
                              . \&
                     
                  \endnumbering
		\end{Leftside}
                  \begin{Rightside}
			\beginnumbering
			\numberstanzafalse
                     
                         \stanza toi qui parcours les profondeurs du ciel &
                     avec ton char étoilé \&
                         \stanza 
                     Pour demander des enfants à la mère de ta famille.  \&
                         \stanza 
                     Revêtus par une roche revêche, et les écailles sont hérissées par des
                              huîtres.  \&
                         \stanza ... autour de lui, il trace le sillon d’enceinte &
                     de la capitale jusqu’à quatre cents pieds du sol \&
                         \stanza 
                     il considère le corps, là où il pourrait le tuer par une plaie  \&
                         \stanza 
                     Le flot fait osciller la bête d’avant en arrière \&
                         \stanza Le flot disperse en lambeaux les autres mem- &
                     bres, les océans salés écument du sang. \&
                         \stanza 
                     À cause de toi, j’ai été exposée, innocente, aux filles de Nérée. \&
                     
                  \endnumbering
		\end{Rightside}
               \end{pairs}
	\Columns
            
            
               \subsection*{Herctor_lytra}
               \begin{abstract}
                   Sources hellènes : Homère l'\textit{Ἰλιάς}; Eschyle
                        \textit{Φρύγες ἢ Ἕκτορος Λύτρα}
                  \par
                   Argument: \textit{Le départ d'Hector} se concentre sur de la rançon
                     qu’offre Priam à Achille en échange de la dépouille de son fils Hector. Les
                     fragments ne permettent pas de décrire précisément le fil de la tragédie. \par
               \end{abstract}
               \begin{pairs}
                  \begin{Leftside}
			\beginnumbering
			\setcounter{stanzaL}{0}
                     
                         \stanza 
                     quæ
                              mea
                              comminus
                              machæra
                              atque
                              hasta
                              †
                              hospius
                              manu† \&
                         \stanza 
                     sublime
                              iter
                              quadrupedantes
                              flammam
                              [
                              halitantes \&
                         \stanza Nos
                              quiescere
                              æquum
                              est
                              ?
                              nomus
                              ambo &
                     Ulixem \&
                         \stanza At
                              ego
                              ,
                              omnipotens &
                     
                              ,
                              ted
                              exposco
                              ,
                              ut
                              hoc
                              consilium
                              Achivis
                              auxilio
                              fuat
                              . \&
                         \stanza 
                     .
                              .
                              .
                              inferum
                              vastos
                              specus \&
                         \stanza Hector
                              vi
                              summa
                              armatos
                              educit
                              foras &
                     ,
                              castrisque
                              {castrisque}
                              castra
                              ultro
                              iam
                              conferre
                              occupat \&
                         \stanza Melius
                              est
                              virtute
                              ius
                              :
                              nam
                              sæpe
                              virtutem
                              mali
                              , &
                     nanciscuntur
                              ;
                              ius
                              atque
                              æcum
                              se
                              a
                              malis
                              spernit
                              procul
                              . \&
                         \stanza †
                              ducet
                              quadrupedum
                              iugo
                              invitam
                              doma
                              infrena &
                     et
                              iuge
                              valida
                              quorum
                              tenacia
                              infrenari
                              minis
                              †
                              . \&
                         \stanza 
                     Constitit
                              ,
                              credo
                              ,
                              Scamander
                              ;
                              arbores
                              vento
                              vacant \&
                         \stanza 
                     
                              Qui
                              cupiant
                              dare
                              arma
                              Achilli
                              †
                              ut
                              ipse
                              †
                              ,
                              cunctent \&
                         \stanza .
                              .
                              .
                              per
                              vos
                              et
                              vostrorum &
                     imperium
                              et
                              fidem
                              ,
                              Myrmidonum
                              vigiles
                              ,
                              commiserescite
                              ! \&
                         \stanza 
                     Quid
                              hoc
                              hic
                              clamoris
                              ?
                              quid
                              tumulti
                              est
                              ?
                              nomen
                              qui
                              usurpat
                              meum? \&
                         \stanza 
                     Quid
                              in
                              castris
                              strepiti
                              est
                              ? \&
                         \stanza 
                     Æs
                              sonit
                              ,
                              franguntur
                              hastæ
                              ,
                              terra
                              sudat
                              sanguine \&
                         \stanza 
                     Sæviter
                              fortunam
                              ferro
                              cernunt
                              de
                              victoria
                              . \&
                         \stanza ecce
                              autem
                              caligo
                              oborta
                              est
                              ,
                              omnem
                              prospectum
                              abstulit &
                     derepente
                              ;
                              contulit
                              sese
                              in
                              pedes \&
                     
                  \endnumbering
		\end{Leftside}
                  \begin{Rightside}
			\beginnumbering
			\numberstanzafalse
                     
                         \stanza 
                     que sans délai le coutelas et le sabre te payent en retour par ma
                              main \&
                         \stanza 
                      Les chevaux émanent du feu vers leur route céleste \&
                         \stanza Quand est-ce que nous nous reposons ? Nous reconnaissons ensemble
                              Ulysse. &
                      \&
                         \stanza Mais moi, tout-puissant, je te demande que &
                      que cette mesure soit une aide pour tous les Grecs. \&
                         \stanza 
                     les immenses cavernes du chtonien \&
                         \stanza Hector chasse les hommes armés dehors &
                     par une puissance absolue, et déjà il s’occupe de réunir les camps
                              au-delà des camps \&
                         \stanza La justice est mieux que la bravoure : de fait, les malfaiteurs
                              acquièrent souvent la &
                     bravoure ; et la justice impartiale s’éloigne loin des
                              malfaiteurs. \&
                         \stanza Elle conduira le cheval par le joug, dompte, malgré elle, les
                              impétueux, et les  &
                     vigoureux, parmi lesquels, rétifs au joug, sont domptés par les
                              menaces. \&
                         \stanza 
                     Je crois que le Scamandre s’est stabilisé : les arbres sont à l’abri
                              du vent. \&
                         \stanza 
                     Ceux-ci désirent que lui-même donne les armes d’Achille si bien qu’ils
                              hésitent \&
                         \stanza ... par vous, par votre autorité et votre &
                     fidélité, patrouilles des Myrmidons, ayez pitié ! \&
                         \stanza 
                     Pourquoi ce cri ici ? Pourquoi y a-t-il du bruit ? Qui usurpe mon nom
                              ?  \&
                         \stanza 
                     Pourquoi y a-t-il du vacarme dans les camps ?  \&
                         \stanza 
                     Le bronze sonne, les lances sont rompues, la terre ruisselle de
                              sang \&
                         \stanza 
                     Ils distinguent avec rigueur la chance par le fer de la victoire. \&
                         \stanza  Mais voilà que le brouillard est apparu, il a emporté toute
                              perspective ; il engage le  &
                     combat. \&
                     
                  \endnumbering
		\end{Rightside}
               \end{pairs}
	\Columns
            
            
               \subsection*{Hecuba}
               \begin{abstract}
                  Sources hellènes : Homère l'\textit{Ἰλιάς}; Euripide
                        \textit{Ἑκάβη} et \textit{Τρῳάδες}.\par
                  Argument: l’action se situe après la chute de Troie, et se concentre sur le
                     sort subi par la reine Hécube. Ainsi, la tragédie rapporte ses lamentations et
                     sa capture par Ulysse.\par
               \end{abstract}
               \begin{pairs}
                  \begin{Leftside}
			\beginnumbering
			\setcounter{stanzaL}{0}
                     
                         \stanza 
                     O
                              magna
                              templa
                              cælitum
                              commixta
                              [
                              stellis
                              splendidis
                              ! \&
                         \stanza Hæc
                              tu
                              etsi
                              perverse
                              dices
                              ,
                              facile
                              Achivos
                              flexeris
                              ; &nam
                              opulenti
                              cum
                              locuntur
                              pariter
                              atque
                              ignobiles
                              , &
                     eadem
                              dicta
                              eademque
                              {eademque}
                              oratio
                              æqua
                              non
                              æque
                              valet \&
                         \stanza 
                     vide
                              hunc
                              ,
                              meæ
                              in
                              quem
                              lacrumæ
                              guttatim
                              cadunt \&
                         \stanza 
                     Juppiter
                              tibi
                              summe
                              tandem
                              male
                              re
                              gesta
                              gratulor
                              . \&
                         \stanza Set
                              numquam
                              scripstis
                              ,
                              qui
                              parentem
                              [
                              aut
                              hospitem &
                     
                              necasset
                              quo
                              quis
                              cruciatu
                              perbiteret \&
                         \stanza 
                     
                              .
                              .
                              .
                              undantem
                              salum \&
                         \stanza 
                     
                              Heume
                              miseram
                              !
                              interii
                              :
                              pergunt
                              lavere
                              sanguen
                              sanguine
                              . \&
                         \stanza 
                     Quæ
                              tibi
                              in
                              concubio
                              verecunde
                              et
                              modice
                              morem
                              gerit
                              . \&
                         \stanza .
                              .
                              .
                              miserete
                              anuis
                              : &
                     
                              date
                              ferrum
                              ,
                              qui
                              me
                              anima
                              privem \&
                         \stanza Senex
                              sum
                              :
                              utinam
                              mortem
                              oppetam
                              ,
                              [
                              prius
                              quam
                              evenat &
                     quod
                              in
                              pauperie
                              mea
                              senex
                              graviter
                              gemam
                              ! \&
                     
                  \endnumbering
		\end{Leftside}
                  \begin{Rightside}
			\beginnumbering
			\numberstanzafalse
                     
                         \stanza 
                     Ô grands sanctuaires des cieux, mêlés [aux étoiles étincelantes! \&
                         \stanza Bien que, tu diras cela de travers, tu auras incliné aisément les
                              grecques; &car les hommes opulents parlent de la même manière que ceux de basse
                              naissance, &
                     avec les mêmes mots et les mêmes paroles, qui n’ont pas la même
                              valeur \&
                         \stanza 
                     Regarde celui sur qui mes larmes tombent goutte à goutte \&
                         \stanza 
                      Ô Jupiter souverain, en fin de compte, je te loue pour l’événement
                              funeste accompli. \&
                        
                           Mais vous n’avez jamais écrit par quel supplice devait périr celui qui
                              avait tué un 
                           parent ou un hôte
                        
                         \stanza 
                     ... la mer agitée  \&
                         \stanza 
                     Hélas, comme je suis misérable ! J’agonise! Ils continuent de laver le
                              sang par le sang. \&
                         \stanza 
                     Laquelle accomplit pendant l’union charnelle tes désirs avec retenue
                              et tempérance \&
                        
                           Ayez pitié d’une vieille femme: 
                           donnez-moi le glaive qui m’ôtera la vie
                        
                        
                           Je suis une vieille femme : si seulement je pouvais affronter la mort
                              avant que cela 
                           n’arrive. parce que moi, en tant que vieille femme, je me lamenterai
                              dans la pauvreté 
                        
                     
                  \endnumbering
		\end{Rightside}
               \end{pairs}
	\Columns
            
            
               \subsection*{Iphigenia}
               \begin{abstract}
                   Sources hellènes : Homère l'\textit{Ἰλιάς}; Euripide
                        \textit{Ἰφιγένεια ἡ ἐν Αὐλίδι}, \\textit{Ἰφιγένεια ἐν Ταύροις}; Eschyle \textit{Ὀρέστεια}, \textit{Ἰφιγένεια}?; Sophocle \textit{Ἰφιγένεια}?..\par
                  Argument: cette tragédie relate le sacrifice d’Iphigénie, que son père commande
                     afin d'obtenir les vents favorables à la navigation. Elle se constitue d’un
                     passage où Achille critique les astrologues, de dialogues entre Agamemnon et
                     une servante, et d'une altercation entre Agamemnon et Ménélas. \par
               \end{abstract}
               \begin{pairs}
                  \begin{Leftside}
			\beginnumbering
			\setcounter{stanzaL}{0}
                     
                        
                         \stanza 
                     lalal \&
                        
                        
                        
                        
                         \stanza 
                     —
                              Achilles
                              —
                              Astrologorum
                              signa
                              in
                              cælo
                              quid
                              sit
                              observationis
                              ,
                              cum
                              Capra
                              aut
                              Nepa
                              aut
                              exoritur
                              nomen
                              aliquod
                              belvarum
                              :
                              quod
                              est
                              ante
                              pedes
                              ,
                              nemo
                              spectat
                              ;
                              cæli
                              scrutantur
                              plagas
                              .
                              Quid
                              noctis
                              videtur
                              ?
                              —
                              in
                              altisono
                              cæli
                              clipeo
                              Temo
                              superat
                              stellas
                              sublimen
                              agens
                              etiam
                              atque
                              etiam
                              noctis
                              iter
                              .
                              —
                              Chorus
                              —
                              Otio
                              qui
                              nescit
                              uti
                              ,
                              plus
                              negoti
                              habet
                              ,
                              quam
                              cum
                              quis
                              negotium
                              in
                              negotio
                              .
                              Nam
                              cui
                              ,
                              quod
                              agat
                              institutum
                              est
                              ,
                              non
                              ullo
                              negotio
                              ;
                              id
                              agit
                              ,
                              id
                              studet
                              ,
                              ibi
                              mentem
                              atque
                              animum
                              delectat
                              suum
                              .
                              Otioso
                              in
                              otio
                              animus
                              nescit
                              quid
                              velit
                              .
                              Hoc
                              idem
                              est
                              :
                              em
                              neque
                              domi
                              nunc
                              nos
                              nec
                              militiæ
                              sumus
                              :
                              imus
                              huc
                              ,
                              hinc
                              illuc
                              ;
                              cum
                              illuc
                              ventum
                              est
                              ,
                              ire
                              illinc
                              lubet
                              .
                              Incerte
                              errat
                              animus
                              ,
                              præter
                              propter
                              vitam
                              vivitur
                              .
                              —
                              Iphigenia
                              —
                              Acherontem
                              obibo
                              ,
                              ubi
                              Mortis
                              thesauri
                              objacent
                              procede
                              :
                              gradum
                              proferre
                              pedum
                              nitere
                              :
                              cessas
                              ,
                              o
                              fide
                              Menelaus
                              me
                              objurgat
                              ?
                              id
                              meis
                              rebus
                              [
                              regimen
                              restitat
                              .
                              quæ
                              nunc
                              abs
                              te
                              viduæ
                              et
                              vastæ
                              virgines
                              sunt
                            \&
                     
                  \endnumbering
		\end{Leftside}
                  \begin{Rightside}
			\beginnumbering
			\numberstanzafalse
                     
                         \stanza 
                      \&
                     
                  \endnumbering
		\end{Rightside}
               \end{pairs}
	\Columns
            
         

         
            Lycurgus
            
               \textit{}
                  \textit{}.\par
               \par
            
            \begin{pairs}
               \begin{Leftside}
			\beginnumbering
			\setcounter{stanzaL}{0}
                  
                      \stanza 
                      \&
                  
               \endnumbering
		\end{Leftside}
               \begin{Rightside}
			\beginnumbering
			\numberstanzafalse
                  
                      \stanza 
                      \&
                  
               \endnumbering
		\end{Rightside}
            \end{pairs}
	\Columns
         



      \end{document}