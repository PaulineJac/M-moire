\documentclass[12pt,onecolumn,twoside,a4paper]{memoir}
\usepackage[no-math]{fontspec}
\usepackage{ebgaramond}
\usepackage{graphicx}
\usepackage{amssymb}
\usepackage{xcolor}
\usepackage{polyglossia}% typographie française
\setdefaultlanguage{french}
\usepackage{microtype}
\usepackage{xspace}% gérer les espaces
\usepackage{amssymb}% symboles utiles (math, etc.)
\usepackage{ulem} % souligner
\usepackage{alltt} % environnement "télétype"
\usepackage{setspace} % réglage des interlignes etc.
\usepackage{fancyhdr} % hauts et pieds de page
\usepackage{multicol} % texte sur plusieurs colonnes
\usepackage{fancybox} % boîtes améliorées
\usepackage{array} % tableaux
\usepackage{multirow} % les tableaux améliorés
\usepackage{url} % écrire les url
\usepackage{appendix} % annexes améliorées
\usepackage{geometry}
\usepackage{csquotes}
\usepackage{hyperref}
\usepackage{ragged2e}
\usepackage[backend=biber, bibstyle=verbose, backref=false, hyperref=false, citestyle=authortitle-ibid]{biblatex}
\addbibresource{/Users/Mac-Pauline/Documents/Texmf/bibtex/bib/Biblio.bib}
%%%%%%%%
\addto{\captionsfrench}{\renewcommand{\abstractname}{ }}
%%%%%%%%
%%%%%%%%%%%%%%%%
\usepackage{calc} 
\usepackage{fourier-orns} 
%%%% voir http://jacques-andre.fr/fontex/Fourier-orn.pdf
%%%%%%%
\pagestyle{fancyplain} \renewcommand{\chaptermark}[1]{% 
	\markboth{\chaptername\ \thechapter.\ #1}% 
	{\chaptername\ \thechapter.\ #1}} \renewcommand{\sectionmark}[1]{% 
	\markright{\thesection\ #1}}
% 
\lhead[\fancyplain{}{\bfseries\thepage}]%
{\fancyplain{}{\bfseries\nouppercase{\leftmark}}} 
\rhead[\fancyplain{}{\bfseries\nouppercase{\rightmark}}]%
{\fancyplain{}{\bfseries\thepage}} \fancyfoot{}
%%%%%%%%%%%%
%%%%%%%%%%%%
\newcommand{\phipaireblanche}
{\newpage{\pagestyle{empty}\cleardoublepage}} \newcommand{\phietc}{\textit{etc.}\xspace}
\newcommand{\phigg}[1]{\og #1 \fg}
%%%%%%%
\makeatletter
\newcommand{\finirphiimpaire}{\clearpage\if@twoside \ifodd\c@page
	\hbox{}\newpage\if@twocolumn\hbox{}\newpage\fi\fi\fi} \makeatother
\newcommand{\phiimpaireblanche}{% 
	\newpage{\pagestyle{empty}\finirphiimpaire}}
\renewcommand{\baselinestretch}{1,2}%%% interligne
\makeatother
%%%%%%%
%%%%%%
%Style Chapitre établi par Vincent Zoonekynd : http://zoonek.free.fr/LaTeX/LaTeX_samples_chapter/0.html
\makeatletter
\setlength\midchapskip{7pt}
\makechapterstyle{VZ21}{
	\renewcommand\chapnamefont{\Large\scshape}
	\renewcommand\chapnumfont{\Large\scshape\centering}
	\renewcommand\chaptitlefont{\huge\bfseries\centering}
	\renewcommand\printchaptertitle[1]{%
		\setlength\tabcolsep{7pt}% used as indentation on both sides
		\settowidth\@tempdimc{\chaptitlefont ##1}%
		\setlength\@tempdimc{\textwidth-\@tempdimc-2\tabcolsep}%
		\chaptitlefont
		\ifdim\@tempdimc > 0pt\relax% one line
		\begin{tabular}{c}
			\toprule  ##1\\ \bottomrule
		\end{tabular}
		\else% two+ lines
		\begin{tabular}{%
				>{\chaptitlefont\arraybackslash}p{\textwidth-2\tabcolsep}}
			\toprule ##1\\ \bottomrule
		\end{tabular}
		\fi
	}
}
%%%%%%%%%%%%%%
%%%%% Pour la traduction %%%%%%%%
\usepackage[ widthliketwocolumns,
nocritical,
noeledsec,
noend,
nofamiliar,
noledgroup,
series={}
]{reledmac}
\usepackage{reledpar}
%%%%%%%
\setcounter{stanzaindentsrepetition}{1}
\setstanzaindents{0,0,0}
\AtEveryStopStanza{\vspace{1\baselineskip}}
\numberstanzatrue
\renewcommand{\thestanzaL}{\MakeUppercase{\roman{stanzaL}}}
%%%%%
\firstlinenum*{100000}
%%%%%%%
\setlength{\Lcolwidth}{.450\textwidth}
\setlength{\Rcolwidth}{.450\textwidth}
\columnsposition{C}
\setlength{\beforecolumnseparator}{0.035\textwidth}
\setlength{\aftercolumnseparator}{0.0001\textwidth}
\sidenotemargin{left}
%%%%%%%%%%%%%%%%%
\title{\textit{Corpus sélectionné des fragments de tragédies de l'époque républicaine}}
\author{
	Pauline
	Jacsont
}
\date{\today}
\begin{document}
	
	
	\section{Livius Andronicus}
	Texte latin établi et organisé selon les choix de reconstruction et d'édition de \cite{TrRF_I_2012}.
	
	\subsection*{Égisthe}
	\begin{abstract}Source hellène : Sophocle \textit{Αἴγισθος}, Eschyle \textit{Όρεστεια}.
		L'intrigue est proche de l'\textit{Άγαμέμνων} d'Eschyle et
		de la version latine de Sénèque: elle se concentre donc sur l'assassinat
		d'Agamemnon par Égisthe et Clytemnestre.\par \end{abstract}
	\begin{pairs} 
		\begin{Leftside}
			\beginnumbering
			\setcounter{stanzaL}{0}
			
			\stanza Nam
			ut
			Pergama &
			accensa
			et
			præda
			per
			participes
			æquiter &
			partita
			est. \&
			\stanza Tum
			autem
			lasciuum
			Nerei
			simum
			pecus &
			ludens
			ad
			cantum
			classem
			lustratur... \&
			\stanza 
			Nemo
			hæce
			uostrum
			ruminetur
			mulieri. \&
			\stanza 
			...Sollemnitusque
			{Sollemnitusque}
			deo
			litat
			laudem
			lubens. \&
			\stanza ... in
			sedes
			conlocat
			se
			regias: &
			Clytemnestra
			iuxtim
			,
			tertias
			natæ
			occupant. \&
			\stanza 
			Ipsus
			se
			in
			terram
			saucius
			fligit
			cadens. \&
			\stanza Quin
			quod
			parere
			mihi
			uos
			maiestas
			mea &
			procat
			,
			toleratis
			temploque
			{temploque}
			hanc
			deducitis
			? \&
			\stanza 
			Iamne
			{Iamne}
			oculos
			specie
			lætauisti
			optabili
			? \&
			\endnumbering
		\end{Leftside}
		\begin{Rightside}
			\beginnumbering
			\numberstanzafalse
			
			\stanza En réalité, quand Pergame fût &enflammée on eût partagé équitablement &
			le butin entre les participants. \&
			\stanza Alors, voici la horde enjouée de Nérée, avec son bec aplati, &
			qui tourne autour de la flotte en s’amusant des chants (des
			marins)... \&
			\stanza 
			Que personne parmi vous ne rumine [ses pensées] à la femme. \&
			\stanza 
			Et la solennité extatique offre un éloge au dieu. \&
			\stanza . . .il s’assied sur le trône royal : &
			à côté Clytemnestre, ses filles occupent les troisièmes [places]. \&
			\stanza 
			Le blessé se heurte lui-même en tombant à terre. \&
			\stanza Comment? Vous ne supportez pas que ma majesté, &
			vous demande de m’obéir ? Pourquoi n’emmenez-vous pas celle-ci au
			sanctuaire ? \&
			\stanza 
			N’as-tu pas désormais contenté tes yeux par cette vision convoitée ? \&
			
			\endnumbering
		\end{Rightside}
	\end{pairs}
	\Columns
	
	
\end{document}